\documentclass{article}
\usepackage[left=3cm,right=3cm,top=2cm,bottom=2cm]{geometry} % page settings

\usepackage{amsfonts,amssymb}
\usepackage[utf8]{inputenc}
\usepackage[russian]{babel}
%\usepackage[dvips]{graphicx}
\usepackage{amsmath}
\usepackage{amsfonts}
\usepackage{amsthm}
\usepackage{MnSymbol}
\usepackage{tikz}
\usepackage{braket}

%\setlength{\parindent}{0mm}
\usepackage{graphicx}
\newcommand{\indep}{\rotatebox[origin=c]{90}{$\models$}}

\begin{document}

\title{Вопросы по курсу "Квантовая механика"}
\author{Н. Попов, М. Славошевский}
\date{\today}
\maketitle

\subsection*{Постулаты}
	\begin{enumerate}
		\item \textit{Как связаны между собой вектор кет $\ket{\psi}$ и вектор бра $\bra{\psi} $?}
		\begin{equation}
			\ket{\psi}=\bra{\psi}^{+}
		\end{equation}	
	\item \textit{Задано скалярное произведение двух векторов состояния $C =\braket{\varphi | \psi}$. Чему равно $\braket{\psi | \varphi}$?}
	\begin{equation}
		\braket{\psi|\phi}=C^{*}
	\end{equation}
	
	\item \textit{Пусть $\braket{\psi_1 | \psi_2} = 0$. Какой смысл имеют коэффициенты $c_1$ и $c_2$ в суперпозиции $\ket{\psi} = c_1 \ket{\psi_1} + c_2 \ket{\psi_2} $?}
	\begin{equation}
		c_{i}=\braket{\psi_{i}|\psi}
	\end{equation}
	\item \textit{Задан вектор состояния в виде суперпозиции двух состояний $\ket{\psi} = c_1 \ket{\psi_1} + c_2 \ket{\psi_2}$. Как
определяется вектор бра $\bra{\psi}$?}
	\begin{equation}
		\bra{\psi}=c_{1}^{*}\bra{\psi_{1}}+c_{2}^{*}\bra{\psi_{2}}
	\end{equation}
	\item \textit{Задан оператор физической величины $\hat{f}$. Как определяется наблюдаемая (физическая величина) квантовой системы, находящейся в состоянии $\ket{\psi}$?
}	
	\begin{equation}
		f=\bra{\psi}\hat{f}\ket{\psi}
	\end{equation}
	
	\end{enumerate}

\subsection*{Уравнение Шредингера}
\begin{enumerate}
	\item \textit{Записать уравнение, которому подчиняется вектор состояния квантовой системы.} 
	\begin{gather}
			i \hbar \frac{\partial \ket{\psi (t)}}{\partial t} = \hat{H} \ket{\psi (t)} \label{shrodingerEquation}\\
		\ket{\psi(0)} = \ket{\psi_0}
	\end{gather}
	где $\hat{H}$ - гамильтониан системы, $\ket{\psi_0}$ - начальное состояние.
	
	\item \textit{Записать стационарное уравнение Шредингера. Какой вид имеет оператор Гамильтона в общем случае?}
	
	Пусть собственные состояния оператора $\hat{H}$ задают базис пространства состояний, тогда любое состояние можно разложить по этому базису:
	\begin{equation}
		\ket{\psi(t)} = \sum\limits_n c_n (t) \ket{n} \label{basisDecomposition}
	\end{equation}
	Так как $\hat{H}\ket{n} = E_n \ket{n}$, то уравнение Шредингера перепишется в следующем виде:
	\begin{equation}
		\sum\limits_n \Big(i\hbar \frac{\partial c_n(t)}{\partial t} - c_n(t) E_n \Big) \ket{n} = 0
	\end{equation}
	Так как собственные вектора образуют базис, то в уравнении Шредингера все коэффициенты равны нулю, откуда получаем искомое состояние:
	\begin{equation}
		\ket{\psi(t)} = \sum\limits_n c_n^{(0)} e^{-\frac{i}{\hbar} E_n t} \ket{n}
	\end{equation}
	Уравнение, из которого определяются собственные состояния оператора Гамильтона, называется \textit{стационарное уравнение Шредингера}:
	\begin{equation}
		\hat{H} \ket{\psi} = E \ket{\psi}
	\end{equation}
	Для консервативной системы оператор Гамильтона имеет вид\footnote{В общем случае это не всегда верно. Из теоретической механики известно, что гамильтониан системы выражается через кинетическую и потенциальную энергию следующим образом:
	\begin{equation}
		H = T_2 + U - T_0 \label{trueHamultonian}
	\end{equation}
	где $T_2$ - квадратичная по скорости часть кинетической энергии, $T_0$ - не зависящая от скорость часть. Если же $\frac{\partial \textbf{r}}{\partial t} = 0$, то формула~\eqref{trueHamultonian} переходит в формулу~\eqref{falseHamultonian}
	}:
	\begin{equation}
		\hat{H} = \hat{T} + \hat{U} \label{falseHamultonian}
	\end{equation}
	где $\hat{T}$ - оператор кинетической энергии системы, $\hat{U}$ - потенциальной.
	\item \textit{Записать волновую функцию свободной нерелятивистской частицы.}
	
	Гамильтониан свободной нерелятивистской частицы имеет вид
	\begin{equation}
		\hat{H} = \frac{\textbf{p}^2}{2m} = -\frac{\hbar^2}{2m} \Delta
	\end{equation}
	в координатном представлении. Из линейности лапласиана следует, что решение стационарного уравнения можно искать в виде $\Psi(\textbf{r}) = \psi_x(x) \psi_y(y) \psi_z(z)$. Подставим такую волновую функцию:
	\begin{equation}
		\frac{d^2 \psi_x(x)}{d x^2}\psi_y(y)\psi_z(z) + \psi_x(x)\frac{d^2 \psi_y(y)}{d y^2}\psi_z(z) + \psi_x(x)\psi_y(y)\frac{d^2 \psi_z(z)}{d z^2} = -\frac{2mE}{\hbar^2} \psi_x(x)\psi_y(y)\psi_z(z) \label{freeParticle}
	\end{equation}
	Поделим равенство~\eqref{freeParticle} на волновую функцию:
	\begin{equation}
		\frac{d^2 \psi_x(x)}{d x^2} \frac{1}{\psi_x(x)} + \frac{d^2 \psi_y(y)}{d y^2} \frac{1}{\psi_y(y)} + \frac{d^2 \psi_z(z)}{d z^2} \frac{1}{\psi_z(z)} = -\frac{2mE}{\hbar^2}
	\end{equation}
	Получилась сумма функций, каждая из которой зависит от отдной из пространственных переменных, причем их сумма равна константе. Следовательно, каждая из этох функций постоянна. Из физических соображений очевидно, что кинетическая энергия частицы не может быть отрицательной. Обозначим $k_\alpha^2 =\frac{2m E_\alpha}{\hbar^2}, E = E_x + E_y + E_z$:
	\begin{equation}
		\psi_\alpha''(\alpha) + k^2_\alpha \psi_\alpha(\alpha) = 0
	\end{equation}
	Решением этого уравнения является
	\begin{equation}
		\psi_\alpha(\alpha) = C_1 e^{i k_\alpha \alpha} + C_2 e^{-i k_\alpha \alpha}
	\end{equation}
	Физический смысл каждой из двух функций в том, что первая описывает движение частицы в $+\infty$, а вторая - в противоположном направлении. Полная волновая функция $\Psi(\textbf{r})$ равна:
	\begin{equation}
		\Psi(\textbf{r}) = C_1 e^{i\textbf{kr}} + C_2 e^{-i\textbf{kr}}
	\end{equation}
	
	\item \textit{Как с помощью оператора эволюции записать решение уравнения Шредингера в произвольный момент времени, если задано начальное условие $\Psi(\textbf{r}, 0) = \psi_0(\textbf{r}) $?}
	
	От дифференциального вида уравнения Шредингера можно перейти к интегральному:
	\begin{equation}
		\Psi(\textbf{r}, t) = \psi_0(\textbf{r}) - \frac{i}{\hbar} \int\limits_0^t \hat{H} \Psi(\textbf{r}, \tau) d\tau
	\end{equation}
	Пусть гамильтониан системы не зависит от времени. Будем решать уравнение методом последовательных приближений. Пусть $\Psi^{(0)}(\textbf{r}, t) = \psi_0 (\textbf{r})$, тогда
	\begin{equation}
		\Psi^{(1)}(\textbf{r}, t) = \psi_0 (\textbf{r}) - \frac{it\hat{H} }{\hbar} \psi_0 (\textbf{r})
	\end{equation}
	Повторяя приближение бесконечное число раз, в пределе получим:
	\begin{equation}
		\Psi(\textbf{r}, t) = \sum\limits_{n=0}^\infty  \Big(\frac{-it\hat{H} }{\hbar} \Big)^n \psi_0 (\textbf{r}) = e^{\frac{-it\hat{H} }{\hbar}} \psi_0 (\textbf{r}) = \hat{U}(t) \psi_0 (\textbf{r})
	\end{equation}
	где введен оператор эволюции $\hat{U}(t) = e^{\frac{-it\hat{H} }{\hbar}}$
	
	\item \textit{
	 Какой вид имеет оператор эволюции консервативной системы?
	}
	\begin{equation}
		\hat{U}(t) = e^{\frac{-it\hat{H} }{\hbar}}
	\end{equation}
	
	\item \textit{
	Записать определение производной оператора по времени.
	}
	
	Производной физической величины можно сопоставить производную оператора, который задает эту величину. Найдем его явный вид:
	\begin{equation}
		\frac{df}{dt} = \frac{d}{dt} \braket{\psi(t) | \hat{f} | \psi(t)} = \frac{d}{dt} \braket{\psi_0 |\hat{U}^+ \hat{f} \hat{U} | \psi_0} = \bra{\psi_0} \frac{\partial \hat{U}^+}{\partial t} \hat{f} \hat{U} + \hat{U}^+\frac{\partial \hat{f}}{\partial t} \hat{U} + \hat{U}^+ \hat{f}\frac{\partial \hat{U}}{\partial t} \ket{\psi_0} \label{derivativeFirst}
	\end{equation}
	По определению оператора эволюции его производная равна
	\begin{equation}
		\frac{\partial U}{\partial t} = -\frac{i}{\hbar} \hat{H} \hat{U}
	\end{equation}
	Тогда равенство~\eqref{derivativeFirst} можно продолжить:
	\begin{equation}
		\frac{df}{dt} = \bra{\psi_0} \hat{U}^+ \Big( \frac{\partial \hat{f}}{\partial t} + \frac{i}{\hbar} \big( \hat{H} \hat{f} - \hat{f} \hat{H} \big) \Big)\hat{U}\ket{\psi_0} = \bra{\psi(t)}\frac{\partial \hat{f}}{\partial t} + \frac{i}{\hbar} \Big[ \hat{H}, \hat{f}\Big] \ket{\psi(t)}
	\end{equation}
	где квадратные скобки обзначают \textit{коммутатор} двух операторов. Оператор, который находится между векторами бра и кет, по определению называют \textit{производной оператора по времени}:
	\begin{equation}
	\frac{d\hat{f}}{dt} = \frac{\partial \hat{f}}{\partial t} + \frac{i}{\hbar} \Big[ \hat{H}, \hat{f}\Big] \label{derivativeLast}
	\end{equation}
	\item \textit{
	Как определяется коммутатор двух операторов?	
	}
	\begin{equation}
		\Big[ \hat{A}, \hat{B} \Big] = \hat{A} \hat{B} - \hat{B} \hat{A}
	\end{equation}
	\item \textit{
 Какому условию удовлетворяют операторы физических величин–интегралов движения в квантовой механике?	
	}
	
	Интеграл движения - величина, сохранающаяся с течением времени. Так как мы знаем оператор, который сопоставляется производной физической величины (формула~\eqref{derivativeLast}), то для интегралов движения этот оператор должен равняться нулю:
	\begin{equation}
		\frac{\partial \hat{f}}{\partial t} + \frac{i}{\hbar} \Big[ \hat{H}, \hat{f}\Big] = 0
	\end{equation}
	\item \textit{
	Какие физические величины могут быть включены в полный набор физических величин, определяющих состояние квантовой системы?	
	}
	
	Полный набор физических величин - это такой набор величин, что каждому уникальному набору значений этих величин соответствует вектор состояния из базиса пространства состояний. Другими словами, это тот набор величин, по которому можно построить базис в пространстве состояний рассматриваемой системы. Так как для решения уравнения Шредингера~\eqref{shrodingerEquation} состояние раскладывают по собственным векторам оператора Гамильтона, то обычно в полный набор включают энергию системы. Однако не всегда энергетический спектр задает базис, то есть одному значению энергии соответствует несколько базисных векторов. Однако для двух коммутирующих операторов можно выбрать общий базис собственных функций. Поэтому в полный набор физических величин можно включать величины, которые коммутируют с оператором Гамильтона.
	
	\item \textit{
		Как можно представить коммутатор $\Big[ \hat{A} \hat{B}, \hat{C}\Big]$?	
	}
	
	\begin{equation}
		\Big[ \hat{A} \hat{B}, \hat{C}\Big] = \hat{A}\hat{B}\hat{C} - \hat{C}\hat{A}\hat{B} \pm \hat{A} \hat{C} \hat{B} = \hat{A}(\hat{B}\hat{C} - \hat{C} \hat{B}) + (\hat{A}\hat{C} - \hat{C} \hat{A}) \hat{B} = \hat{A}\Big[ \hat{B}, \hat{C}\Big] + \Big[ \hat{A}, \hat{C}\Big] \hat{B} \label{abc}
	\end{equation}
	
	\item \textit{Определить полный набор физических величин свободной бесспиновой частицы.}

\end{enumerate}

\subsection*{Операторы и теория представлений}

\begin{enumerate}
	\item \textit{Чему равен (“табличный”) коммутатор $\big[\hat{x}_\alpha, \hat{p}_\beta \big]$?}
	
	Коммутаторы операторов физических величин постулируются - для двух физических случайных величин коммутатор пропорционален скобке Пуассона этих двух величин, в которых физические величины заменены на их операторы:
	\begin{equation}
		\Big\{ f, g \Big\} \rightarrow -\frac{i}{\hbar} \Big[\hat{f}, \hat{g}\Big]
	\end{equation}
	По определению, скобка Пуассона равна:
	\begin{equation}
		\Big\{ f, g \Big\} = \sum\limits_n \frac{\partial f}{\partial q_n} \frac{\partial g}{\partial p_n} - \frac{\partial f}{\partial p_n} \frac{\partial g}{\partial q_n}
	\end{equation}
	где $q_n, p_n$ - обобщенные координатв и импульс соответственно. Для самих координаты и импульса скобка Пуассона равна:
	\begin{equation}
		\Big\{ x_\alpha, p_\beta \Big\} = \delta_{\alpha \beta}
	\end{equation}
	и, следовательно коммутатор их операторов равен:
	\begin{equation}
		\big[\hat{x}_\alpha, \hat{p}_\beta \big] = i\hbar \delta_{\alpha\beta}
	\end{equation}
	
	\item \textit{Зная табличный коммутатор операторов координаты и импульса, определить оператор скорости нерелятивистской частицы.}
	
	Воспользуемся определением производной оператора по времени:
	\begin{equation}
		\hat{\textbf{v}} = \frac{d \textbf{r}}{dt} = \frac{\partial \textbf{r}}{\partial t} + \frac{i}{\hbar} \Big[ \hat{H}, \hat{\textbf{r}} \Big]
	\end{equation}
	Так как потенциальная энергия является функцией координаты, то её оператор коммутирует с оператором координаты. Посчитаем коммутатор оператора кинетической энергии с оператором координаты:
	\begin{equation}
		\Big[ \hat{T}, \hat{\textbf{r}} \Big] = \frac{1}{2m} \Big[ \hat{\textbf{p}}^2, \hat{\textbf{r}} \Big] \stackrel{~\eqref{abc}}{=} \frac{1}{2m} \Big(\hat{p}_\alpha \Big[\hat{p}_\alpha, \hat{\textbf{r}}\Big] + \Big[\hat{p}_\alpha, \hat{\textbf{r}}\Big]\hat{p}_\alpha \Big) = -\frac{i\hbar\hat{\textbf{p}}}{m}
	\end{equation}
	И тогда оператор скорости, как и в классическом случае, равен
	\begin{equation}
		\hat{\textbf{v}} = \frac{\hat{\textbf{p}}}{m}
	\end{equation}
	\item \textit{Как можно записать оператор, соответствующий физической величине \textbf{pr}?}
	
	Физическим величинам соответствуют эрмитовые операторы. Попробуем сопоставить величине $\varphi = p_\alpha x_\alpha$ оператор напрямую, и посчитаем эрмитово сопряженный оператор $\hat{\varphi}^+$:
	\begin{equation}
		\varphi^+ = (\hat{p}_\alpha \hat{x}_\alpha)^+ = \hat{x}_\alpha \hat{p}_\alpha = \hat{p}_\alpha \hat{x}_\alpha + \big[\hat{p}_\alpha, \hat{x}_\alpha \big] = \hat{p}_\alpha \hat{x}_\alpha - i\hbar \delta_{\alpha\alpha} = \hat{p}_\alpha \hat{x}_\alpha - 3i\hbar
	\end{equation}
	Как видно, напрямую сопоставленный оператор неэрмитов; однако, если взять оператор $\hat{\varphi} = \hat{p}_\alpha \hat{x}_\alpha - \frac{3}{2}i\hbar$, то такой оператор является эрмитовым. При переходе к классической механике ($\hbar \to 0$) введенный оператор переходит в величину $\varphi$, следовательно, введенный таким образом оператор соответствует заданной величине.
	
	\item \textit{Что определяет выражение }$\braket{\textbf{r}|\Psi}$\textit{ = ?}
	
	Функция $\psi(\textbf{r}) = \braket{\textbf{r}|\Psi}$ (которую называют \textit{волновой функцией}) является проекцией состояния $\ket{\Psi}$ на базис собственных состояний оператора $\hat{\textbf{r}}$; другими словами, это состояние $\ket{\psi}$ в координатном представлении. Квадрат модуля этой функции есть плотность вероятности того, что частица находится в точке $\textbf{r}$.
	
	\item \textit{Для оператора координаты $\hat{\textbf{r} } \ket{\textbf{r}_0} = \ ?$}
	
	По определению, состояние $\ket{\textbf{r}_0}$ - состояние с определенной координатой, то есть это собственное состояние оператора $\hat{\textbf{r}}$. Следовательно,
	\begin{equation}
		\hat{\textbf{r}}\ket{\textbf{r}_0} = \textbf{r}_0 \ket{\textbf{r}_0}
	\end{equation}
	
	\item \textit{Для оператора импульса  $\hat{\textbf{p} } \ket{\textbf{p}_0} = \ ?$}
	Аналогично оператору координаты:
	\begin{equation}
		\hat{\textbf{p}}\ket{\textbf{p}_0} = \textbf{p}_0 \ket{\textbf{p}_0}
	\end{equation}
	
	\item \textit{Пусть совокупность векторов $\ket{n}$ -составляет базис. Чему равен оператор $\sum\limits_n \ket{n}\bra{n} = \ ?$}
	
	Так как вектора $\ket{n}$ состовляют базис, то любое состояние можно разложить по этому базису (смотри, например, формулу~\eqref{basisDecomposition}). Посмотрим, как этот оператор действует на произвольное состояние:
	\begin{equation}
		\sum\limits_n \big(\ket{n}\bra{n}\big) \ket{\psi} = \sum\limits_{n, n'} \big(\ket{n}\bra{n}\big) c_{n'}\ket{n'} = \sum\limits_{n, n'} c_{n'}\ket{n}\braket{n|n'} = \sum\limits_{n, n'} c_{n'}\ket{n}\delta_{nn'} = \sum\limits_{n,} c_{n}\ket{n} = \ket{\psi}
	\end{equation}
	то есть оператор не меняет произвольное состояние. Значит, этот оператор равен тождественному оператору $\hat{1}$.
	
	\item \textit{Для некоторого (не обязательно эрмитова) оператора $\hat{f}\ket{f_n} = f_n\ket{f_n}$, чему равно $\bra{f_n}\hat{f} = \ ?$}
	
	Пусть $\bra{f_n}\hat{f} = \bra{\psi}$, спроецируем $\bra{\psi}$ на базисные вектора $\bra{f_{n'}}$:
	\begin{equation}
		\braket{\psi|f_{n'}} = \braket{f_n|\hat{f}|f_{n'}} = f_{n'} \braket{f_n|f_{n'}} = f_{n'} \delta_{nn'}
	\end{equation}
	то есть проекции на вектора, отличные от $\bra{f_n}$, равны нулю. Значит, $\bra{\psi} = f_n\bra{f_n}$
	
	\item \textit{Пусть $\hat{f}^+=\hat{f}$, а $\hat{f}\ket{f_n} = f_n \ket{f_n}$, чему равен оператор $\sum\limits_n \ket{f_n}\bra{f_n} = \ ?$}
	
	У эрмитового оператора всегда существует базис из собственных векторов, следовательно, оператор равен тождетвенному (смотри вопрос 7).
	
	\item $\braket{\textbf{r}'|\hat{\textbf{r}}|\textbf{r}} = \ ?$
	\begin{equation}
		\braket{\textbf{r}'|\hat{\textbf{r}}|\textbf{r}} = \textbf{r} \braket{\textbf{r}'|\textbf{r}} =\textbf{r} \delta(\textbf{r}' - \textbf{r})
	\end{equation}
	где $\delta(\textbf{r})$ - дельта-функция Дирака.
	
	\item $\braket{\textbf{r}|\textbf{p}} = \ ?$
	
	Это значение есть ни что иное, как волновая функция состояния с определенным импульсом. Задача решается при помощи некоторого искусственого приема. Введем оператор
	\begin{equation}
		\hat{Q}_{\textbf{a}} = e^{-\frac{i}{\hbar}\textbf{a}\hat{\textbf{p}}}
	\end{equation}
	Этот оператор является функцией оператора координаты. Нам нужно вычислить его коммутатор с оператором координаты. Для этого вычислим коммутатор $\big[\hat{x}_\alpha,\hat{p}_\alpha^l \big]$ при помощи скобок Пуассона:
	\begin{equation}
		\Big\{x, p_x \Big\} = \frac{\partial x}{\partial x} \frac{\partial p_x^l}{\partial p_x} = \frac{\partial p_x^l}{\partial p_x}
	\end{equation}
	Следовательно, коммутатор $\big[\hat{x}_\alpha,\hat{p}_\alpha^l \big]$ равен:
	\begin{equation}
		\big[\hat{x}_\alpha,\hat{p}_\alpha^l \big] = i\hbar \frac{\partial \hat{\textbf{p}}^l}{\partial \hat{\textbf{p}}}
	\end{equation}
	И коммутатор $\hat{Q}_\textbf{a}$ с оператором координаты, по определению функции от оператора, равен
	\begin{equation}
		\Big[\hat{\textbf{r}},\hat{Q}_\textbf{a} \Big] = i\hbar \frac{\partial \hat{Q}_\textbf{a}}{\partial \hat{\textbf{p}}} = \textbf{a}\hat{Q}_\textbf{a}
	\end{equation}
	Теперь подействуем оператором $\hat{\textbf{r}}\hat{Q}_\textbf{a}$ на состояние с определенной координатой:
	\begin{equation}
		\hat{\textbf{r}}\hat{Q}_\textbf{a} \ket{\textbf{r}_0} = \Big(\hat{Q}_\textbf{a}\hat{\textbf{r}} + \hat{\textbf{a}}\hat{Q}_\textbf{a}\Big)\ket{\textbf{r}_0} = (\textbf{r}_0 + \textbf{a})\hat{Q}_\textbf{a} \ket{\textbf{r}_0}
	\end{equation}
	То есть вектор $\hat{Q}_\textbf{a} \ket{\textbf{r}_0}$ есть собственный вектор оператора координаты. Очевидно, что если взять в качестве вектора $\textbf{a}$ необходимую координату, и действовать на состояние с нулевой координатой, то мы получим состояние с необходимой координатой. Следовательно, оператор $\hat{Q}_\textbf{a}$ есть оператор трансляции. Аналогичным образом можно определить оператор трансляции $\hat{P}_\textbf{k}$ для импульсного представления. Тогда
	\begin{equation}
		\braket{\textbf{r}|\textbf{p}} = \braket{\textbf{r}|\hat{P}_\textbf{p}|0}_p = e^{-\frac{i}{\hbar}\textbf{pr}} \braket{\textbf{r}|0}_p = {e^{-\frac{i}{\hbar}\textbf{pr}}}_r \braket{0|\hat{Q}_\textbf{r}^+|0}_p = {e^{-\frac{i}{\hbar}\textbf{pr}}}_r \braket{0|0}_p \label{rp}
	\end{equation}
	Остается только определить константу $_r\braket{0|0}_p$. Для этого можно сделать следующее:
	\begin{equation}
		\braket{\textbf{p}|\textbf{p}'} = \delta(\textbf{p} - \textbf{p}') = \int d\textbf{r} \braket{\textbf{p}|\textbf{r}}\braket{\textbf{r}|\textbf{p}'}
	\end{equation}
	Подставляя в интеграл значение~\eqref{rp}, получаем:
	\begin{equation}
		|_r\braket{0|0}_p|^2 \int d\textbf{r} e^{\frac{i}{h}(\textbf{p}' - \textbf{p})\textbf{r}} = \delta(\textbf{p} - \textbf{p}')
	\end{equation}
	Интеграл в равенстве пропорционален дельта-функции, значит, искомый коэффициент равен:
	\begin{equation}
		|_r\braket{0|0}_p|^2 = (2\pi\hbar)^{-3}
	\end{equation}
	Тогда искомое выражение равно:
	\begin{equation}
		\braket{\textbf{r}|\textbf{p}} = \frac{1}{(2\pi\hbar)^{\frac{3}{2}}} e^{\frac{i}{\hbar}\textbf{pr}}
	\end{equation}
	
	\item $\braket{\textbf{p}|\textbf{r}} = \ ?$
		
	По определению,
	\begin{equation}
		\braket{\textbf{p}|\textbf{r}} = (\braket{\textbf{r}|\textbf{p}})^* = \frac{1}{(2\pi\hbar)^{\frac{3}{2}}} e^{-\frac{i}{\hbar}\textbf{pr}}
	\end{equation}
	
	\item \textit{Записать волновую функцию свободной нерелятивистской частицы.}
	
	Смотри вопрос №3 раздела "Уравнение Шредингера"
	\item $\braket{\textbf{r}'|\textbf{r}} = \ ?$
	
	Векторы $\ket{\textbf{r}}$ - базисные вектора, нормированные на дельта-функцию:
	\begin{equation}
		\braket{\textbf{r}'|\textbf{r}} = \delta(\textbf{r}' - \textbf{r})
	\end{equation}
	
	\item $\braket{\textbf{p}'|\textbf{p}} = \ ?$
	
	Аналогично состояниям с определенной координатой:
	\begin{equation}
		\braket{\textbf{p}'|\textbf{p}} = \delta(\textbf{p}' - \textbf{p})	
	\end{equation}
	
	\item \textit{Записать уравнение Шредингера для частицы с массой $m$ в координатном представлении.}
	
	Для того, чтобы перейти в координатное представление, нам нужно выяснить, как действуют операторы координаты и импульса на волновую функцию. Рассмотрим действие оператора координаты: пусть $\ket{\varphi} = \hat{\textbf{r}}\psi$, тогда
	\begin{equation}
		\braket{\textbf{r}|\varphi} = \braket{\textbf{r}|\hat{\textbf{r}}|\psi} = \braket{\textbf{r}|\hat{\textbf{r}}\hat{1}|\psi} = \int d\textbf{r}' \braket{\textbf{r}|\hat{\textbf{r}}|\textbf{r}'} \braket{\textbf{r}'|\psi} = \int d\textbf{r}' \textbf{r}' \delta(\textbf{r} - \textbf{r}') \psi(\textbf{r}') = \textbf{r}\psi(\textbf{r})
	\end{equation}
	Таким образом, действие оператора координаты на волновую функцию аналогично действию оператора координаты на состояние:
	\begin{equation}
		\hat{\textbf{r}}\psi(\textbf{r}) = \textbf{r} \psi(\textbf{r})
	\end{equation}
	Теперь выясним действие оператора импульса на волновую функцию. Пусть $\ket{\phi} = \hat{\textbf{p}}\ket{\psi}$. Тогда
	\begin{equation}
		\braket{\textbf{r}|\phi} = \braket{\textbf{r}|\hat{\textbf{p}}\hat{1}|\psi} = \int d\textbf{r}' \braket{\textbf{r}|\hat{\textbf{p}}|\textbf{r}'} \psi(\textbf{r}') \label{imulseFirst}
	\end{equation}
	Найдем матричный элемент оператора $\hat{\textbf{p}}$ в координатном предствалении:
	\begin{equation}
	\begin{split}
		\braket{\textbf{r}|\hat{\textbf{p}}|\textbf{r}'} = \braket{\textbf{r}|\hat{1}\hat{\textbf{p}}\hat{1}|\textbf{r}'} = \int d\textbf{p}d\textbf{p}'\braket{\textbf{r}|\textbf{p}} \braket{\textbf{p}|\hat{\textbf{p}}|\textbf{p}'} \braket{\textbf{p}'|\textbf{r}'} = \int d\textbf{p} \textbf{p} \braket{\textbf{r}|\textbf{p}} \braket{\textbf{p}|\textbf{r}'} =\\= \frac{1}{(2\pi\hbar)^3} \int d\textbf{p} \textbf{p} e^{\frac{i}{\hbar}\textbf{p}(\textbf{r} - \textbf{r}')} = i\hbar \frac{\partial}{\partial \textbf{r}'}\Big(\frac{1}{(2\pi\hbar)^3}\int d\textbf{p} e^{\frac{i}{\hbar}\textbf{p}(\textbf{r} - \textbf{r}')}  \Big) = i\hbar \frac{\partial}{\partial \textbf{r}'} \delta(\textbf{r} - \textbf{r}')
		\end{split}
	\end{equation}
	Подставим получившееся выражение в~\eqref{imulseFirst}:
	\begin{equation}
		\braket{\textbf{r}|\phi} = i\hbar \int d\textbf{r}' \Big(\frac{\partial}{\partial \textbf{r}'} \delta(\textbf{r} - \textbf{r}')\Big) \psi(\textbf{r}') = - i\hbar \int d\textbf{r}' \delta(\textbf{r} - \textbf{r}') \frac{\partial \psi}{\partial \textbf{r}'} = - i\hbar \frac{\partial \psi}{\partial \textbf{r}}
	\end{equation}
	Следовательно, действие оператора импульса на волновую функцию сводится к дифференцированию:
	\begin{equation}
		\hat{\textbf{p}}\psi(\textbf{r}) = -i\hbar \frac{\partial \psi}{\partial \textbf{r}}
	\end{equation}
	Теперь можно легко записать уравнение Шредингера в координатном представлении:
	\begin{equation}
		i\hbar \frac{\partial \psi}{\partial t} = -\frac{\hbar^2}{2m} \Delta \psi + U(\textbf{r})\psi
	\end{equation}
	
	\item \textit{Записать стационарное уравнение Шредингера для частицы с массой m в импульсном представлении.}
	
	Аналогично предыдущему пункту легко найти, что оператор координаты в импульсном представлении выглядит следующим образом:
	\begin{equation}
		\hat{\textbf{r}} = i\hbar \nabla
	\end{equation}
	Тогда стационарное уравнение Шредингера в импульсном представлении запишется следующим образом:
	\begin{equation}
		\frac{\textbf{p}^2}{2m} \psi_\textbf{p} + U(i\hbar\nabla)\psi_\textbf{p} = E \psi_\textbf{p}
	\end{equation}
\end{enumerate}

\subsection*{Основные коммутационные соотношения}
\begin{enumerate}
	\item \textit{Чему равен коммутатор $[x,\hat{p}_{x}]$?}
	\begin{equation}
		[x,\hat{p}_{x}]=i\hbar
	\end{equation}
	\item \textit{Чему равен коммутатор $[\hat{x}_{\alpha},\hat{p}_{\beta}]$?}
	\begin{equation}
		[\hat{x}_{\alpha},\hat{p}_{\beta}]=i\hbar\delta_{\alpha\beta}
	\end{equation}
	\item \textit{Чему равен коммутатор $[\hat{\boldsymbol{p}},U(\boldsymbol{r})]$?}
	\begin{equation}
		[\hat{\boldsymbol{p}},U(\boldsymbol{r})]=i\hbar\nabla U(\boldsymbol{r})
	\end{equation}
	\item \textit{Чему равен коммутатор $[\hat{l}_{x},\hat{l}_{y}]$?}
	\begin{equation}
		[\hat{l}_{x},\hat{l}_{y}]=i\hat{l}_{z}
	\end{equation}
	\item \textit{Чему равен коммутатор $[\hat{l}_{\alpha},\hat{l}_{\beta}]$?}
	\begin{equation}
		[\hat{l}_{\alpha},\hat{l}_{\beta}]=ie_{\alpha\beta\gamma}\hat{l}_{\gamma}
	\end{equation}
	\item \textit{Чему равен коммутатор $[\hat{l}_{\alpha},\hat{\boldsymbol{l}}^{2}]$?}
	\begin{equation}
		[\hat{l}_{\alpha},\hat{\boldsymbol{l}}^{2}]=0
	\end{equation}
	\item \textit{Чему равен коммутатор $[\hat{l}_{\text{z}},\hat{l}_{+}]$?}
	\begin{equation}
		[\hat{l}_{\text{z}},\hat{l}_{+}]=\hat{l}_{+}
	\end{equation}
	\item \textit{Чему равен коммутатор $[\hat{l}_{\text{z}},\hat{l}_{-}]$?}
	\begin{equation}
		[\hat{l}_{\text{z}},\hat{l}_{-}]=-\hat{l}_{-}
	\end{equation}
\end{enumerate}

\subsection*{Момент}
	\begin{enumerate}
		\item \textit{Выразить $\hat{\textbf{l}}^2$ через $\hat{l}_z$ и $\hat{l}_\pm$.}	
		\begin{equation}
			\hat{\boldsymbol{l^{2}}}=\hat{l_{z}^{2}}+\frac{1}{2}(\hat{l_{+}}\hat{l_{-}}+\hat{l_{-}}\hat{l_{+}})=\hat{l_{z}}^{2}+\hat{l_{z}}+\hat{l_{-}}\hat{l_{+}}=\hat{l_{z}}^{2}-\hat{l_{z}}+\hat{l_{+}}\hat{l_{-}}
		\end{equation}
		\item \textit{$\hat{\textbf{l}}^2 \ket{l, m} = \ ?$}
		\begin{equation}
			\hat{\boldsymbol{l}}^2\ket{l,m}=l(l+1)\ket{l,m}
		\end{equation}			
		\item \textit{$\hat{l}_z \ket{l, m} = \ ?$}
		\begin{equation}
			\hat{l}_z\ket{l,m}=m\ket{l,m}
		\end{equation}	
		\item \textit{$\hat{l}_+ \ket{l, l} = \ ?$}
		\begin{equation}
			\hat{l}_+\ket{l,l}=0
		\end{equation}	
		\item \textit{$\hat{l}_- \ket{l, -l} = \ ?$}
		\begin{equation}
			\hat{l}_-\ket{l,-l}=0
		\end{equation}		
		\item \textit{$\hat{l}_\pm \ket{l, m} = \ ?$}
		\begin{equation}
			\hat{l}_\pm\ket{l,m}=\sqrt{(l\mp m)(l\pm m-1)}\ket{l,m\pm1}
		\end{equation}	
	\end{enumerate}

\end{document}