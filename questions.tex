\documentclass{article}
\usepackage[left=3cm,right=3cm,top=2cm,bottom=2cm]{geometry} % page settings

\usepackage{amsfonts,amssymb}
\usepackage[utf8]{inputenc}
\usepackage[russian]{babel}
%\usepackage[dvips]{graphicx}
\usepackage{amsmath}
\usepackage{amsfonts}
\usepackage{amsthm}
%\usepackage{MnSymbol}
\usepackage{tikz}
\usepackage{braket}

%\setlength{\parindent}{0mm}
\usepackage{graphicx}
\newcommand{\indep}{\rotatebox[origin=c]{90}{$\models$}}

\begin{document}

\title{Вопросы по курсу "Квантовая механика"}
\author{Н. Попов, М. Славошевский}
\date{\today}
\maketitle

{\center{\section*{Квантовая механика, часть 1}}}

\subsection*{Постулаты}
	\begin{enumerate}
		\item \textit{Как связаны между собой вектор кет $\ket{\psi}$ и вектор бра $\bra{\psi} $?}
		\begin{equation}
			\ket{\psi}=\bra{\psi}^{+}
		\end{equation}	
	\item \textit{Задано скалярное произведение двух векторов состояния $C =\braket{\varphi | \psi}$. Чему равно $\braket{\psi | \varphi}$?}
	\begin{equation}
		\braket{\psi|\phi}=C^{*}
	\end{equation}
	
	\item \textit{Пусть $\braket{\psi_1 | \psi_2} = 0$. Какой смысл имеют коэффициенты $c_1$ и $c_2$ в суперпозиции $\ket{\psi} = c_1 \ket{\psi_1} + c_2 \ket{\psi_2} $?}
	\begin{equation}
		c_{i}=\braket{\psi_{i}|\psi}
	\end{equation}
	\item \textit{Задан вектор состояния в виде суперпозиции двух состояний $\ket{\psi} = c_1 \ket{\psi_1} + c_2 \ket{\psi_2}$. Как
определяется вектор бра $\bra{\psi}$?}
	\begin{equation}
		\bra{\psi}=c_{1}^{*}\bra{\psi_{1}}+c_{2}^{*}\bra{\psi_{2}}
	\end{equation}
	\item \textit{Задан оператор физической величины $\hat{f}$. Как определяется наблюдаемая (физическая величина) квантовой системы, находящейся в состоянии $\ket{\psi}$?
}	
	\begin{equation}
		f=\bra{\psi}\hat{f}\ket{\psi}
	\end{equation}
	
	\end{enumerate}

\subsection*{Уравнение Шредингера}
\begin{enumerate}
	\item \textit{Записать уравнение, которому подчиняется вектор состояния квантовой системы.} 
	\begin{gather}
			i \hbar \frac{\partial \ket{\psi (t)}}{\partial t} = \hat{H} \ket{\psi (t)} \label{shrodingerEquation}\\
		\ket{\psi(0)} = \ket{\psi_0}
	\end{gather}
	где $\hat{H}$ - гамильтониан системы, $\ket{\psi_0}$ - начальное состояние.
	
	\item \textit{Записать стационарное уравнение Шредингера. Какой вид имеет оператор Гамильтона в общем случае?}
	
	Пусть собственные состояния оператора $\hat{H}$ задают базис пространства состояний, тогда любое состояние можно разложить по этому базису:
	\begin{equation}
		\ket{\psi(t)} = \sum\limits_n c_n (t) \ket{n} \label{basisDecomposition}
	\end{equation}
	Так как $\hat{H}\ket{n} = E_n \ket{n}$, то уравнение Шредингера перепишется в следующем виде:
	\begin{equation}
		\sum\limits_n \Big(i\hbar \frac{\partial c_n(t)}{\partial t} - c_n(t) E_n \Big) \ket{n} = 0
	\end{equation}
	Так как собственные вектора образуют базис, то в уравнении Шредингера все коэффициенты равны нулю, откуда получаем искомое состояние:
	\begin{equation}
		\ket{\psi(t)} = \sum\limits_n c_n^{(0)} e^{-\frac{i}{\hbar} E_n t} \ket{n}
	\end{equation}
	Уравнение, из которого определяются собственные состояния оператора Гамильтона, называется \textit{стационарное уравнение Шредингера}:
	\begin{equation}
		\hat{H} \ket{\psi} = E \ket{\psi}
	\end{equation}
	Для консервативной системы оператор Гамильтона имеет вид\footnote{В общем случае это не всегда верно. Из теоретической механики известно, что гамильтониан системы выражается через кинетическую и потенциальную энергию следующим образом:
	\begin{equation}
		H = T_2 + U - T_0 \label{trueHamultonian}
	\end{equation}
	где $T_2$ - квадратичная по скорости часть кинетической энергии, $T_0$ - не зависящая от скорость часть. Если же $\frac{\partial \textbf{r}}{\partial t} = 0$, то формула~\eqref{trueHamultonian} переходит в формулу~\eqref{falseHamultonian}
	}:
	\begin{equation}
		\hat{H} = \hat{T} + \hat{U} \label{falseHamultonian}
	\end{equation}
	где $\hat{T}$ - оператор кинетической энергии системы, $\hat{U}$ - потенциальной.
	\item \textit{Записать волновую функцию свободной нерелятивистской частицы.}
	
	Гамильтониан свободной нерелятивистской частицы имеет вид
	\begin{equation}
		\hat{H} = \frac{\textbf{p}^2}{2m} = -\frac{\hbar^2}{2m} \Delta
	\end{equation}
	в координатном представлении. Из линейности лапласиана следует, что решение стационарного уравнения можно искать в виде $\Psi(\textbf{r}) = \psi_x(x) \psi_y(y) \psi_z(z)$. Подставим такую волновую функцию:
	\begin{equation}
		\frac{d^2 \psi_x(x)}{d x^2}\psi_y(y)\psi_z(z) + \psi_x(x)\frac{d^2 \psi_y(y)}{d y^2}\psi_z(z) + \psi_x(x)\psi_y(y)\frac{d^2 \psi_z(z)}{d z^2} = -\frac{2mE}{\hbar^2} \psi_x(x)\psi_y(y)\psi_z(z) \label{freeParticle}
	\end{equation}
	Поделим равенство~\eqref{freeParticle} на волновую функцию:
	\begin{equation}
		\frac{d^2 \psi_x(x)}{d x^2} \frac{1}{\psi_x(x)} + \frac{d^2 \psi_y(y)}{d y^2} \frac{1}{\psi_y(y)} + \frac{d^2 \psi_z(z)}{d z^2} \frac{1}{\psi_z(z)} = -\frac{2mE}{\hbar^2}
	\end{equation}
	Получилась сумма функций, каждая из которой зависит от отдной из пространственных переменных, причем их сумма равна константе. Следовательно, каждая из этох функций постоянна. Из физических соображений очевидно, что кинетическая энергия частицы не может быть отрицательной. Обозначим $k_\alpha^2 =\frac{2m E_\alpha}{\hbar^2}, E = E_x + E_y + E_z$:
	\begin{equation}
		\psi_\alpha''(\alpha) + k^2_\alpha \psi_\alpha(\alpha) = 0
	\end{equation}
	Решением этого уравнения является
	\begin{equation}
		\psi_\alpha(\alpha) = C_1 e^{i k_\alpha \alpha} + C_2 e^{-i k_\alpha \alpha}
	\end{equation}
	Физический смысл каждой из двух функций в том, что первая описывает движение частицы в $+\infty$, а вторая - в противоположном направлении. Полная волновая функция $\Psi(\textbf{r})$ равна:
	\begin{equation}
		\Psi(\textbf{r}) = C_1 e^{i\textbf{kr}} + C_2 e^{-i\textbf{kr}}
	\end{equation}
	
	\item \textit{Как с помощью оператора эволюции записать решение уравнения Шредингера в произвольный момент времени, если задано начальное условие $\Psi(\textbf{r}, 0) = \psi_0(\textbf{r}) $?}
	
	От дифференциального вида уравнения Шредингера можно перейти к интегральному:
	\begin{equation}
		\Psi(\textbf{r}, t) = \psi_0(\textbf{r}) - \frac{i}{\hbar} \int\limits_0^t \hat{H} \Psi(\textbf{r}, \tau) d\tau
	\end{equation}
	Пусть гамильтониан системы не зависит от времени. Будем решать уравнение методом последовательных приближений. Пусть $\Psi^{(0)}(\textbf{r}, t) = \psi_0 (\textbf{r})$, тогда
	\begin{equation}
		\Psi^{(1)}(\textbf{r}, t) = \psi_0 (\textbf{r}) - \frac{it\hat{H} }{\hbar} \psi_0 (\textbf{r})
	\end{equation}
	Повторяя приближение бесконечное число раз, в пределе получим:
	\begin{equation}
		\Psi(\textbf{r}, t) = \sum\limits_{n=0}^\infty \frac{1}{n!} \Big(\frac{-it\hat{H} }{\hbar} \Big)^n \psi_0 (\textbf{r}) = e^{\frac{-it\hat{H} }{\hbar}} \psi_0 (\textbf{r}) = \hat{U}(t) \psi_0 (\textbf{r})
	\end{equation}
	где введен оператор эволюции $\hat{U}(t) = e^{\frac{-it\hat{H} }{\hbar}}$
	
	\item \textit{Какой вид имеет оператор эволюции консервативной системы?}
	\begin{equation}
		\hat{U}(t) = e^{\frac{-it\hat{H} }{\hbar}}
	\end{equation}
	
	\item \textit{
	Записать определение производной оператора по времени.
	}
	
	Производной физической величины можно сопоставить производную оператора, который задает эту величину. Найдем его явный вид:
	\begin{equation}
		\frac{df}{dt} = \frac{d}{dt} \braket{\psi(t) | \hat{f} | \psi(t)} = \frac{d}{dt} \braket{\psi_0 |\hat{U}^+ \hat{f} \hat{U} | \psi_0} = \bra{\psi_0} \frac{\partial \hat{U}^+}{\partial t} \hat{f} \hat{U} + \hat{U}^+\frac{\partial \hat{f}}{\partial t} \hat{U} + \hat{U}^+ \hat{f}\frac{\partial \hat{U}}{\partial t} \ket{\psi_0} \label{derivativeFirst}
	\end{equation}
	По определению оператора эволюции его производная равна
	\begin{equation}
		\frac{\partial U}{\partial t} = -\frac{i}{\hbar} \hat{H} \hat{U}
	\end{equation}
	Тогда равенство~\eqref{derivativeFirst} можно продолжить:
	\begin{equation}
		\frac{df}{dt} = \bra{\psi_0} \hat{U}^+ \Big( \frac{\partial \hat{f}}{\partial t} + \frac{i}{\hbar} \big( \hat{H} \hat{f} - \hat{f} \hat{H} \big) \Big)\hat{U}\ket{\psi_0} = \bra{\psi(t)}\frac{\partial \hat{f}}{\partial t} + \frac{i}{\hbar} \Big[ \hat{H}, \hat{f}\Big] \ket{\psi(t)}
	\end{equation}
	где квадратные скобки обзначают \textit{коммутатор} двух операторов. Оператор, который находится между векторами бра и кет, по определению называют \textit{производной оператора по времени}:
	\begin{equation}
	\frac{d\hat{f}}{dt} = \frac{\partial \hat{f}}{\partial t} + \frac{i}{\hbar} \Big[ \hat{H}, \hat{f}\Big] \label{derivativeLast}
	\end{equation}
	\item \textit{
	Как определяется коммутатор двух операторов?	
	}
	\begin{equation}
		\Big[ \hat{A}, \hat{B} \Big] = \hat{A} \hat{B} - \hat{B} \hat{A}
	\end{equation}
	\item \textit{
 Какому условию удовлетворяют операторы физических величин–интегралов движения в квантовой механике?	
	}
	
	Интеграл движения - величина, сохранающаяся с течением времени. Так как мы знаем оператор, который сопоставляется производной физической величины (формула~\eqref{derivativeLast}), то для интегралов движения этот оператор должен равняться нулю:
	\begin{equation}
		\frac{\partial \hat{f}}{\partial t} + \frac{i}{\hbar} \Big[ \hat{H}, \hat{f}\Big] = 0
	\end{equation}
	\item \textit{
	Какие физические величины могут быть включены в полный набор физических величин, определяющих состояние квантовой системы?	
	}
	
	Полный набор физических величин - это такой набор величин, что каждому уникальному набору значений этих величин соответствует вектор состояния из базиса пространства состояний. Другими словами, это тот набор величин, по которому можно построить базис в пространстве состояний рассматриваемой системы. Так как для решения уравнения Шредингера~\eqref{shrodingerEquation} состояние раскладывают по собственным векторам оператора Гамильтона, то обычно в полный набор включают энергию системы. Однако не всегда энергетический спектр задает базис, то есть одному значению энергии соответствует несколько базисных векторов. Однако для двух коммутирующих операторов можно выбрать общий базис собственных функций. Поэтому в полный набор физических величин можно включать величины, которые коммутируют с оператором Гамильтона.
	
	\item \textit{
		Как можно представить коммутатор $\Big[ \hat{A} \hat{B}, \hat{C}\Big]$?	
	}
	
	\begin{equation}
		\Big[ \hat{A} \hat{B}, \hat{C}\Big] = \hat{A}\hat{B}\hat{C} - \hat{C}\hat{A}\hat{B} \pm \hat{A} \hat{C} \hat{B} = \hat{A}(\hat{B}\hat{C} - \hat{C} \hat{B}) + (\hat{A}\hat{C} - \hat{C} \hat{A}) \hat{B} = \hat{A}\Big[ \hat{B}, \hat{C}\Big] + \Big[ \hat{A}, \hat{C}\Big] \hat{B} \label{abc}
	\end{equation}
	
	\item \textit{Определить полный набор физических величин свободной бесспиновой частицы.}

\end{enumerate}

\subsection*{Операторы и теория представлений}

\begin{enumerate}
	\item \textit{Чему равен (“табличный”) коммутатор $\big[\hat{x}_\alpha, \hat{p}_\beta \big]$?}
	
	Коммутаторы операторов физических величин постулируются - для двух физических случайных величин коммутатор пропорционален скобке Пуассона этих двух величин, в которых физические величины заменены на их операторы:
	\begin{equation}
		\Big\{ f, g \Big\} \rightarrow -\frac{i}{\hbar} \Big[\hat{f}, \hat{g}\Big]
	\end{equation}
	По определению, скобка Пуассона равна:
	\begin{equation}
		\Big\{ f, g \Big\} = \sum\limits_n \frac{\partial f}{\partial q_n} \frac{\partial g}{\partial p_n} - \frac{\partial f}{\partial p_n} \frac{\partial g}{\partial q_n}
	\end{equation}
	где $q_n, p_n$ - обобщенные координатв и импульс соответственно. Для самих координаты и импульса скобка Пуассона равна:
	\begin{equation}
		\Big\{ x_\alpha, p_\beta \Big\} = \delta_{\alpha \beta}
	\end{equation}
	и, следовательно коммутатор их операторов равен:
	\begin{equation}
		\big[\hat{x}_\alpha, \hat{p}_\beta \big] = i\hbar \delta_{\alpha\beta}
	\end{equation}
	
	\item \textit{Зная табличный коммутатор операторов координаты и импульса, определить оператор скорости нерелятивистской частицы.}
	
	Воспользуемся определением производной оператора по времени:
	\begin{equation}
		\hat{\textbf{v}} = \frac{d \textbf{r}}{dt} = \frac{\partial \textbf{r}}{\partial t} + \frac{i}{\hbar} \Big[ \hat{H}, \hat{\textbf{r}} \Big]
	\end{equation}
	Так как потенциальная энергия является функцией координаты, то её оператор коммутирует с оператором координаты. Посчитаем коммутатор оператора кинетической энергии с оператором координаты:
	\begin{equation}
		\Big[ \hat{T}, \hat{\textbf{r}} \Big] = \frac{1}{2m} \Big[ \hat{\textbf{p}}^2, \hat{\textbf{r}} \Big] \stackrel{~\eqref{abc}}{=} \frac{1}{2m} \Big(\hat{p}_\alpha \Big[\hat{p}_\alpha, \hat{\textbf{r}}\Big] + \Big[\hat{p}_\alpha, \hat{\textbf{r}}\Big]\hat{p}_\alpha \Big) = -\frac{i\hbar\hat{\textbf{p}}}{m}
	\end{equation}
	И тогда оператор скорости, как и в классическом случае, равен
	\begin{equation}
		\hat{\textbf{v}} = \frac{\hat{\textbf{p}}}{m}
	\end{equation}
	\item \textit{Как можно записать оператор, соответствующий физической величине \textbf{pr}?}
	
	Физическим величинам соответствуют эрмитовые операторы. Попробуем сопоставить величине $\varphi = p_\alpha x_\alpha$ оператор напрямую, и посчитаем эрмитово сопряженный оператор $\hat{\varphi}^+$:
	\begin{equation}
		\varphi^+ = (\hat{p}_\alpha \hat{x}_\alpha)^+ = \hat{x}_\alpha \hat{p}_\alpha = \hat{p}_\alpha \hat{x}_\alpha - \big[\hat{p}_\alpha, \hat{x}_\alpha \big] = \hat{p}_\alpha \hat{x}_\alpha + i\hbar \delta_{\alpha\alpha} = \hat{p}_\alpha \hat{x}_\alpha + 3i\hbar
	\end{equation}
	Как видно, напрямую сопоставленный оператор неэрмитов; однако, если взять оператор $\hat{\varphi} = \hat{p}_\alpha \hat{x}_\alpha + \frac{3}{2}i\hbar$, то такой оператор является эрмитовым. При переходе к классической механике ($\hbar \to 0$) введенный оператор переходит в величину $\varphi$, следовательно, введенный таким образом оператор соответствует заданной величине.
	
	Канонически верным оператором, который можно сопоставить величине $\textbf{pr}$, является следующий оператор:
	\begin{equation}
		\hat{\varphi} = \frac{\hat{p}_\alpha \hat{x}_\alpha + \hat{x}_\alpha \hat{p}_\alpha}{2}
	\end{equation}
	Этот оператор автоматически является эрмитовым и при предельном переходе к классической механике дает исходную величину $\textbf{pr}$.
	
	\item \textit{Что определяет выражение }$\braket{\textbf{r}|\Psi}$\textit{ = ?}
	
	Функция $\psi(\textbf{r}) = \braket{\textbf{r}|\Psi}$ (которую называют \textit{волновой функцией}) является проекцией состояния $\ket{\Psi}$ на базис собственных состояний оператора $\hat{\textbf{r}}$; другими словами, это состояние $\ket{\psi}$ в координатном представлении. Квадрат модуля этой функции есть плотность вероятности того, что частица находится в точке $\textbf{r}$.
	
	\item \textit{Для оператора координаты $\hat{\textbf{r} } \ket{\textbf{r}_0} = \ ?$}
	
	По определению, состояние $\ket{\textbf{r}_0}$ - состояние с определенной координатой, то есть это собственное состояние оператора $\hat{\textbf{r}}$. Следовательно,
	\begin{equation}
		\hat{\textbf{r}}\ket{\textbf{r}_0} = \textbf{r}_0 \ket{\textbf{r}_0}
	\end{equation}
	
	\item \textit{Для оператора импульса  $\hat{\textbf{p} } \ket{\textbf{p}_0} = \ ?$}
	
	Аналогично оператору координаты:
	\begin{equation}
		\hat{\textbf{p}}\ket{\textbf{p}_0} = \textbf{p}_0 \ket{\textbf{p}_0}
	\end{equation}
	
	\item \textit{Пусть совокупность векторов $\ket{n}$ -составляет базис. Чему равен оператор $\sum\limits_n \ket{n}\bra{n} = \ ?$}
	
	Так как вектора $\ket{n}$ состовляют базис, то любое состояние можно разложить по этому базису (смотри, например, формулу~\eqref{basisDecomposition}). Посмотрим, как этот оператор действует на произвольное состояние:
	\begin{equation}
		\sum\limits_n \big(\ket{n}\bra{n}\big) \ket{\psi} = \sum\limits_{n, n'} \big(\ket{n}\bra{n}\big) c_{n'}\ket{n'} = \sum\limits_{n, n'} c_{n'}\ket{n}\braket{n|n'} = \sum\limits_{n, n'} c_{n'}\ket{n}\delta_{nn'} = \sum\limits_{n,} c_{n}\ket{n} = \ket{\psi}
	\end{equation}
	то есть оператор не меняет произвольное состояние. Значит, этот оператор равен тождественному оператору $\hat{1}$.
	
	\item \textit{Для некоторого (не обязательно эрмитова) оператора $\hat{f}\ket{f_n} = f_n\ket{f_n}$, чему равно $\bra{f_n}\hat{f} = \ ?$}
	
	Пусть $\bra{f_n}\hat{f} = \bra{\psi}$, спроецируем $\bra{\psi}$ на базисные вектора $\bra{f_{n'}}$:
	\begin{equation}
		\braket{\psi|f_{n'}} = \braket{f_n|\hat{f}|f_{n'}} = f_{n'} \braket{f_n|f_{n'}} = f_{n'} \delta_{nn'}
	\end{equation}
	то есть проекции на вектора, отличные от $\bra{f_n}$, равны нулю. Значит, $\bra{\psi} = f_n\bra{f_n}$
	
	\item \textit{Пусть $\hat{f}^+=\hat{f}$, а $\hat{f}\ket{f_n} = f_n \ket{f_n}$, чему равен оператор $\sum\limits_n \ket{f_n}\bra{f_n} = \ ?$}
	
	У эрмитового оператора всегда существует базис из собственных векторов, следовательно, оператор равен тождетвенному (смотри вопрос 7).
	
	\item $\braket{\textbf{r}'|\hat{\textbf{r}}|\textbf{r}} = \ ?$
	\begin{equation}
		\braket{\textbf{r}'|\hat{\textbf{r}}|\textbf{r}} = \textbf{r} \braket{\textbf{r}'|\textbf{r}} =\textbf{r} \delta(\textbf{r}' - \textbf{r})
	\end{equation}
	где $\delta(\textbf{r})$ - дельта-функция Дирака.
	
	\item $\braket{\textbf{r}|\textbf{p}} = \ ?$
	
	Это значение есть ни что иное, как волновая функция состояния с определенным импульсом. Задача решается при помощи некоторого искусственого приема. Введем оператор
	\begin{equation}
		\hat{Q}_{\textbf{a}} = e^{-\frac{i}{\hbar}\textbf{a}\hat{\textbf{p}}}
	\end{equation}
	Этот оператор является функцией оператора координаты. Нам нужно вычислить его коммутатор с оператором координаты. Для этого вычислим коммутатор $\big[\hat{x}_\alpha,\hat{p}_\alpha^l \big]$ при помощи скобок Пуассона:
	\begin{equation}
		\Big\{x, p_x \Big\} = \frac{\partial x}{\partial x} \frac{\partial p_x^l}{\partial p_x} = \frac{\partial p_x^l}{\partial p_x}
	\end{equation}
	Следовательно, коммутатор $\big[\hat{x}_\alpha,\hat{p}_\alpha^l \big]$ равен:
	\begin{equation}
		\big[\hat{x}_\alpha,\hat{p}_\alpha^l \big] = i\hbar \frac{\partial \hat{\textbf{p}}^l}{\partial \hat{\textbf{p}}}
	\end{equation}
	И коммутатор $\hat{Q}_\textbf{a}$ с оператором координаты, по определению функции от оператора, равен
	\begin{equation}
		\Big[\hat{\textbf{r}},\hat{Q}_\textbf{a} \Big] = i\hbar \frac{\partial \hat{Q}_\textbf{a}}{\partial \hat{\textbf{p}}} = \textbf{a}\hat{Q}_\textbf{a}
	\end{equation}
	Теперь подействуем оператором $\hat{\textbf{r}}\hat{Q}_\textbf{a}$ на состояние с определенной координатой:
	\begin{equation}
		\hat{\textbf{r}}\hat{Q}_\textbf{a} \ket{\textbf{r}_0} = \Big(\hat{Q}_\textbf{a}\hat{\textbf{r}} + \hat{\textbf{a}}\hat{Q}_\textbf{a}\Big)\ket{\textbf{r}_0} = (\textbf{r}_0 + \textbf{a})\hat{Q}_\textbf{a} \ket{\textbf{r}_0}
	\end{equation}
	То есть вектор $\hat{Q}_\textbf{a} \ket{\textbf{r}_0}$ есть собственный вектор оператора координаты. Очевидно, что если взять в качестве вектора $\textbf{a}$ необходимую координату, и действовать на состояние с нулевой координатой, то мы получим состояние с необходимой координатой. Следовательно, оператор $\hat{Q}_\textbf{a}$ есть оператор трансляции. Аналогичным образом можно определить оператор трансляции $\hat{P}_\textbf{k}$ для импульсного представления. Тогда
	\begin{equation}
		\braket{\textbf{r}|\textbf{p}} = \braket{\textbf{r}|\hat{P}_\textbf{p}|0}_p = e^{-\frac{i}{\hbar}\textbf{pr}} \braket{\textbf{r}|0}_p = {e^{-\frac{i}{\hbar}\textbf{pr}}}_r \braket{0|\hat{Q}_\textbf{r}^+|0}_p = {e^{-\frac{i}{\hbar}\textbf{pr}}}_r \braket{0|0}_p \label{rp}
	\end{equation}
	Остается только определить константу $_r\braket{0|0}_p$. Для этого можно сделать следующее:
	\begin{equation}
		\braket{\textbf{p}|\textbf{p}'} = \delta(\textbf{p} - \textbf{p}') = \int d\textbf{r} \braket{\textbf{p}|\textbf{r}}\braket{\textbf{r}|\textbf{p}'}
	\end{equation}
	Подставляя в интеграл значение~\eqref{rp}, получаем:
	\begin{equation}
		|_r\braket{0|0}_p|^2 \int d\textbf{r} e^{\frac{i}{h}(\textbf{p}' - \textbf{p})\textbf{r}} = \delta(\textbf{p} - \textbf{p}')
	\end{equation}
	Интеграл в равенстве пропорционален дельта-функции, значит, искомый коэффициент равен:
	\begin{equation}
		|_r\braket{0|0}_p|^2 = (2\pi\hbar)^{-3}
	\end{equation}
	Тогда искомое выражение равно:
	\begin{equation}
		\braket{\textbf{r}|\textbf{p}} = \frac{1}{(2\pi\hbar)^{\frac{3}{2}}} e^{\frac{i}{\hbar}\textbf{pr}}
	\end{equation}
	
	\item $\braket{\textbf{p}|\textbf{r}} = \ ?$
		
	По определению,
	\begin{equation}
		\braket{\textbf{p}|\textbf{r}} = (\braket{\textbf{r}|\textbf{p}})^* = \frac{1}{(2\pi\hbar)^{\frac{3}{2}}} e^{-\frac{i}{\hbar}\textbf{pr}}
	\end{equation}
	
	\item \textit{Записать волновую функцию свободной нерелятивистской частицы.}
	
	Смотри вопрос №3 раздела "Уравнение Шредингера"
	\item $\braket{\textbf{r}'|\textbf{r}} = \ ?$
	
	Векторы $\ket{\textbf{r}}$ - базисные вектора, нормированные на дельта-функцию:
	\begin{equation}
		\braket{\textbf{r}'|\textbf{r}} = \delta(\textbf{r}' - \textbf{r})
	\end{equation}
	
	\item $\braket{\textbf{p}'|\textbf{p}} = \ ?$
	
	Аналогично состояниям с определенной координатой:
	\begin{equation}
		\braket{\textbf{p}'|\textbf{p}} = \delta(\textbf{p}' - \textbf{p})	
	\end{equation}
	
	\item \textit{Записать уравнение Шредингера для частицы с массой $m$ в координатном представлении.}
	
	Для того, чтобы перейти в координатное представление, нам нужно выяснить, как действуют операторы координаты и импульса на волновую функцию. Рассмотрим действие оператора координаты: пусть $\ket{\varphi} = \hat{\textbf{r}}\psi$, тогда
	\begin{equation}
		\braket{\textbf{r}|\varphi} = \braket{\textbf{r}|\hat{\textbf{r}}|\psi} = \braket{\textbf{r}|\hat{\textbf{r}}\hat{1}|\psi} = \int d\textbf{r}' \braket{\textbf{r}|\hat{\textbf{r}}|\textbf{r}'} \braket{\textbf{r}'|\psi} = \int d\textbf{r}' \textbf{r}' \delta(\textbf{r} - \textbf{r}') \psi(\textbf{r}') = \textbf{r}\psi(\textbf{r})
	\end{equation}
	Таким образом, действие оператора координаты на волновую функцию аналогично действию оператора координаты на состояние:
	\begin{equation}
		\hat{\textbf{r}}\psi(\textbf{r}) = \textbf{r} \psi(\textbf{r})
	\end{equation}
	Теперь выясним действие оператора импульса на волновую функцию. Пусть $\ket{\phi} = \hat{\textbf{p}}\ket{\psi}$. Тогда
	\begin{equation}
		\braket{\textbf{r}|\phi} = \braket{\textbf{r}|\hat{\textbf{p}}\hat{1}|\psi} = \int d\textbf{r}' \braket{\textbf{r}|\hat{\textbf{p}}|\textbf{r}'} \psi(\textbf{r}') \label{imulseFirst}
	\end{equation}
	Найдем матричный элемент оператора $\hat{\textbf{p}}$ в координатном предствалении:
	\begin{equation}
	\begin{split}
		\braket{\textbf{r}|\hat{\textbf{p}}|\textbf{r}'} = \braket{\textbf{r}|\hat{1}\hat{\textbf{p}}\hat{1}|\textbf{r}'} = \int d\textbf{p}d\textbf{p}'\braket{\textbf{r}|\textbf{p}} \braket{\textbf{p}|\hat{\textbf{p}}|\textbf{p}'} \braket{\textbf{p}'|\textbf{r}'} = \int d\textbf{p} \textbf{p} \braket{\textbf{r}|\textbf{p}} \braket{\textbf{p}|\textbf{r}'} =\\= \frac{1}{(2\pi\hbar)^3} \int d\textbf{p} \textbf{p} e^{\frac{i}{\hbar}\textbf{p}(\textbf{r} - \textbf{r}')} = i\hbar \frac{\partial}{\partial \textbf{r}'}\Big(\frac{1}{(2\pi\hbar)^3}\int d\textbf{p} e^{\frac{i}{\hbar}\textbf{p}(\textbf{r} - \textbf{r}')}  \Big) = i\hbar \frac{\partial}{\partial \textbf{r}'} \delta(\textbf{r} - \textbf{r}')
		\end{split}
	\end{equation}
	Подставим получившееся выражение в~\eqref{imulseFirst}:
	\begin{equation}
		\braket{\textbf{r}|\phi} = i\hbar \int d\textbf{r}' \Big(\frac{\partial}{\partial \textbf{r}'} \delta(\textbf{r} - \textbf{r}')\Big) \psi(\textbf{r}') = - i\hbar \int d\textbf{r}' \delta(\textbf{r} - \textbf{r}') \frac{\partial \psi}{\partial \textbf{r}'} = - i\hbar \frac{\partial \psi}{\partial \textbf{r}}
	\end{equation}
	Следовательно, действие оператора импульса на волновую функцию сводится к дифференцированию:
	\begin{equation}
		\hat{\textbf{p}}\psi(\textbf{r}) = -i\hbar \frac{\partial \psi}{\partial \textbf{r}}
	\end{equation}
	Теперь можно легко записать уравнение Шредингера в координатном представлении:
	\begin{equation}
		i\hbar \frac{\partial \psi}{\partial t} = -\frac{\hbar^2}{2m} \Delta \psi + U(\textbf{r})\psi
	\end{equation}
	
	\item \textit{Записать стационарное уравнение Шредингера для частицы с массой m в импульсном представлении.}
	
	Аналогично предыдущему пункту легко найти, что оператор координаты в импульсном представлении выглядит следующим образом:
	\begin{equation}
		\hat{\textbf{r}} = i\hbar \nabla
	\end{equation}
	Тогда стационарное уравнение Шредингера в импульсном представлении запишется следующим образом:
	\begin{equation}
		\frac{\textbf{p}^2}{2m} \psi_\textbf{p} + U(i\hbar\nabla)\psi_\textbf{p} = E \psi_\textbf{p}
	\end{equation}
\end{enumerate}

\subsection*{Основные коммутационные соотношения}
\begin{enumerate}
	\item \textit{Чему равен коммутатор $[x,\hat{p}_{x}]$?}
	\begin{equation}
		[x,\hat{p}_{x}]=i\hbar
	\end{equation}
	\item \textit{Чему равен коммутатор $[\hat{x}_{\alpha},\hat{p}_{\beta}]$?}
	\begin{equation}
		[\hat{x}_{\alpha},\hat{p}_{\beta}]=i\hbar\delta_{\alpha\beta}
	\end{equation}
	\item \textit{Чему равен коммутатор $[\hat{\boldsymbol{p}},U(\boldsymbol{r})]$?}
	\begin{equation}
		[\hat{\boldsymbol{p}},U(\boldsymbol{r})]=-i\hbar\nabla U(\boldsymbol{r})
	\end{equation}
	\item \textit{Чему равен коммутатор $[\hat{l}_{x},\hat{l}_{y}]$?}
	\begin{equation}
		[\hat{l}_{x},\hat{l}_{y}]=i\hat{l}_{z}
	\end{equation}
	\item \textit{Чему равен коммутатор $[\hat{l}_{\alpha},\hat{l}_{\beta}]$?}
	\begin{equation}
		[\hat{l}_{\alpha},\hat{l}_{\beta}]=ie_{\alpha\beta\gamma}\hat{l}_{\gamma}
	\end{equation}
	\item \textit{Чему равен коммутатор $[\hat{l}_{\alpha},\hat{\boldsymbol{l}}^{2}]$?}
	\begin{equation}
		[\hat{l}_{\alpha},\hat{\boldsymbol{l}}^{2}]=0
	\end{equation}
	\item \textit{Чему равен коммутатор $[\hat{l}_{\text{z}},\hat{l}_{+}]$?}
	\begin{equation}
		[\hat{l}_{\text{z}},\hat{l}_{+}]=\hat{l}_{+}
	\end{equation}
	\item \textit{Чему равен коммутатор $[\hat{l}_{\text{z}},\hat{l}_{-}]$?}
	\begin{equation}
		[\hat{l}_{\text{z}},\hat{l}_{-}]=-\hat{l}_{-}
	\end{equation}
\end{enumerate}

\subsection*{Граничные условия}

\begin{enumerate}
	\item \textit{Как формулируются граничные условия для нахождения связанных состояний?}
	
	В связном состоянии вероятность найти частицу на бесконечности равна нулю. Так как плотность вероятности определяется через модуль квадрата волновой функции, то сама волновая функция на бесконечности должна равняться нулю:
	\begin{equation}
		\left.\psi(\textbf{r})\right|_{|\textbf{r}| \to \infty} = 0
	\end{equation}
	
	\item \textit{Как формулируются граничные условия для задач непрерывного спектра?}
	
	Первое условие всегда состоит в том, что волновая функция должна быть непрерывна. Второе условие определяется видом потенциала и находится напрямую из уравнения Шредингера. Проинтегрируем стационарное уравнение вокруг особенности потенциала (для простоты рассмотрим одномерную задачу с особенностью в нуле):
	\begin{equation}
		-\frac{\hbar^2}{2m}\int\limits_{-\varepsilon}^{\varepsilon}\psi''(x)dx + \int\limits_{-\varepsilon}^{\varepsilon} U(x)\psi(x)dx = E\int\limits_{-\varepsilon}^{\varepsilon}\psi(x)
	\end{equation}
	Устремляя $\varepsilon$ к нулю, получим следующее условие:
	\begin{equation}
		\psi'(+0) - \psi'(-0) = -\frac{2m}{\hbar^2} \lim_{\varepsilon \to +0}\int\limits_{-\varepsilon}^{\varepsilon} U(x)\psi(x)dx
	\end{equation}
	Как видно из получившейся формулы, в случае достаточно хорошей функции потенциала получаем требование непрерывности производной волновой функции.

\end{enumerate}

\subsection*{Осциллятор}
\begin{enumerate}
	\item \textit{Записать гамильтониан линейного гармонического осциллятора.}
	
	Как известно еще из курса теоретической механики, функция Гамильтона одномерного осциллятора выглядит следующим образом:
	\begin{equation}
		H = \frac{p^2}{2m} + \frac{m\omega^2 x^2}{2}
	\end{equation}
	
	\item \textit{Как определяются осцилляторные единицы энергии, длины, импульса?}
	
	Осцилляторной единицей энергии берется величина $E_0 = \hbar\omega$ - данная величина имеет размерность энергии. Пусть $E = \varepsilon E_0$. Разделим стационарное уравнение Шредингера на $E_0$:
	\begin{equation}
		\frac{1}{2}\big( \frac{\hat{p}^2}{m\hbar\omega} + \frac{m\omega}{\hbar}\hat{x}^2 \big)\ket{\psi_\varepsilon} = \varepsilon \ket{\psi_\varepsilon}
	\end{equation}
	Приняв за единицы координаты и импульса $x_0 = \sqrt{\frac{\hbar}{m\omega}}$ и $p_0 = \sqrt{\hbar m \omega}$ соответственно, получим канонический вид уравнения Шредингра для осциллятора:
	\begin{equation}
		\frac{1}{2}\big( \hat{P}^2 + \hat{Q}^2\big) \ket{\psi_\varepsilon} = \varepsilon \ket{\psi_\varepsilon}
	\end{equation}
	где $\hat{p} = p_0 \hat{P}, \ \hat{x} = x_0 \hat{Q}$. Размерность введенных единиц соответствует сопоставленным величинам.
	
	\item \textit{Записать определение операторов $a$ и $a^+$ через операторы координаты и импульса.}
	
	Операторы $a$ и $a^+$ вводятся для упрощения вида гамильтониана. Определяются они следующим образом:
	\begin{equation}
		a = \frac{1}{\sqrt{2}}\big( \hat{Q} + i \hat{P} \big)
	\end{equation}
	где операторы $\hat{Q}, \ \hat{P}$ были введены в предыдущем вопросе. Сопряженный оператор задается прямым сопряжением:
	\begin{equation}
		a^+ = \frac{1}{\sqrt{2}}\big( \hat{Q} - i \hat{P} \big)
	\end{equation}
	Тогда оператор Гамильтона перепишется в следующем виде:
	\begin{equation}
		\hat{H} = \frac{1}{2}(aa^+ + a^+ a) \label{oscillatorFirst}
	\end{equation}
	
	\item \textit{Выразить операторы координаты и импульса через операторы $a$ и $a^+$.}
	
	Решая систему линейных уравнений, находим:
	\begin{equation}
		\hat{Q} = \frac{1}{\sqrt{2}}(a + a^+), \ \hat{P} = \frac{1}{i\sqrt{2}}(a - a^+)
	\end{equation}
	
	\item \textit{Чему равен коммутатор $[a,a^+] = \ ?$}
	\begin{equation}
		[a,a^+] = \frac{1}{2}[\hat{Q} + i\hat{P}, \hat{Q} - i\hat{P}] = \frac{1}{2}\Big(-i[\hat{Q},\hat{P}] + i[\hat{P},\hat{Q}] \Big) = \frac{i}{p_0 x_0}[\hat{p}, \hat{x}] = 1
	\end{equation}
	
	\item \textit{Записать выражение гамильтониана осциллятора через $a$ и $a^+$.}
	
	Зная коммутатор операторов $a, a^+$, можно переписать выражение~\eqref{oscillatorFirst}:
	\begin{equation}
		\hat{H} = a^+a + \frac{1}{2}
	\end{equation}
	
	\item \textit{Как определяется спектр осциллятора?}
	
	Рассмотрим спектр оператора $\hat{\nu} = a^+a$. Вычислим коммутатор $[a,a^+a]$:
	\begin{equation}
		[a,a^+a] = a^+[a,a] + [a,a^+]a = a
	\end{equation}
	Аналогичным образом $[a^+,a^+a] = a^+$. Пусть у нас есть собственное состояние $\ket{\nu}$ оператора $\hat{\nu}$, и $\ket{\psi} = a\ket{\nu}$. Рассмотрим действие оператора $\hat{\nu}$ на $\ket{\psi}$:
	\begin{equation}
		\hat{\nu}\ket{\psi} = \hat{\nu} a\ket{\nu} = (a\hat{\nu} - a)\ket{\nu} = (\nu - 1)a\ket{\nu} = (\nu - 1)\ket{\psi}
	\end{equation}
	Таким образом, вектор $a\ket{\nu}$ является собственным вектором оператора $\hat{\nu}$ со значением $\nu - 1$. Получается, что оператор $a$ понижает собственное число на единицу. Проделывая аналогичные выкладки, можно показать, что оператор $a^+$ повышает собственное значение на единицу. Заметим, что собственные значения оператора $\hat{\nu}$ неотрицательны:
	\begin{equation}
		\braket{\nu|a^+a|\nu} = \nu\braket{\nu|\nu} = \nu = \| a\ket{\nu}\|^2 \geq 0
	\end{equation}
	Следовательно, существует минимальное значение $\nu_0$ такое, что $a\ket{\nu_0} = 0$. Подействовав оператором $\hat{\nu}$ на это состояние выясняем, что минимальное собственное значени равно нулю:
	\begin{equation}
		a^+a\ket{\nu_0} = \nu_0 = a^+ (a\ket{\nu_0}) = 0
	\end{equation}
	Таким образом, спектр осциллятора дискретен, расстояние между значениями равно $\hbar\omega$ и минимальной энергией $\frac{\hbar\omega}{2}$.
	
	\item \textit{Записать полный набор квантовых чисел, определяющих состояния одномерного гармонического осциллятора. Какие значения могут принимать квантовые числа?}
	
	Спектр оператора Гамильтона не является вырожденным, следовательно полный набор квантовых чисел для осциллятора будет состоять только из номера уровня. В предыдущем вопросе было показано, что спектр гамильтониана дискретен.
	
	\item \textit{Чему равен результат действия оператора $a\ket{n} = \ ?$}
	
	Ранее было показано, что оператор $a$ понижает собственное число на единицу. Остается только выяснить коэффициент, на который домножается состояние. Пусть $a\ket{n} = \alpha_{n-1}\ket{n-1}$, тогда $\bra{n}a^+ = \alpha^*_{n-1}\bra{n-1}$, и
	\begin{equation}
		|\alpha_{n-1}|^2 = \braket{n|a^+a|n} = n
	\end{equation}
	Следовательно, $\alpha_{n-1} = \sqrt{n-1}e^{i\varphi}$. Опуская несущественный фазовый множитель, получаем искомый результат:
	\begin{gather}
		a\ket{n} = \sqrt{n}\ket{n-1} \\
		a^+\ket{n} = \sqrt{n+1} \ket{n+1}
	\end{gather}
	Действие оператора $a^+$ вычисляется аналогично.
	\item \textit{ Чему равно $a^+\ket{n} = \ ?$}
	
	Смотри предыдущий вопрос.
	\item \textit{Чему равно $a\ket{0} = \ ?$}
	
	При поиске спектра было показано, что результат равен нулю. Аналогично ответ можно найти, воспользовавшись явным действием оператора $a$.
	\item \textit{Выразить произвольное состояние осциллятора $\ket{n}$ через основное состояние $\ket{0}$.}
	
	Так как оператор $a^+$ повышает собственное число на единицу, то состояние $\ket{n}$ можно выразить через состояние $\ket{0}$ следующим образом:
	\begin{equation}
		\ket{n} = C_n (a^+)^n \ket{0}
	\end{equation}
	Остается только найти константу $C_n$. Зная явный вид действия оператора $a^+$, получаем:
	\begin{equation}
		\ket{n} = C_n \sqrt{n!}\ket{n}
	\end{equation}
	то есть константа $C_n = \frac{1}{\sqrt{n!}}$. Таким образом, получаем ответ:
	\begin{equation}
		\ket{n} = \frac{(a^+)^n}{\sqrt{n!}}\ket{0}
	\end{equation}
	
	\item \textit{Какой вид имеет волновая функция основного состояния осциллятора $\psi(Q)$= (в безразмерных единицах)?}
	
	Волновая функция $\psi(Q) = \braket{Q|0}$. Для нахождения явного вида воспользуемся искусственным приемом - найдем явный вид $\braket{Q|a|0}$ (это значение равно нулю):
	\begin{equation}
	\begin{split}
		\braket{Q|a|0} = \frac{1}{\sqrt{2}}\braket{Q|\hat{Q} + i \hat{P}|0} = \frac{1}{\sqrt{2}}\int dQ' \Big( \braket{Q|\hat{Q}|Q'} + i\braket{Q|\hat{P}|Q'} \Big)\braket{Q'|0} = \\ =\frac{1}{\sqrt{2}}\int dQ'\Big(Q'\delta(Q-Q') - \frac{\partial}{\partial Q'}\delta(Q-Q') \Big) \psi(Q) = \frac{1}{\sqrt{2}} \Big(Q\psi(Q) + \psi'(Q)\Big) = 0
	\end{split}
	\end{equation}	
	Решая получившееся дифференциальное уравнение, получаем:
	\begin{equation}
		\psi(Q) = A e^{-\frac{1}{2}Q^2}
	\end{equation}
	Нормируя квадрат полученной функции, находим $A = \pi^{-\frac{1}{4}}$.
\end{enumerate}


\subsection*{Момент}
	\begin{enumerate}
		\item \textit{Выразить $\hat{\textbf{l}}^2$ через $\hat{l}_z$ и $\hat{l}_\pm$.}	
		\begin{equation}
			\hat{\boldsymbol{l^{2}}}=\hat{l_{z}^{2}}+\frac{1}{2}(\hat{l_{+}}\hat{l_{-}}+\hat{l_{-}}\hat{l_{+}})=\hat{l_{z}}^{2}+\hat{l_{z}}+\hat{l_{-}}\hat{l_{+}}=\hat{l_{z}}^{2}-\hat{l_{z}}+\hat{l_{+}}\hat{l_{-}}
		\end{equation}
		\item \textit{$\hat{\textbf{l}}^2 \ket{l, m} = \ ?$}
		\begin{equation}
			\hat{\boldsymbol{l}}^2\ket{l,m}=l(l+1)\ket{l,m}
		\end{equation}			
		\item \textit{$\hat{l}_z \ket{l, m} = \ ?$}
		\begin{equation}
			\hat{l}_z\ket{l,m}=m\ket{l,m}
		\end{equation}	
		\item \textit{$\hat{l}_+ \ket{l, l} = \ ?$}
		\begin{equation}
			\hat{l}_+\ket{l,l}=0
		\end{equation}	
		\item \textit{$\hat{l}_- \ket{l, -l} = \ ?$}
		\begin{equation}
			\hat{l}_-\ket{l,-l}=0
		\end{equation}		
		\item \textit{$\hat{l}_\pm \ket{l, m} = \ ?$}
		\begin{equation}
			\hat{l}_\pm\ket{l,m}=\sqrt{(l\mp m)(l\pm m-1)}\ket{l,m\pm1}
		\end{equation}	
	\end{enumerate}
	
\subsection*{Центральное поле}
	\begin{enumerate}
		\item \textit{Как определяется полный набор физических величин бесспиновой частицы в центральном поле?}
		
		По теореме Кёнига энергию системы можно переписать в следующем виде:
		\begin{equation}
			\hat{H} = \frac{\hat{p}_r^2}{2m} + \frac{\hat{\textbf{M}}^2}{2I} + U(\hat{r})
		\end{equation}
		где $\textbf{M}$ - момент импульса, $I$ момент инерции. Переходя в сферические координаты, можно доказать, что оператор момента импульса действует только на угловые координаты. Из описанных выше коммутационных соотношений для оператора момента импульса следует, что операторы $\hat{H}, \hat{M}_z$ и $\hat{\textbf{M}}^2$ коммутируют между собой, и их можно включить в полный набор физических величин. Явные вычисления показывают, что этот набор набор оказывается полным.
		 
		 \item \textit{Как разделяются переменные в волновой функции, описывающей состояние частицы в центральном поле $\psi(\textbf{r}) = \ ?$}
		 
		 Из-за того, что лапласиан в сферических координатах разбиватся на две части - радиальную и угловую, можно предположить, что возможно следующее разделение переменных:
		 \begin{equation}
		 	\psi(\textbf{r}) = R(r)Y(\theta,\varphi)
		 \end{equation}
		 Подставим эту волновую функцию в стационарное уравнение Шредингера и поделив на неё, получим:
		 \begin{equation}
		 	-\frac{\hbar^2}{2m} \frac{\Delta_r R(r)}{R(r)} - \frac{\hbar^2}{2mr^2}\frac{\Delta_{\theta,\varphi} Y(\theta, \varphi)}{Y(\theta, \varphi)} + U(r) = E
		 \end{equation}
		 Данное уравнение можно привести к виду, в котором будет сумма функции, зависящей только от $r$, и функции, зависящей только от $\theta,\varphi$, причем их сумма будет равна константе. Следовательно, данное разделение переменных имеет место.
		 
		 \item \textit{Асимптотическое поведение радиальной функции связанного состояния $\left.R_{nl}(r)\right|_{r \to 0} \sim \ ?$}
		 
		 Угловая функция $Y(\theta, \varphi)$ выбирается как собственная для оператора квадрата момента импульса. Тогда уравнение на радиальную часть $R_{nl}$ получится следующим:
		 \begin{equation}
		 	R''_{nl} + \frac{2}{r}R_{nl}' - \frac{l(l+1)}{r^2}R_{nl} - \frac{2m}{\hbar^2}(E_n-U(r))R_{nl} = 0
		 \end{equation}
		 При устремлении $r$ к нулю преобладающими оказываются следующие слагаемые\footnote{При достаточно адекватном потенциале}:
		 \begin{equation}
		 	R_{nl}'' +\frac{2}{r} R_{nl}' - \frac{l(l+1)}{r^2}R_{nl} = 0
		 \end{equation}
		 Решениями этого уравнения являются функции вида $Cr^s$. Найдем $s$:
		 \begin{equation}
		 	s(s+1) - l(l+1) = 0 \Rightarrow s = l, -(l+1)
		 \end{equation}
		 Второе значение дает неограниченную волновую функцию, что противоречит условию нормировки состояния (интеграл будет расходиться). Следовательно, асимптотика $R_{nl}$ в нуле следующая:
		 \begin{equation}
		 	\left.R_{nl}(r)\right|_{r \to 0} \sim r^l
		 \end{equation}
		 
		 \item \textit{Асимптотическое поведение радиальной функции связанного состояния $\left.R_{nl}(r)\right|_{r \to \infty} \sim \ ?$}
		 
		 Аналогично предыдущему вопросу, при стремлении $r$ к бесконечности преобладают следующие слагаемые:
		 \begin{equation}
		 	R_{nl}'' - \frac{2mE}{\hbar^2}R_{nl} = 0
		 \end{equation}
		 Решением этого уравнения является экспоненты вида $c e^{\pm \varkappa r}$, где $\varkappa = \sqrt{\frac{2mE}{\hbar^2}}$. Положительная степень экпоненты противоречит условию связности состояния, следовательно, асимптотика следующая:
		 \begin{equation}
		 	\left.R_{nl}(r)\right|_{r \to 0} \sim e^{- \varkappa r}
		 \end{equation}
		 
		 \item \textit{Записать гамильтониан атома водорода.}
		 
		 В атоме водорода потенциал кулоновский ($U(r) = -\frac{e^2}{r}$), и гамильтониан выглядит следующим образом (в координатном представлении):
		 \begin{equation}
		 	\hat{H} = -\frac{\hbar^2}{2m}\Delta -\frac{e^2}{r}
		 \end{equation}
		 
		 \item \textit{Как определяется атомная система единиц?}
		 
		 Атомная система единиц за основу берет единицу атомной скорости $v_0 = \frac{e^2}{\hbar}$, из которой получается постоянная тонкой структуры $\frac{v_0}{c} = \frac{1}{137}$. Единица энергии задается следующим образом:
		 \begin{equation}
		 	E_0 = mv_0^2 = \frac{me^4}{\hbar} = \frac{\hbar^2}{ma_0^2}
		 \end{equation}
		 где $a_0 = \frac{\hbar}{mv_0} = \frac{\hbar^2}{me^2}$ - атомная единица расстояния.
		 
		 \item \textit{Записать полный набор и значения, которые могут принимать квантовые числа, определяющие состояние атома водорода.}
		 
		 В первом вопросе было показано, что полным набором являются энергия, квадрат момента импульса и его проекция на выбранную ось. Для удобства в качестве квантовых чисел берут номер уровня энергии $n$, максимальную проекцию момента импульса $l$ в единицах $\hbar$ и проекцию на выбранную ось $m$ также в единицах $\hbar$. Число $n$ задается равенством
		 \begin{equation}
		 	n = \sqrt{\frac{me^4}{2|E_n|\hbar^2}}
		 \end{equation}
		 откуда легко находится спектр атома водорода:
		 \begin{equation}
		 	E_n = -\frac{me^4}{2\hbar n^2}
		 \end{equation}
		 Ограничения на число $l$ задаются неравенством $n - l - 1 \geq 0$. Число $m$ как проекция спина на выбранную ось по модулю не может превосходить максимальной.
		 \item \textit{Записать спектр атома водорода и определить кратность вырождения уровней энергии.}
		 
		 В предыдущем вопросе спектр атома водорода был получен, подсчитаем кратность вырождения:
		 \begin{equation}
		 	N = \sum\limits_{l=0}^{n-1}\sum\limits_{m=-l}^{l}1 = \sum\limits_{l=0}^{n-1} 2l+1 = n^2
		 \end{equation}
		 
		 \item \textit{Какой вид имеет волновая функция основного состояния атома водорода $\psi_0(\textbf{r}) = \ ?$}
		 
		 Радиальная часть волновой функции основного состояния выглядит следующим образом:
		 \begin{equation}
		 	R_{10}(r) = \frac{2}{\sqrt{a_0}}e^{-\frac{r}{a_0}}
		 \end{equation}
		 Угловая часть основного состояния:
		 \begin{equation}
		 	Y_{00}(\theta, \varphi) = \frac{1}{\sqrt{4\pi}}
		 \end{equation}
	\end{enumerate}

{\center{\section*{Квантовая механика, часть 2}}}

\subsection*{Квазиклассика}
	\begin{enumerate}
		\item \textit{Представление волновой функции частицы через квантовое действие.}

\begin{equation}
\psi(\boldsymbol{r},t)=Ae^{\frac{i}{\hbar}\hat{S_{q}}}
\end{equation}


		\item \textit{Уравнение для квантового действия (консервативной системы).}
\begin{equation}
-\frac{i\hbar}{2m}\triangle s+\frac{1}{2m}(\nabla s)^{2}+U(q)-E=0
\end{equation}


\begin{equation}
s=S_{q}+Et
\end{equation}


		\item \textit{Разложение квантового действия по степеням $\hbar$.}
\begin{equation}
S_{q}=\sum_{k=0}^{+\infty}(\frac{\hbar}{i})^{k}s_{k}
\end{equation}


		\item \textit{Уравнение для квантового действия в одномерном случае.} 
\begin{equation}
-\frac{i\hbar}{2m}s^{\prime\prime}+\frac{1}{2m}(s^{\prime})^{2}+U(q)-E=0
\end{equation}


		\item \textit{Уравнение для определения нулевого приближения квантового действия. Выражение для квазиклассического импульса.}  
\begin{equation}
\frac{1}{2m}(s_{0}^{\prime})^{2}=E\text{-}U(q)
\end{equation}


\begin{equation}
s_{0}=\pm\int\sqrt{2m(E-U(q))}dq=\pm\int p(q)dq
\end{equation}


\begin{equation}
2s{}_{0}^{\prime}s{}_{1}^{\prime}+s{}_{0}^{\prime\prime}=0
\end{equation}


\begin{equation}
s_{1}=-\frac{1}{2}\ln p(q)+C
\end{equation}


		\item \textit{Критерий применимости квазиклассического приближения.}  
\begin{equation}
|\frac{\hbar s^{\prime\prime}}{(s^{\prime})^{2}}|=|\frac{d}{dq}\frac{\hbar}{s^{\prime}}|=|\frac{d}{dq}\frac{\hbar}{p}|=|\frac{d}{dq}\lambda|\ll1
\end{equation}


\begin{equation}
=|\frac{\hbar}{p^{2}}\frac{dp}{dq}|=|\frac{\hbar m}{p^{3}}(-\frac{dU}{dq})|\ll1
\end{equation}


		\item \textit{Общий вид волновой функции квазиклассического приближения в классически разрешенной области.} 
\begin{equation}
\psi(q)=\frac{C_{1}}{\sqrt{p(q)}}\exp\{\frac{i}{\hbar}\int p(q)dq\}(1+O(\frac{d\lambda}{dq}))+\frac{C_{2}}{\sqrt{p(q)}}\exp\{-\frac{i}{\hbar}\int p(q)dq\}(1+O(\frac{d\lambda}{dq}))
\end{equation}


		\item \textit{Возможные постановки задачи в квазиклассическом приближении.} 
		
		Поскольку сшивка волновых функций в классически разрешнной и в
классически запрещнной зонах существенным образом зависит от граничных
условий то рассматриваются разные задачи а именно:

1. нахождение уровней энергии в потенциальной яме

2. нахождение коэффициента проникновения частицы через потенциаль
ный барьер.

		\item \textit{Точки поворота (определение).}  
\begin{equation}
p(q_{0})=0
\end{equation}
или что то же самое:
\begin{equation}
E=U(q_{0})
\end{equation}


		\item \textit{Правило квантования Бора-Зоммерфельда для определения энергетического спектра связанных состояний.}  
\begin{equation}
\intop_{a}^{b}p(q)dq=\pi\hbar(n+\frac{1}{2})
\end{equation}


или: 
\begin{equation}
\oint p(q)dq=2\pi\hbar(n+\frac{1}{2})
\end{equation}


		\item \textit{Определение квазиклассического периода.} 

\begin{equation}
T=\oint\frac{dq}{v}
\end{equation}


		\item \textit{Нормировка волновой функции связанного состояния в квазиклассическом приближении.} 
\begin{equation}
\begin{split}
1=\intop_{-\infty}^{+\infty}|\psi(q)|^{2}dq\approx\intop_{a}^{b}|\psi(q)|^{2}dq=4|C|^{2}\intop_{a}^{b}\frac{\cos^{2}\{\frac{\intop_{q}^{b}p(q)dq}{\hbar}-\frac{\pi}{4}\}}{p(q)}dq \\
\approx2|C|^{2}\intop_{a}^{b}\frac{dq}{p(q)}=\frac{|C|^{2}T}{m}=\frac{2\pi|C|^{2}}{m\omega}
\end{split}
\end{equation}


\begin{equation}
C=\sqrt{\frac{m\omega}{2\pi}}
\end{equation}


\begin{equation}
\psi(q)=\sqrt{\frac{2m\omega}{\pi}}\cos^{2}\{\frac{\intop_{q}^{b}p(q)dq}{\hbar}-\frac{\pi}{4}\}
\end{equation}


		\item \textit{Плотность энергетического спектра связанных состояний в квазиклассическом приближении.} 

\begin{equation}
\begin{split}
\frac{\triangle n}{\triangle E}\approx\frac{dn}{dE}=\frac{d}{dE}\frac{1}{2\pi\hbar}\oint p(q)dq=\frac{d}{dE}\frac{1}{2\pi\hbar}\oint\sqrt{2m(E-U(q))}dq \\
=\frac{1}{2\pi\hbar}\oint\frac{m}{p(q)}dq=\frac{1}{2\pi\hbar}\oint\frac{dq}{v}=\frac{T}{2\pi\hbar}=\frac{1}{\hbar\omega}
\end{split}
\end{equation}


		\item \textit{Постановка задачи о проникновении частицы через потенциальный барьер.} 

\begin{equation}
\text{ясно-понятно}
\end{equation}


		\item \textit{Вероятность проникновения через потенциальный барьер.}
\begin{equation}
D\text{\ensuremath{\approx}}\exp\{-\frac{2}{\hbar}\int_{a}^{b}|p(q)|dq\}
\end{equation}
	\end{enumerate}
	
\subsection*{Стационарная теория возмущений}
	\begin{enumerate}
		\item \textit{Формулировка задачи стационарной теории возмущений.}
		
		Теория возмущений применяется для приближенного решения задач, которые не получается посчитать явно. Суть состоит в следующем: пусть возможно следующее разделение гамильтониана:
		\begin{equation}
			\hat{H} = \hat{H}_0 + \hat{V}
		\end{equation}
		и для оператора $\hat{H}_0$ задача решается. Тогда оператор $\hat{V}$ называют \textit{возмущением}. Далее решение стационарного уравнения ищется в следующем виде:
		\begin{equation}
			\ket{\psi_n} = \sum\limits_{k} c_k^{(n)} \ket{\psi_k^{(0)}} \label{disturbanceVectors}
		\end{equation}
		где $\ket{\psi_k^{(0)}}$ - собственные векторы оператора $\hat{H_0}$. Подставим волновую функцию в стационарное уравнение:
		\begin{equation}
			\sum\limits_{k}c_k^{(n)}\big(E_k^{(n)} \ket{\psi_k^{(0)}} + \hat{V}\ket{\psi_k^{(0)}}\big) = \sum\limits_{k}c_k^{(n)}E_k \ket{\psi_k^{(0)}}
		\end{equation}
		Умножим скалярно на $\bra{\psi_m^{(0)}}$:
		\begin{equation}
			\sum\limits_{k}c_k^{(n)}\big(E_k^{(n)} \delta_{km} + V_{km}\big) = \sum\limits_{k}c_k^{(n)}E_k \delta_{km}
		\end{equation}
		где $V_{nm}$ - матричный элемент оператора $\hat{V}$. Перенесем все слагаемые в левую часть:
		\begin{equation}
			\sum\limits_{k}c_k^{(n)}\big(V_{km} - (E_k^{(n)} - E_k) \delta_{km}\big) = 0
		\end{equation}
		Из этого уравнения находятся поправки к уровням энергии.
		
		\item \textit{Критерий применимости теории возмущений.}
		
		Ряд в формуле~\eqref{disturbanceVectors} может не сходиться. Обычно поправку к имеющимся состояниям ищут в виде:
		\begin{equation}
			\ket{\psi_n} = \ket{\psi_n^{(0)}} + \ket{\delta \psi_n}
		\end{equation}
		где $\braket{\psi_n^{(0)}|\delta \psi_n} = 0$. Для того, чтобы результаты приближенного вычисления слабо отличались от точного решения, необходимо, чтобы поправка была мала:
		\begin{equation}
			\|\ket{\delta \psi_n}\| \ll 1
		\end{equation}
		
		\item \textit{Поправка первого порядка к уровням энергии для невырожденного спектра.}
		
		\begin{equation}
			E_{n}^{(1)} = V_{nn}
		\end{equation}
		\item \textit{Поправки к состояниям невырожденного спектра в первом порядке теории возмущений.}
		
		\begin{equation}
			\ket{\delta\psi_n^{(1)}} = \sum\limits_{m\neq n} \frac{V_{mn}}{E_n^{(0)} - E_m^{(0)}} \ket{\psi_m^{(0)}}
		\end{equation}
		
		\item \textit{Поправка второго порядка к уровням энергии для невырожденного спектра.}
		
		\begin{equation}
			E_n^{(2)} = \sum\limits_{m\neq n} \frac{|V_{mn}|^2}{E_n^{(0)} - E_m^{(0)}}
		\end{equation}	
		
		\item \textit{Как изменится энергия основного невырожденного уровня системы, при помещении ее во внешнее поле?}	
		
		\item \textit{Постановка стационарной теории возмущений в случае вырожденного энергетического спектра.}
		При вырожденном спектре гамильтониана задача ставится несколько иначе:
		\begin{equation}
			\hat{H}_0 \ket{\psi^{(0)}_{n,\alpha}} = E_n^{(0)}\ket{\psi^{(0)}_{n,\alpha}}
		\end{equation}
		где $\alpha$ - параметр вырождения. Для такой ситуации выбор нулевых собственных векторов неоднозначен, так как любая комбинация вида
		\begin{equation}
			\ket{\psi^{(0)}_{n,\alpha}} = \sum\limits_{\beta} C_\beta \ket{\psi^{(0)}_{n,\beta}} 
		\end{equation}
		удовлетворяет уравнению Шредингера. Трюк состоит в том, чтобы найти такую линейную комбинацию, которая удовлетворяла бы критерию применимости стационарной теории возмущений. Найдем их, подставляя в уравнение Шредингера:
		\begin{equation}
			\sum\limits_{\beta} C_\beta \Big( \hat{V} - (E - E_n^{(0)}) \Big) \ket{\psi_{n,\beta}^{(0)}} = 0
		\end{equation}
		Домножив скалярно слева на $\bra{\psi_{n,\beta}^{(0)}}$, получаем:
		\begin{equation}
			\sum\limits_{\beta} C_\beta \Big( V_{\beta\alpha} - (E - E_n^{(0)})\delta_{\beta\alpha} \Big) = 0
		\end{equation}
		Это уравнение имеет нетривиальное решение тогда и только тогда, когда матрица $V_{\beta\alpha} - \varepsilon \delta_{\beta\alpha}$ вырождена. Уравнение на параметр $\varepsilon$, при котором матрица вырождена, называется \textit{секулярное уравнение}:
		\begin{equation}
			\det \|V_{\beta\alpha} - \varepsilon \delta_{\beta\alpha}\| = 0
		\end{equation}
		
		\item \textit{Правильные функции нулевого приближения и секулярное уравнение в теории возмущений.}
		
		Секулярное уравнение было определено выше. Правильные волновые функции это собственные векторы матрицы $V_{\beta\alpha} - \varepsilon \delta_{\beta\alpha}$.
		
		\item \textit{Уравнение для определения функции Грина стационарного уравнения Шредингера.}
		
		\begin{equation}
			(\hat{H} - E) G_E(\textbf{r}, \textbf{r}') = \delta(\textbf{r} - \textbf{r}')
		\end{equation}
		\item \textit{Определение функции Грина стационарного уравнения Шредингера как представление оператора.}
		
		Функцию Грина можно считать матрицей некоторого оператора:
		\begin{equation}
			G_E(\textbf{r}, \textbf{r}') = \braket{\textbf{r}|\hat{G}_E|\textbf{r}'}
		\end{equation}
		Найдем явный вид этого оператора:
		\begin{equation}
			\braket{\textbf{r}|(\hat{H} - E)\hat{G}_E | \textbf{r}'} = \delta (\textbf{r} - \textbf{r}') = \braket{\textbf{r}|\hat{1}| \textbf{r}'}
		\end{equation}
		Следовательно, оператор $\hat{G}_E$ равен:
		\begin{equation}
			\hat{G}_E = (\hat{H} - E)^{-1} = \frac{1}{\hat{H} - E}
		\end{equation}
		
		\item \textit{Ряд теории возмущений для функции Грина стационарного уравнения Шредингера.}
		
		Переходя к теории возмущений, зная оператор Грина невозмущенной задачи $\hat{G}^{(0)}$, можно переписать выражение для $\hat{G}$:
		\begin{equation}
			\hat{G} = (\hat{H}_0 - E + \hat{V})^{-1} = \hat{G}_0 (\hat{1} - \hat{V}\hat{G})
		\end{equation}
		
		
		\item \textit{Что определяют полюса функции Грина стационарного уравнения Шредингера?}				
		
		
		\clearpage		
		Чтобы понять, каким образом поправки классифицируются по порядкам, необходимо ввести оператор Грина для данной задачи. В обычных дифференциальных задачах функция Грина позволяет найти частное решение для неоднородного дифференициального уравнения
		\begin{equation}
			\psi_{\text{ч}}(\textbf{r}) = \int G(\textbf{r},\textbf{r}') f(\textbf{r}') d\textbf{r}'
		\end{equation}
		для уравнения
		\begin{equation}
			\hat{H} \psi = f(\textbf{r})
		\end{equation}
		Функция Грина определяется из уравнения
		\begin{equation}
			\hat{H}G = \delta(\textbf{r}-\textbf{r}')
		\end{equation}
		Для задачи теории возмущения стационарное уравнение Шредингера переписывается в следующем виде:
		\begin{equation}
			\Big(\hat{H}_0 - E \Big)\ket{\psi} = - \hat{V}\ket{\psi}
		\end{equation}
		Найдем выражение для оператора Грина:
		\begin{equation}
			\braket{\textbf{r}| (\hat{H}_0 - E) \hat{G}|\textbf{r}'} = \delta(\textbf{r}-\textbf{r}') = \braket{\textbf{r}|\hat{1}|\textbf{r}'}
		\end{equation}
		Следовательно, явный вид оператора Грина
		\begin{equation}
			\hat{G} = (\hat{H}_0 - E)^{-1} = \frac{1}{\hat{H}_0 - E}
		\end{equation}
	\end{enumerate}
\end{document}
