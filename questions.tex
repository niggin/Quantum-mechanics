\documentclass{article}
\usepackage[left=3cm,right=3cm,top=2cm,bottom=2cm]{geometry} % page settings

\usepackage{amsfonts,amssymb}
\usepackage[utf8]{inputenc}
\usepackage[russian]{babel}
%\usepackage[dvips]{graphicx}
\usepackage{amsmath}
\usepackage{amsfonts}
\usepackage{amsthm}
%\usepackage{MnSymbol}
\usepackage{tikz}
\usepackage{braket}

%\setlength{\parindent}{0mm}
\usepackage{graphicx}
\newcommand{\indep}{\rotatebox[origin=c]{90}{$\models$}}

\begin{document}

\title{Вопросы по курсу "Квантовая механика"}
\author{Н. Попов, М. Славошевский}
\date{\today}
\maketitle

{\center{\section*{Квантовая механика, часть 1}}}

\subsection*{Постулаты}
	\begin{enumerate}
		\item {Как связаны между собой вектор кет $\ket{\psi}$ и вектор бра $\bra{\psi} $?}
		\begin{equation}
			\ket{\psi}=\bra{\psi}^{+}
		\end{equation}	
	\item {Задано скалярное произведение двух векторов состояния $C =\braket{\varphi | \psi}$. Чему равно $\braket{\psi | \varphi}$?}
	\begin{equation}
		\braket{\psi|\varphi}=C^{*}
	\end{equation}
	
	\item {Пусть $\braket{\psi_1 | \psi_2} = 0$. Какой смысл имеют коэффициенты $c_1$ и $c_2$ в суперпозиции $\ket{\psi} = c_1 \ket{\psi_1} + c_2 \ket{\psi_2} $?}
	\begin{equation}
		c_{i}=\braket{\psi_{i}|\psi}
	\end{equation}
	\item {Задан вектор состояния в виде суперпозиции двух состояний $\ket{\psi} = c_1 \ket{\psi_1} + c_2 \ket{\psi_2}$. Как
определяется вектор бра $\bra{\psi}$?}
	\begin{equation}
		\bra{\psi}=c_{1}^{*}\bra{\psi_{1}}+c_{2}^{*}\bra{\psi_{2}}
	\end{equation}
	\item {Задан оператор физической величины $\hat{f}$. Как определяется наблюдаемая (физическая величина) квантовой системы, находящейся в состоянии $\ket{\psi}$?
}	
	\begin{equation}
		f=\bra{\psi}\hat{f}\ket{\psi}
	\end{equation}
	
	\end{enumerate}

\subsection*{Уравнение Шредингера}
\begin{enumerate}
	\item {Записать уравнение, которому подчиняется вектор состояния квантовой системы.} 
	\begin{gather}
			i \hbar \frac{\partial \ket{\psi (t)}}{\partial t} = \hat{H} \ket{\psi (t)} \label{shrodingerEquation}\\
		\ket{\psi(0)} = \ket{\psi_0}
	\end{gather}
	где $\hat{H}$ - гамильтониан системы, $\ket{\psi_0}$ - начальное состояние.
	
	\item {Записать стационарное уравнение Шредингера. Какой вид имеет оператор Гамильтона в общем случае?}
	
	Пусть собственные состояния оператора $\hat{H}$ задают базис пространства состояний, тогда любое состояние можно разложить по этому базису:
	\begin{equation}
		\ket{\psi(t)} = \sum\limits_n c_n (t) \ket{n} \label{basisDecomposition}
	\end{equation}
	Так как $\hat{H}\ket{n} = E_n \ket{n}$, то уравнение Шредингера перепишется в следующем виде:
	\begin{equation}
		\sum\limits_n \Big(i\hbar \frac{\partial c_n(t)}{\partial t} - c_n(t) E_n \Big) \ket{n} = 0
	\end{equation}
	Так как собственные вектора образуют базис, то в уравнении Шредингера все коэффициенты равны нулю, откуда получаем искомое состояние:
	\begin{equation}
		\ket{\psi(t)} = \sum\limits_n c_n^{(0)} e^{-\frac{i}{\hbar} E_n t} \ket{n}
	\end{equation}
	Уравнение, из которого определяются собственные состояния оператора Гамильтона, называется {стационарное уравнение Шредингера}:
	\begin{equation}
		\hat{H} \ket{\psi} = E \ket{\psi}
	\end{equation}
	Для консервативной системы оператор Гамильтона имеет вид\footnote{В общем случае это не всегда верно. Из теоретической механики известно, что гамильтониан системы выражается через кинетическую и потенциальную энергию следующим образом:
	\begin{equation}
		H = T_2 + U - T_0 \label{trueHamultonian}
	\end{equation}
	где $T_2$ - квадратичная по скорости часть кинетической энергии, $T_0$ - не зависящая от скорость часть. Если же $\frac{\partial \textbf{r}}{\partial t} = 0$, то формула~\eqref{trueHamultonian} переходит в формулу~\eqref{falseHamultonian}
	}:
	\begin{equation}
		\hat{H} = \hat{T} + \hat{U} \label{falseHamultonian}
	\end{equation}
	где $\hat{T}$ - оператор кинетической энергии системы, $\hat{U}$ - потенциальной.
	\item {Записать волновую функцию свободной нерелятивистской частицы.}
	
	Гамильтониан свободной нерелятивистской частицы имеет вид
	\begin{equation}
		\hat{H} = \frac{\textbf{p}^2}{2m} = -\frac{\hbar^2}{2m} \Delta
	\end{equation}
	в координатном представлении. Из линейности лапласиана следует, что решение стационарного уравнения можно искать в виде $\Psi(\textbf{r}) = \psi_x(x) \psi_y(y) \psi_z(z)$. Подставим такую волновую функцию:
	\begin{equation}
		\frac{d^2 \psi_x(x)}{d x^2}\psi_y(y)\psi_z(z) + \psi_x(x)\frac{d^2 \psi_y(y)}{d y^2}\psi_z(z) + \psi_x(x)\psi_y(y)\frac{d^2 \psi_z(z)}{d z^2} = -\frac{2mE}{\hbar^2} \psi_x(x)\psi_y(y)\psi_z(z) \label{freeParticle}
	\end{equation}
	Поделим равенство~\eqref{freeParticle} на волновую функцию:
	\begin{equation}
		\frac{d^2 \psi_x(x)}{d x^2} \frac{1}{\psi_x(x)} + \frac{d^2 \psi_y(y)}{d y^2} \frac{1}{\psi_y(y)} + \frac{d^2 \psi_z(z)}{d z^2} \frac{1}{\psi_z(z)} = -\frac{2mE}{\hbar^2}
	\end{equation}
	Получилась сумма функций, каждая из которой зависит от отдной из пространственных переменных, причем их сумма равна константе. Следовательно, каждая из этох функций постоянна. Из физических соображений очевидно, что кинетическая энергия частицы не может быть отрицательной. Обозначим $k_\alpha^2 =\frac{2m E_\alpha}{\hbar^2}, E = E_x + E_y + E_z$:
	\begin{equation}
		\psi_\alpha''(\alpha) + k^2_\alpha \psi_\alpha(\alpha) = 0
	\end{equation}
	Решением этого уравнения является
	\begin{equation}
		\psi_\alpha(\alpha) = C_1 e^{i k_\alpha \alpha} + C_2 e^{-i k_\alpha \alpha}
	\end{equation}
	Физический смысл каждой из двух функций в том, что первая описывает движение частицы в $+\infty$, а вторая - в противоположном направлении. Координатная часть волновой функции $\Psi(\textbf{r})$ равна:
	\begin{equation}
		\Psi(\textbf{r}) = C_1 e^{i\textbf{kr}} + C_2 e^{-i\textbf{kr}}
	\end{equation}
	
	Полная волновая функция, соответственно, равна:
	\begin{equation}
		\psi(\textbf{r},t) = C e^{i(\textbf{kr} - \frac{E}{\hbar}t)} = C e^{i(\textbf{kr} - \omega t)}
	\end{equation}
	
	\item {Как с помощью оператора эволюции записать решение уравнения Шредингера в произвольный момент времени, если задано начальное условие $\Psi(\textbf{r}, 0) = \psi_0(\textbf{r}) $?}
	
	От дифференциального вида уравнения Шредингера можно перейти к интегральному:
	\begin{equation}
		\Psi(\textbf{r}, t) = \psi_0(\textbf{r}) - \frac{i}{\hbar} \int\limits_0^t \hat{H} \Psi(\textbf{r}, \tau) d\tau
	\end{equation}
	Пусть гамильтониан системы не зависит от времени. Будем решать уравнение методом последовательных приближений. Пусть $\Psi^{(0)}(\textbf{r}, t) = \psi_0 (\textbf{r})$, тогда
	\begin{equation}
		\Psi^{(1)}(\textbf{r}, t) = \psi_0 (\textbf{r}) - \frac{it\hat{H} }{\hbar} \psi_0 (\textbf{r})
	\end{equation}
	Повторяя приближение бесконечное число раз, в пределе получим:
	\begin{equation}
		\Psi(\textbf{r}, t) = \sum\limits_{n=0}^\infty \frac{1}{n!} \Big(\frac{-it\hat{H} }{\hbar} \Big)^n \psi_0 (\textbf{r}) = e^{\frac{-it\hat{H} }{\hbar}} \psi_0 (\textbf{r}) = \hat{U}(t) \psi_0 (\textbf{r})
	\end{equation}
	где введен оператор эволюции $\hat{U}(t) = e^{\frac{-it\hat{H} }{\hbar}}$
	
	\item {Какой вид имеет оператор эволюции консервативной системы?}
	\begin{equation}
		\hat{U}(t) = e^{\frac{-it\hat{H} }{\hbar}}
	\end{equation}
	
	\item {Записать определение производной оператора по времени.}
		
	
	Производной физической величины можно сопоставить производную оператора, который задает эту величину, то есть, по определению:
	\begin{equation}
		\braket{\frac{d\hat{f}}{dt}} = \frac{d}{dt}\braket{f}
	\end{equation}
	Найдем его явный вид:
	\begin{equation}
		\frac{df}{dt} = \frac{d}{dt} \braket{\psi(t) | \hat{f} | \psi(t)} = \frac{d}{dt} \braket{\psi_0 |\hat{U}^+ \hat{f} \hat{U} | \psi_0} = \bra{\psi_0} \frac{\partial \hat{U}^+}{\partial t} \hat{f} \hat{U} + \hat{U}^+\frac{\partial \hat{f}}{\partial t} \hat{U} + \hat{U}^+ \hat{f}\frac{\partial \hat{U}}{\partial t} \ket{\psi_0} \label{derivativeFirst}
	\end{equation}
	По определению оператора эволюции его производная равна
	\begin{equation}
		\frac{\partial U}{\partial t} = -\frac{i}{\hbar} \hat{H} \hat{U}
	\end{equation}
	Тогда равенство~\eqref{derivativeFirst} можно продолжить:
	\begin{equation}
		\frac{df}{dt} = \bra{\psi_0} \hat{U}^+ \Big( \frac{\partial \hat{f}}{\partial t} + \frac{i}{\hbar} \big( \hat{H} \hat{f} - \hat{f} \hat{H} \big) \Big)\hat{U}\ket{\psi_0} = \bra{\psi(t)}\frac{\partial \hat{f}}{\partial t} + \frac{i}{\hbar} \Big[ \hat{H}, \hat{f}\Big] \ket{\psi(t)}
	\end{equation}
	где квадратные скобки обзначают {коммутатор} двух операторов. Следовательно, явный вид оператора производной по времени следующй:
	\begin{equation}
	\frac{d\hat{f}}{dt} = \frac{\partial \hat{f}}{\partial t} + \frac{i}{\hbar} \Big[ \hat{H}, \hat{f}\Big] \label{derivativeLast}
	\end{equation}
	\item {Как определяется коммутатор двух операторов?}
	\begin{equation}
		\Big[ \hat{A}, \hat{B} \Big] = \hat{A} \hat{B} - \hat{B} \hat{A}
	\end{equation}
	\item {
 Какому условию удовлетворяют операторы физических величин–интегралов движения в квантовой механике?	
	}
	
	Интеграл движения - величина, сохранающаяся с течением времени. Так как мы знаем оператор, который сопоставляется производной физической величины (формула~\eqref{derivativeLast}), то для интегралов движения этот оператор должен равняться нулю:
	\begin{equation}
		\frac{\partial \hat{f}}{\partial t} + \frac{i}{\hbar} \Big[ \hat{H}, \hat{f}\Big] = 0
	\end{equation}
	\item {Какие физические величины могут быть включены в полный набор физических величин, определяющих состояние квантовой системы?}
	
	Полный набор физических величин - максимальный набор физических величин, которые можно измерить одновременно.
	
	\item {Как можно представить коммутатор $\Big[ \hat{A} \hat{B}, \hat{C}\Big]$?}
	
	\begin{equation}
		\Big[ \hat{A} \hat{B}, \hat{C}\Big] = \hat{A}\hat{B}\hat{C} - \hat{C}\hat{A}\hat{B} \pm \hat{A} \hat{C} \hat{B} = \hat{A}(\hat{B}\hat{C} - \hat{C} \hat{B}) + (\hat{A}\hat{C} - \hat{C} \hat{A}) \hat{B} = \hat{A}\Big[ \hat{B}, \hat{C}\Big] + \Big[ \hat{A}, \hat{C}\Big] \hat{B} \label{abc}
	\end{equation}
	
	\item {Определить полный набор физических величин свободной бесспиновой частицы.}
	
	Полный набор физичеких величин для свободной частицы состоит из импульса частицы $\textbf{p}$.

\end{enumerate}

\subsection*{Операторы и теория представлений}

\begin{enumerate}
	\item {Чему равен (“табличный”) коммутатор $\big[\hat{x}_\alpha, \hat{p}_\beta \big]$?}
	
	Коммутаторы операторов физических величин постулируются - для двух физических случайных величин коммутатор пропорционален скобке Пуассона этих двух величин, в которых физические величины заменены на их операторы:
	\begin{equation}
		\Big\{ f, g \Big\} \rightarrow -\frac{i}{\hbar} \Big[\hat{f}, \hat{g}\Big]
	\end{equation}
	По определению, скобка Пуассона равна:
	\begin{equation}
		\Big\{ f, g \Big\} = \sum\limits_n \frac{\partial f}{\partial q_n} \frac{\partial g}{\partial p_n} - \frac{\partial f}{\partial p_n} \frac{\partial g}{\partial q_n}
	\end{equation}
	где $q_n, p_n$ - обобщенные координатв и импульс соответственно. Для самих координаты и импульса скобка Пуассона равна:
	\begin{equation}
		\Big\{ x_\alpha, p_\beta \Big\} = \delta_{\alpha \beta}
	\end{equation}
	и, следовательно коммутатор их операторов равен:
	\begin{equation}
		\big[\hat{x}_\alpha, \hat{p}_\beta \big] = i\hbar \delta_{\alpha\beta}
	\end{equation}
	
	\item {Зная табличный коммутатор операторов координаты и импульса, определить оператор скорости нерелятивистской частицы.}
	
	Воспользуемся определением производной оператора по времени:
	\begin{equation}
		\hat{\textbf{v}} = \frac{d \textbf{r}}{dt} = \frac{\partial \textbf{r}}{\partial t} + \frac{i}{\hbar} \Big[ \hat{H}, \hat{\textbf{r}} \Big]
	\end{equation}
	Так как потенциальная энергия является функцией координаты, то её оператор коммутирует с оператором координаты. Посчитаем коммутатор оператора кинетической энергии с оператором координаты:
	\begin{equation}
		\Big[ \hat{T}, \hat{\textbf{r}} \Big] = \frac{1}{2m} \Big[ \hat{\textbf{p}}^2, \hat{\textbf{r}} \Big] \stackrel{~\eqref{abc}}{=} \frac{1}{2m} \Big(\hat{p}_\alpha \Big[\hat{p}_\alpha, \hat{\textbf{r}}\Big] + \Big[\hat{p}_\alpha, \hat{\textbf{r}}\Big]\hat{p}_\alpha \Big) = -\frac{i\hbar\hat{\textbf{p}}}{m}
	\end{equation}
	И тогда оператор скорости, как и в классическом случае, равен
	\begin{equation}
		\hat{\textbf{v}} = \frac{\hat{\textbf{p}}}{m}
	\end{equation}
	\item {Как можно записать оператор, соответствующий физической величине \textbf{pr}?}
	
	Физическим величинам соответствуют эрмитовые операторы. Попробуем сопоставить величине $\varphi = p_\alpha x_\alpha$ оператор напрямую, и посчитаем эрмитово сопряженный оператор $\hat{\varphi}^+$:
	\begin{equation}
		\varphi^+ = (\hat{p}_\alpha \hat{x}_\alpha)^+ = \hat{x}_\alpha \hat{p}_\alpha = \hat{p}_\alpha \hat{x}_\alpha - \big[\hat{p}_\alpha, \hat{x}_\alpha \big] = \hat{p}_\alpha \hat{x}_\alpha + i\hbar \delta_{\alpha\alpha} = \hat{p}_\alpha \hat{x}_\alpha + 3i\hbar
	\end{equation}
	Как видно, напрямую сопоставленный оператор неэрмитов; однако, если взять оператор $\hat{\varphi} = \hat{p}_\alpha \hat{x}_\alpha + \frac{3}{2}i\hbar$, то такой оператор является эрмитовым. При переходе к классической механике ($\hbar \to 0$) введенный оператор переходит в величину $\varphi$, следовательно, введенный таким образом оператор соответствует заданной величине.
	
	Канонически верным оператором, который можно сопоставить величине $\textbf{pr}$, является следующий оператор:
	\begin{equation}
		\hat{\varphi} = \frac{\hat{p}_\alpha \hat{x}_\alpha + \hat{x}_\alpha \hat{p}_\alpha}{2}
	\end{equation}
	Этот оператор автоматически является эрмитовым и при предельном переходе к классической механике дает исходную величину $\textbf{pr}$.
	
	\item {Что определяет выражение }$\braket{\textbf{r}|\Psi}${ = ?}
	
	Функция $\psi(\textbf{r}) = \braket{\textbf{r}|\Psi}$ (которую называют {волновой функцией}) является проекцией состояния $\ket{\Psi}$ на базис собственных состояний оператора $\hat{\textbf{r}}$; другими словами, это состояние $\ket{\psi}$ в координатном представлении. Квадрат модуля этой функции есть плотность вероятности того, что частица находится в точке $\textbf{r}$.
	
	\item {Для оператора координаты $\hat{\textbf{r} } \ket{\textbf{r}_0} = \ ?$}
	
	По определению, состояние $\ket{\textbf{r}_0}$ - состояние с определенной координатой, то есть это собственное состояние оператора $\hat{\textbf{r}}$. Следовательно,
	\begin{equation}
		\hat{\textbf{r}}\ket{\textbf{r}_0} = \textbf{r}_0 \ket{\textbf{r}_0}
	\end{equation}
	
	\item {Для оператора импульса  $\hat{\textbf{p} } \ket{\textbf{p}_0} = \ ?$}
	
	Аналогично оператору координаты:
	\begin{equation}
		\hat{\textbf{p}}\ket{\textbf{p}_0} = \textbf{p}_0 \ket{\textbf{p}_0}
	\end{equation}
	
	\item {Пусть совокупность векторов $\ket{n}$ -составляет базис. Чему равен оператор $\sum\limits_n \ket{n}\bra{n} = \ ?$}
	
	Так как вектора $\ket{n}$ состовляют базис, то любое состояние можно разложить по этому базису (смотри, например, формулу~\eqref{basisDecomposition}). Посмотрим, как этот оператор действует на произвольное состояние:
	\begin{equation}
		\sum\limits_n \big(\ket{n}\bra{n}\big) \ket{\psi} = \sum\limits_{n, n'} \big(\ket{n}\bra{n}\big) c_{n'}\ket{n'} = \sum\limits_{n, n'} c_{n'}\ket{n}\braket{n|n'} = \sum\limits_{n, n'} c_{n'}\ket{n}\delta_{nn'} = \sum\limits_{n,} c_{n}\ket{n} = \ket{\psi}
	\end{equation}
	то есть оператор не меняет произвольное состояние. Значит, этот оператор равен тождественному оператору $\hat{1}$.
	
	\item {Для некоторого (не обязательно эрмитова) оператора $\hat{f}\ket{f_n} = f_n\ket{f_n}$, чему равно $\bra{f_n}\hat{f} = \ ?$}
	
	Пусть $\bra{f_n}\hat{f} = \bra{\psi}$, спроецируем $\bra{\psi}$ на базисные вектора $\bra{f_{n'}}$:
	\begin{equation}
		\braket{\psi|f_{n'}} = \braket{f_n|\hat{f}|f_{n'}} = f_{n'} \braket{f_n|f_{n'}} = f_{n'} \delta_{nn'}
	\end{equation}
	то есть проекции на вектора, отличные от $\bra{f_n}$, равны нулю. Значит, $\bra{\psi} = f_n\bra{f_n}$
	
	\item {Пусть $\hat{f}^+=\hat{f}$, а $\hat{f}\ket{f_n} = f_n \ket{f_n}$, чему равен оператор $\sum\limits_n \ket{f_n}\bra{f_n} = \ ?$}
	
	У эрмитового оператора всегда существует базис из собственных векторов, следовательно, оператор равен тождетвенному (смотри вопрос 7).
	
	\item $\braket{\textbf{r}'|\hat{\textbf{r}}|\textbf{r}} = \ ?$
	\begin{equation}
		\braket{\textbf{r}'|\hat{\textbf{r}}|\textbf{r}} = \textbf{r} \braket{\textbf{r}'|\textbf{r}} =\textbf{r} \delta(\textbf{r}' - \textbf{r})
	\end{equation}
	где $\delta(\textbf{r})$ - дельта-функция Дирака.
	
	\item $\braket{\textbf{r}|\textbf{p}} = \ ?$
	
	Это значение есть ни что иное, как волновая функция состояния с определенным импульсом. Задача решается при помощи некоторого искусственого приема. Введем оператор
	\begin{equation}
		\hat{Q}_{\textbf{a}} = e^{-\frac{i}{\hbar}\textbf{a}\hat{\textbf{p}}}
	\end{equation}
	Этот оператор является функцией оператора координаты. Нам нужно вычислить его коммутатор с оператором координаты. Для этого вычислим коммутатор $\big[\hat{x}_\alpha,\hat{p}_\alpha^l \big]$ при помощи скобок Пуассона:
	\begin{equation}
		\Big\{x, p_x \Big\} = \frac{\partial x}{\partial x} \frac{\partial p_x^l}{\partial p_x} = \frac{\partial p_x^l}{\partial p_x}
	\end{equation}
	Следовательно, коммутатор $\big[\hat{x}_\alpha,\hat{p}_\alpha^l \big]$ равен:
	\begin{equation}
		\big[\hat{x}_\alpha,\hat{p}_\alpha^l \big] = i\hbar \frac{\partial \hat{\textbf{p}}^l}{\partial \hat{\textbf{p}}}
	\end{equation}
	И коммутатор $\hat{Q}_\textbf{a}$ с оператором координаты, по определению функции от оператора, равен
	\begin{equation}
		\Big[\hat{\textbf{r}},\hat{Q}_\textbf{a} \Big] = i\hbar \frac{\partial \hat{Q}_\textbf{a}}{\partial \hat{\textbf{p}}} = \textbf{a}\hat{Q}_\textbf{a}
	\end{equation}
	Теперь подействуем оператором $\hat{\textbf{r}}\hat{Q}_\textbf{a}$ на состояние с определенной координатой:
	\begin{equation}
		\hat{\textbf{r}}\hat{Q}_\textbf{a} \ket{\textbf{r}_0} = \Big(\hat{Q}_\textbf{a}\hat{\textbf{r}} + \hat{\textbf{a}}\hat{Q}_\textbf{a}\Big)\ket{\textbf{r}_0} = (\textbf{r}_0 + \textbf{a})\hat{Q}_\textbf{a} \ket{\textbf{r}_0}
	\end{equation}
	То есть вектор $\hat{Q}_\textbf{a} \ket{\textbf{r}_0}$ есть собственный вектор оператора координаты. Очевидно, что если взять в качестве вектора $\textbf{a}$ необходимую координату, и действовать на состояние с нулевой координатой, то мы получим состояние с необходимой координатой. Следовательно, оператор $\hat{Q}_\textbf{a}$ есть оператор трансляции. Аналогичным образом можно определить оператор трансляции $\hat{P}_\textbf{k}$ для импульсного представления. Тогда
	\begin{equation}
		\braket{\textbf{r}|\textbf{p}} = \braket{\textbf{r}|\hat{P}_\textbf{p}|0}_p = e^{-\frac{i}{\hbar}\textbf{pr}} \braket{\textbf{r}|0}_p = {e^{-\frac{i}{\hbar}\textbf{pr}}}_r \braket{0|\hat{Q}_\textbf{r}^+|0}_p = {e^{-\frac{i}{\hbar}\textbf{pr}}}_r \braket{0|0}_p \label{rp}
	\end{equation}
	Остается только определить константу $_r\braket{0|0}_p$. Для этого можно сделать следующее:
	\begin{equation}
		\braket{\textbf{p}|\textbf{p}'} = \delta(\textbf{p} - \textbf{p}') = \int d\textbf{r} \braket{\textbf{p}|\textbf{r}}\braket{\textbf{r}|\textbf{p}'}
	\end{equation}
	Подставляя в интеграл значение~\eqref{rp}, получаем:
	\begin{equation}
		|_r\braket{0|0}_p|^2 \int d\textbf{r} e^{\frac{i}{h}(\textbf{p}' - \textbf{p})\textbf{r}} = \delta(\textbf{p} - \textbf{p}')
	\end{equation}
	Интеграл в равенстве пропорционален дельта-функции, значит, искомый коэффициент равен:
	\begin{equation}
		|_r\braket{0|0}_p|^2 = (2\pi\hbar)^{-3}
	\end{equation}
	Тогда искомое выражение равно:
	\begin{equation}
		\braket{\textbf{r}|\textbf{p}} = \frac{1}{(2\pi\hbar)^{\frac{3}{2}}} e^{\frac{i}{\hbar}\textbf{pr}}
	\end{equation}
	
	\item $\braket{\textbf{p}|\textbf{r}} = \ ?$
		
	По определению,
	\begin{equation}
		\braket{\textbf{p}|\textbf{r}} = (\braket{\textbf{r}|\textbf{p}})^* = \frac{1}{(2\pi\hbar)^{\frac{3}{2}}} e^{-\frac{i}{\hbar}\textbf{pr}}
	\end{equation}
	
	\item {Записать волновую функцию свободной нерелятивистской частицы.}
	
	Смотри вопрос №3 раздела "Уравнение Шредингера"
	\item $\braket{\textbf{r}'|\textbf{r}} = \ ?$
	
	Векторы $\ket{\textbf{r}}$ - базисные вектора, нормированные на дельта-функцию:
	\begin{equation}
		\braket{\textbf{r}'|\textbf{r}} = \delta(\textbf{r}' - \textbf{r})
	\end{equation}
	
	\item $\braket{\textbf{p}'|\textbf{p}} = \ ?$
	
	Аналогично состояниям с определенной координатой:
	\begin{equation}
		\braket{\textbf{p}'|\textbf{p}} = \delta(\textbf{p}' - \textbf{p})	
	\end{equation}
	
	\item {Записать уравнение Шредингера для частицы с массой $m$ в координатном представлении.}
	
	Для того, чтобы перейти в координатное представление, нам нужно выяснить, как действуют операторы координаты и импульса на волновую функцию. Рассмотрим действие оператора координаты: пусть $\ket{\varphi} = \hat{\textbf{r}}\psi$, тогда
	\begin{equation}
		\braket{\textbf{r}|\varphi} = \braket{\textbf{r}|\hat{\textbf{r}}|\psi} = \braket{\textbf{r}|\hat{\textbf{r}}\hat{1}|\psi} = \int d\textbf{r}' \braket{\textbf{r}|\hat{\textbf{r}}|\textbf{r}'} \braket{\textbf{r}'|\psi} = \int d\textbf{r}' \textbf{r}' \delta(\textbf{r} - \textbf{r}') \psi(\textbf{r}') = \textbf{r}\psi(\textbf{r})
	\end{equation}
	Таким образом, действие оператора координаты на волновую функцию аналогично действию оператора координаты на состояние:
	\begin{equation}
		\hat{\textbf{r}}\psi(\textbf{r}) = \textbf{r} \psi(\textbf{r})
	\end{equation}
	Теперь выясним действие оператора импульса на волновую функцию. Пусть $\ket{\phi} = \hat{\textbf{p}}\ket{\psi}$. Тогда
	\begin{equation}
		\braket{\textbf{r}|\phi} = \braket{\textbf{r}|\hat{\textbf{p}}\hat{1}|\psi} = \int d\textbf{r}' \braket{\textbf{r}|\hat{\textbf{p}}|\textbf{r}'} \psi(\textbf{r}') \label{imulseFirst}
	\end{equation}
	Найдем матричный элемент оператора $\hat{\textbf{p}}$ в координатном предствалении:
	\begin{equation}
	\begin{split}
		\braket{\textbf{r}|\hat{\textbf{p}}|\textbf{r}'} = \braket{\textbf{r}|\hat{1}\hat{\textbf{p}}\hat{1}|\textbf{r}'} = \int d\textbf{p}d\textbf{p}'\braket{\textbf{r}|\textbf{p}} \braket{\textbf{p}|\hat{\textbf{p}}|\textbf{p}'} \braket{\textbf{p}'|\textbf{r}'} = \int d\textbf{p} \textbf{p} \braket{\textbf{r}|\textbf{p}} \braket{\textbf{p}|\textbf{r}'} =\\= \frac{1}{(2\pi\hbar)^3} \int d\textbf{p} \textbf{p} e^{\frac{i}{\hbar}\textbf{p}(\textbf{r} - \textbf{r}')} = i\hbar \frac{\partial}{\partial \textbf{r}'}\Big(\frac{1}{(2\pi\hbar)^3}\int d\textbf{p} e^{\frac{i}{\hbar}\textbf{p}(\textbf{r} - \textbf{r}')}  \Big) = i\hbar \frac{\partial}{\partial \textbf{r}'} \delta(\textbf{r} - \textbf{r}')
		\end{split}
	\end{equation}
	Подставим получившееся выражение в~\eqref{imulseFirst}:
	\begin{equation}
		\braket{\textbf{r}|\phi} = i\hbar \int d\textbf{r}' \Big(\frac{\partial}{\partial \textbf{r}'} \delta(\textbf{r} - \textbf{r}')\Big) \psi(\textbf{r}') = - i\hbar \int d\textbf{r}' \delta(\textbf{r} - \textbf{r}') \frac{\partial \psi}{\partial \textbf{r}'} = - i\hbar \frac{\partial \psi}{\partial \textbf{r}}
	\end{equation}
	Следовательно, действие оператора импульса на волновую функцию сводится к дифференцированию:
	\begin{equation}
		\hat{\textbf{p}}\psi(\textbf{r}) = -i\hbar \frac{\partial \psi}{\partial \textbf{r}}
	\end{equation}
	Теперь можно легко записать уравнение Шредингера в координатном представлении:
	\begin{equation}
		i\hbar \frac{\partial \psi}{\partial t} = -\frac{\hbar^2}{2m} \Delta \psi + U(\textbf{r})\psi
	\end{equation}
	
	\item {Записать стационарное уравнение Шредингера для частицы с массой m в импульсном представлении.}
	
	Аналогично предыдущему пункту легко найти, что оператор координаты в импульсном представлении выглядит следующим образом:
	\begin{equation}
		\hat{\textbf{r}} = i\hbar \nabla
	\end{equation}
	Тогда стационарное уравнение Шредингера в импульсном представлении запишется следующим образом:
	\begin{equation}
		\frac{\textbf{p}^2}{2m} \psi_\textbf{p} + U(i\hbar\nabla)\psi_\textbf{p} = E \psi_\textbf{p}
	\end{equation}
	Однако применять оператор $U(i\hbar\nabla)$ на практике очень неудобно. Для этого переходят к интегральному виду:
	\begin{equation}
		\braket{\textbf{p}|\hat{U}|\textbf{p}'} = \int d\textbf{r} d\textbf{r}' \braket{\textbf{p}|\textbf{r}} U(\textbf{r}')\delta(\textbf{r}-\textbf{r}')\braket{\textbf{p}'|\textbf{r}'} = \int \frac{d\textbf{r}}{(2\pi\hbar)^3} e^{-i\frac{(\textbf{p}-\textbf{p}')\textbf{r}}{\hbar}} U(\textbf{r}) = \frac{1}{(2\pi\hbar)^3}U_{\textbf{p}-\textbf{p}'}
	\end{equation}
	где $U_{\textbf{p}-\textbf{p}'}$ - фурье-образ потенциала. Тогда уравнение Шредингера запишется в следующем виде:
	\begin{equation}
		\frac{\textbf{p}^2}{2m} \psi_\textbf{p} + \int \frac{d\textbf{p}'}{(2\pi\hbar)^3} U_{\textbf{p}-\textbf{p}'}\psi_{\textbf{p}'} = E \psi_\textbf{p}
	\end{equation}
\end{enumerate}

\subsection*{Основные коммутационные соотношения}
\begin{enumerate}
	\item {Чему равен коммутатор $[x,\hat{p}_{x}]$?}
	\begin{equation}
		[x,\hat{p}_{x}]=i\hbar
	\end{equation}
	\item {Чему равен коммутатор $[\hat{x}_{\alpha},\hat{p}_{\beta}]$?}
	\begin{equation}
		[\hat{x}_{\alpha},\hat{p}_{\beta}]=i\hbar\delta_{\alpha\beta}
	\end{equation}
	\item {Чему равен коммутатор $[\hat{\boldsymbol{p}},U(\boldsymbol{r})]$?}
	\begin{equation}
		[\hat{\boldsymbol{p}},U(\boldsymbol{r})]=-i\hbar\nabla U(\boldsymbol{r})
	\end{equation}
	\item {Чему равен коммутатор $[\hat{l}_{x},\hat{l}_{y}]$?}
	\begin{equation}
		[\hat{l}_{x},\hat{l}_{y}]=i\hat{l}_{z}
	\end{equation}
	\item {Чему равен коммутатор $[\hat{l}_{\alpha},\hat{l}_{\beta}]$?}
	\begin{equation}
		[\hat{l}_{\alpha},\hat{l}_{\beta}]=ie_{\alpha\beta\gamma}\hat{l}_{\gamma}
	\end{equation}
	\item {Чему равен коммутатор $[\hat{l}_{\alpha},\hat{\boldsymbol{l}}^{2}]$?}
	\begin{equation}
		[\hat{l}_{\alpha},\hat{\boldsymbol{l}}^{2}]=0
	\end{equation}
	\item {Чему равен коммутатор $[\hat{l}_{\text{z}},\hat{l}_{+}]$?}
	\begin{equation}
		[\hat{l}_{\text{z}},\hat{l}_{+}]=\hat{l}_{+}
	\end{equation}
	\item {Чему равен коммутатор $[\hat{l}_{\text{z}},\hat{l}_{-}]$?}
	\begin{equation}
		[\hat{l}_{\text{z}},\hat{l}_{-}]=-\hat{l}_{-}
	\end{equation}
\end{enumerate}

\subsection*{Граничные условия}

\begin{enumerate}
	\item {Как формулируются граничные условия для нахождения связанных состояний?}
	
	В связном состоянии вероятность найти частицу на бесконечности равна нулю. Так как плотность вероятности определяется через модуль квадрата волновой функции, то сама волновая функция на бесконечности должна равняться нулю:
	\begin{equation}
		\left.\psi(\textbf{r})\right|_{|\textbf{r}| \to \infty} = 0
	\end{equation}
	
	\item {Как формулируются граничные условия для задач непрерывного спектра?}
	
	Первое условие всегда состоит в том, что волновая функция должна быть непрерывна. Второе условие определяется видом потенциала и находится напрямую из уравнения Шредингера. Проинтегрируем стационарное уравнение вокруг особенности потенциала (для простоты рассмотрим одномерную задачу с особенностью в нуле):
	\begin{equation}
		-\frac{\hbar^2}{2m}\int\limits_{-\varepsilon}^{\varepsilon}\psi''(x)dx + \int\limits_{-\varepsilon}^{\varepsilon} U(x)\psi(x)dx = E\int\limits_{-\varepsilon}^{\varepsilon}\psi(x)
	\end{equation}
	Устремляя $\varepsilon$ к нулю, получим следующее условие:
	\begin{equation}
		\psi'(+0) - \psi'(-0) = -\frac{2m}{\hbar^2} \lim_{\varepsilon \to +0}\int\limits_{-\varepsilon}^{\varepsilon} U(x)\psi(x)dx
	\end{equation}
	Как видно из получившейся формулы, в случае достаточно хорошей функции потенциала получаем требование непрерывности производной волновой функции.

\end{enumerate}

\subsection*{Осциллятор}
\begin{enumerate}
	\item {Записать гамильтониан линейного гармонического осциллятора.}
	
	Как известно еще из курса теоретической механики, функция Гамильтона одномерного осциллятора выглядит следующим образом:
	\begin{equation}
		H = \frac{p^2}{2m} + \frac{m\omega^2 x^2}{2}
	\end{equation}
	
	\item {Как определяются осцилляторные единицы энергии, длины, импульса?}
	
	Осцилляторной единицей энергии берется величина $E_0 = \hbar\omega$ - данная величина имеет размерность энергии. Пусть $E = \varepsilon E_0$. Разделим стационарное уравнение Шредингера на $E_0$:
	\begin{equation}
		\frac{1}{2}\big( \frac{\hat{p}^2}{m\hbar\omega} + \frac{m\omega}{\hbar}\hat{x}^2 \big)\ket{\psi_\varepsilon} = \varepsilon \ket{\psi_\varepsilon}
	\end{equation}
	Приняв за единицы координаты и импульса $x_0 = \sqrt{\frac{\hbar}{m\omega}}$ и $p_0 = \sqrt{\hbar m \omega}$ соответственно, получим канонический вид уравнения Шредингра для осциллятора:
	\begin{equation}
		\frac{1}{2}\big( \hat{P}^2 + \hat{Q}^2\big) \ket{\psi_\varepsilon} = \varepsilon \ket{\psi_\varepsilon}
	\end{equation}
	где $\hat{p} = p_0 \hat{P}, \ \hat{x} = x_0 \hat{Q}$. Размерность введенных единиц соответствует сопоставленным величинам.
	
	\item {Записать определение операторов $a$ и $a^+$ через операторы координаты и импульса.}
	
	Операторы $a$ и $a^+$ вводятся для упрощения вида гамильтониана. Определяются они следующим образом:
	\begin{equation}
		a = \frac{1}{\sqrt{2}}\big( \hat{Q} + i \hat{P} \big)
	\end{equation}
	где операторы $\hat{Q}, \ \hat{P}$ были введены в предыдущем вопросе. Сопряженный оператор задается прямым сопряжением:
	\begin{equation}
		a^+ = \frac{1}{\sqrt{2}}\big( \hat{Q} - i \hat{P} \big)
	\end{equation}
	Тогда оператор Гамильтона перепишется в следующем виде:
	\begin{equation}
		\hat{H} = \frac{1}{2}(aa^+ + a^+ a) \label{oscillatorFirst}
	\end{equation}
	
	\item {Выразить операторы координаты и импульса через операторы $a$ и $a^+$.}
	
	Решая систему линейных уравнений, находим:
	\begin{equation}
		\hat{Q} = \frac{1}{\sqrt{2}}(a + a^+), \ \hat{P} = \frac{1}{i\sqrt{2}}(a - a^+)
	\end{equation}
	
	\item {Чему равен коммутатор $[a,a^+] = \ ?$}
	\begin{equation}
		[a,a^+] = \frac{1}{2}[\hat{Q} + i\hat{P}, \hat{Q} - i\hat{P}] = \frac{1}{2}\Big(-i[\hat{Q},\hat{P}] + i[\hat{P},\hat{Q}] \Big) = \frac{i}{p_0 x_0}[\hat{p}, \hat{x}] = 1
	\end{equation}
	
	\item {Записать выражение гамильтониана осциллятора через $a$ и $a^+$.}
	
	Зная коммутатор операторов $a, a^+$, можно переписать выражение~\eqref{oscillatorFirst}:
	\begin{equation}
		\hat{H} = a^+a + \frac{1}{2}
	\end{equation}
	
	\item {Как определяется спектр осциллятора?}
	
	Рассмотрим спектр оператора $\hat{\nu} = a^+a$. Вычислим коммутатор $[a,a^+a]$:
	\begin{equation}
		[a,a^+a] = a^+[a,a] + [a,a^+]a = a
	\end{equation}
	Аналогичным образом $[a^+,a^+a] = a^+$. Пусть у нас есть собственное состояние $\ket{\nu}$ оператора $\hat{\nu}$, и $\ket{\psi} = a\ket{\nu}$. Рассмотрим действие оператора $\hat{\nu}$ на $\ket{\psi}$:
	\begin{equation}
		\hat{\nu}\ket{\psi} = \hat{\nu} a\ket{\nu} = (a\hat{\nu} - a)\ket{\nu} = (\nu - 1)a\ket{\nu} = (\nu - 1)\ket{\psi}
	\end{equation}
	Таким образом, вектор $a\ket{\nu}$ является собственным вектором оператора $\hat{\nu}$ со значением $\nu - 1$. Получается, что оператор $a$ понижает собственное число на единицу. Проделывая аналогичные выкладки, можно показать, что оператор $a^+$ повышает собственное значение на единицу. Заметим, что собственные значения оператора $\hat{\nu}$ неотрицательны:
	\begin{equation}
		\braket{\nu|a^+a|\nu} = \nu\braket{\nu|\nu} = \nu = \| a\ket{\nu}\|^2 \geq 0
	\end{equation}
	Следовательно, существует минимальное значение $\nu_0$ такое, что $a\ket{\nu_0} = 0$. Подействовав оператором $\hat{\nu}$ на это состояние выясняем, что минимальное собственное значени равно нулю:
	\begin{equation}
		a^+a\ket{\nu_0} = \nu_0 = a^+ (a\ket{\nu_0}) = 0
	\end{equation}
	Таким образом, спектр осциллятора дискретен, расстояние между значениями равно $\hbar\omega$ и минимальной энергией $\frac{\hbar\omega}{2}$.
	
	\item {Записать полный набор квантовых чисел, определяющих состояния одномерного гармонического осциллятора. Какие значения могут принимать квантовые числа?}
	
	Спектр оператора Гамильтона не является вырожденным, следовательно полный набор квантовых чисел для осциллятора будет состоять только из номера уровня. В предыдущем вопросе было показано, что спектр гамильтониана дискретен.
	
	\item {Чему равен результат действия оператора $a\ket{n} = \ ?$}
	
	Ранее было показано, что оператор $a$ понижает собственное число на единицу. Остается только выяснить коэффициент, на который домножается состояние. Пусть $a\ket{n} = \alpha_{n-1}\ket{n-1}$, тогда $\bra{n}a^+ = \alpha^*_{n-1}\bra{n-1}$, и
	\begin{equation}
		|\alpha_{n-1}|^2 = \braket{n|a^+a|n} = n
	\end{equation}
	Следовательно, $\alpha_{n-1} = \sqrt{n-1}e^{i\varphi}$. Опуская несущественный фазовый множитель, получаем искомый результат:
	\begin{gather}
		a\ket{n} = \sqrt{n}\ket{n-1} \\
		a^+\ket{n} = \sqrt{n+1} \ket{n+1}
	\end{gather}
	Действие оператора $a^+$ вычисляется аналогично.
	\item { Чему равно $a^+\ket{n} = \ ?$}
	
	Смотри предыдущий вопрос.
	\item {Чему равно $a\ket{0} = \ ?$}
	
	При поиске спектра было показано, что результат равен нулю. Аналогично ответ можно найти, воспользовавшись явным действием оператора $a$.
	\item {Выразить произвольное состояние осциллятора $\ket{n}$ через основное состояние $\ket{0}$.}
	
	Так как оператор $a^+$ повышает собственное число на единицу, то состояние $\ket{n}$ можно выразить через состояние $\ket{0}$ следующим образом:
	\begin{equation}
		\ket{n} = C_n (a^+)^n \ket{0}
	\end{equation}
	Остается только найти константу $C_n$. Зная явный вид действия оператора $a^+$, получаем:
	\begin{equation}
		\ket{n} = C_n \sqrt{n!}\ket{n}
	\end{equation}
	то есть константа $C_n = \frac{1}{\sqrt{n!}}$. Таким образом, получаем ответ:
	\begin{equation}
		\ket{n} = \frac{(a^+)^n}{\sqrt{n!}}\ket{0}
	\end{equation}
	
	\item {Какой вид имеет волновая функция основного состояния осциллятора $\psi(Q)$= (в безразмерных единицах)?}
	
	Волновая функция $\psi(Q) = \braket{Q|0}$. Для нахождения явного вида воспользуемся искусственным приемом - найдем явный вид $\braket{Q|a|0}$ (это значение равно нулю):
	\begin{equation}
	\begin{split}
		\braket{Q|a|0} = \frac{1}{\sqrt{2}}\braket{Q|\hat{Q} + i \hat{P}|0} = \frac{1}{\sqrt{2}}\int dQ' \Big( \braket{Q|\hat{Q}|Q'} + i\braket{Q|\hat{P}|Q'} \Big)\braket{Q'|0} = \\ =\frac{1}{\sqrt{2}}\int dQ'\Big(Q'\delta(Q-Q') - \frac{\partial}{\partial Q'}\delta(Q-Q') \Big) \psi(Q) = \frac{1}{\sqrt{2}} \Big(Q\psi(Q) + \psi'(Q)\Big) = 0
	\end{split}
	\end{equation}	
	Решая получившееся дифференциальное уравнение, получаем:
	\begin{equation}
		\psi(Q) = A e^{-\frac{1}{2}Q^2}
	\end{equation}
	Нормируя квадрат полученной функции, находим $A = \pi^{-\frac{1}{4}}$.
\end{enumerate}


\subsection*{Момент}
	\begin{enumerate}
		\item {Выразить $\hat{\textbf{l}}^2$ через $\hat{l}_z$ и $\hat{l}_\pm$.}	
		\begin{equation}
			\hat{\boldsymbol{l^{2}}}=\hat{l_{z}^{2}}+\frac{1}{2}(\hat{l_{+}}\hat{l_{-}}+\hat{l_{-}}\hat{l_{+}})=\hat{l_{z}}^{2}+\hat{l_{z}}+\hat{l_{-}}\hat{l_{+}}=\hat{l_{z}}^{2}-\hat{l_{z}}+\hat{l_{+}}\hat{l_{-}}
		\end{equation}
		\item {$\hat{\textbf{l}}^2 \ket{l, m} = \ ?$}
		\begin{equation}
			\hat{\boldsymbol{l}}^2\ket{l,m}=l(l+1)\ket{l,m}
		\end{equation}			
		\item {$\hat{l}_z \ket{l, m} = \ ?$}
		\begin{equation}
			\hat{l}_z\ket{l,m}=m\ket{l,m}
		\end{equation}	
		\item {$\hat{l}_+ \ket{l, l} = \ ?$}
		\begin{equation}
			\hat{l}_+\ket{l,l}=0
		\end{equation}	
		\item {$\hat{l}_- \ket{l, -l} = \ ?$}
		\begin{equation}
			\hat{l}_-\ket{l,-l}=0
		\end{equation}		
		\item {$\hat{l}_\pm \ket{l, m} = \ ?$}
		\begin{equation}
			\hat{l}_\pm\ket{l,m}=\sqrt{(l\mp m)(l\pm m+1)}\ket{l,m\pm1}
		\end{equation}	
	\end{enumerate}
	
\subsection*{Центральное поле}
	\begin{enumerate}
		\item {Как определяется полный набор физических величин бесспиновой частицы в центральном поле?}
		
		По теореме Кёнига энергию системы можно переписать в следующем виде:
		\begin{equation}
			\hat{H} = \frac{\hat{p}_r^2}{2m} + \frac{\hat{\textbf{M}}^2}{2I} + U(\hat{r})
		\end{equation}
		где $\textbf{M}$ - момент импульса, $I$ момент инерции. Переходя в сферические координаты, можно доказать, что оператор момента импульса действует только на угловые координаты. Из описанных выше коммутационных соотношений для оператора момента импульса следует, что операторы $\hat{H}, \hat{M}_z$ и $\hat{\textbf{M}}^2$ коммутируют между собой, и их можно включить в полный набор физических величин. Явные вычисления показывают, что этот набор набор оказывается полным.
		 
		 \item {Как разделяются переменные в волновой функции, описывающей состояние частицы в центральном поле $\psi(\textbf{r}) = \ ?$}
		 
		 Из-за того, что лапласиан в сферических координатах разбиватся на две части - радиальную и угловую, можно предположить, что возможно следующее разделение переменных:
		 \begin{equation}
		 	\psi(\textbf{r}) = R(r)Y(\theta,\varphi)
		 \end{equation}
		 Подставим эту волновую функцию в стационарное уравнение Шредингера и поделив на неё, получим:
		 \begin{equation}
		 	-\frac{\hbar^2}{2m} \frac{\Delta_r R(r)}{R(r)} - \frac{\hbar^2}{2mr^2}\frac{\Delta_{\theta,\varphi} Y(\theta, \varphi)}{Y(\theta, \varphi)} + U(r) = E
		 \end{equation}
		 Данное уравнение можно привести к виду, в котором будет сумма функции, зависящей только от $r$, и функции, зависящей только от $\theta,\varphi$, причем их сумма будет равна константе. Следовательно, данное разделение переменных имеет место.
		 
		 \item {Асимптотическое поведение радиальной функции связанного состояния $\left.R_{nl}(r)\right|_{r \to 0} \sim \ ?$}
		 
		 Угловая функция $Y(\theta, \varphi)$ выбирается как собственная для оператора квадрата момента импульса. Тогда уравнение на радиальную часть $R_{nl}$ получится следующим:
		 \begin{equation}
		 	R''_{nl} + \frac{2}{r}R_{nl}' - \frac{l(l+1)}{r^2}R_{nl} - \frac{2m}{\hbar^2}(E_n-U(r))R_{nl} = 0
		 \end{equation}
		 При устремлении $r$ к нулю преобладающими оказываются следующие слагаемые\footnote{При достаточно адекватном потенциале}:
		 \begin{equation}
		 	R_{nl}'' +\frac{2}{r} R_{nl}' - \frac{l(l+1)}{r^2}R_{nl} = 0
		 \end{equation}
		 Решениями этого уравнения являются функции вида $Cr^s$. Найдем $s$:
		 \begin{equation}
		 	s(s+1) - l(l+1) = 0 \Rightarrow s = l, -(l+1)
		 \end{equation}
		 Второе значение дает неограниченную волновую функцию, что противоречит условию нормировки состояния (интеграл будет расходиться). Следовательно, асимптотика $R_{nl}$ в нуле следующая:
		 \begin{equation}
		 	\left.R_{nl}(r)\right|_{r \to 0} \sim r^l
		 \end{equation}
		 
		 \item {Асимптотическое поведение радиальной функции связанного состояния $\left.R_{nl}(r)\right|_{r \to \infty} \sim \ ?$}
		 
		 Аналогично предыдущему вопросу, при стремлении $r$ к бесконечности преобладают следующие слагаемые:
		 \begin{equation}
		 	R_{nl}'' - \frac{2mE}{\hbar^2}R_{nl} = 0
		 \end{equation}
		 Решением этого уравнения является экспоненты вида $c e^{\pm \varkappa r}$, где $\varkappa = \sqrt{\frac{2mE}{\hbar^2}}$. Положительная степень экпоненты противоречит условию связности состояния, следовательно, асимптотика следующая:
		 \begin{equation}
		 	\left.R_{nl}(r)\right|_{r \to \infty} \sim e^{- \varkappa r}
		 \end{equation}
		 
		 \item {Записать гамильтониан атома водорода.}
		 
		 В атоме водорода потенциал кулоновский ($U(r) = -\frac{e^2}{r}$), и гамильтониан выглядит следующим образом (в координатном представлении):
		 \begin{equation}
		 	\hat{H} = -\frac{\hbar^2}{2m}\Delta -\frac{e^2}{r}
		 \end{equation}
		 
		 \item {Как определяется атомная система единиц?}
		 
		 Атомная система единиц за основу берет единицу атомной скорости $v_0 = \frac{e^2}{\hbar}$, из которой получается постоянная тонкой структуры $\frac{v_0}{c} = \frac{1}{137}$. Единица энергии задается следующим образом:
		 \begin{equation}
		 	E_0 = mv_0^2 = \frac{me^4}{\hbar} = \frac{\hbar^2}{ma_0^2}
		 \end{equation}
		 где $a_0 = \frac{\hbar}{mv_0} = \frac{\hbar^2}{me^2}$ - атомная единица расстояния.
		 
		 \item {Записать полный набор и значения, которые могут принимать квантовые числа, определяющие состояние атома водорода.}
		 
		 В первом вопросе было показано, что полным набором являются энергия, квадрат момента импульса и его проекция на выбранную ось. Для удобства в качестве квантовых чисел берут номер уровня энергии $n$, максимальную проекцию момента импульса $l$ в единицах $\hbar$ и проекцию на выбранную ось $m$ также в единицах $\hbar$. Число $n$ задается равенством
		 \begin{equation}
		 	n = \sqrt{\frac{me^4}{2|E_n|\hbar^2}}
		 \end{equation}
		 откуда легко находится спектр атома водорода:
		 \begin{equation}
		 	E_n = -\frac{me^4}{2\hbar^2 n^2}
		 \end{equation}
		 Ограничения на число $l$ задаются неравенством $n - l - 1 \geq 0$. Число $m$ как проекция спина на выбранную ось по модулю не может превосходить максимальной.
		 \item {Записать спектр атома водорода и определить кратность вырождения уровней энергии.}
		 
		 В предыдущем вопросе спектр атома водорода был получен, подсчитаем кратность вырождения:
		 \begin{equation}
		 	N = \sum\limits_{l=0}^{n-1}\sum\limits_{m=-l}^{l}1 = \sum\limits_{l=0}^{n-1} 2l+1 = n^2
		 \end{equation}
		 
		 \item {Какой вид имеет волновая функция основного состояния атома водорода $\psi_0(\textbf{r}) = \ ?$}
		 
		 Радиальная часть волновой функции основного состояния выглядит следующим образом:
		 \begin{equation}
		 	R_{10}(r) = \frac{2}{\sqrt{a_0^3}}e^{-\frac{r}{a_0}}
		 \end{equation}
		 Угловая часть основного состояния:
		 \begin{equation}
		 	Y_{00}(\theta, \varphi) = \frac{1}{\sqrt{4\pi}}
		 \end{equation}
	\end{enumerate}

{\center{\section*{Квантовая механика, часть 2}}}

\subsection*{Квазиклассика}
	\begin{enumerate}
		\item {Представление волновой функции частицы через квантовое действие.}

\begin{equation}
\psi(\boldsymbol{r},t)=Ae^{\frac{i}{\hbar}\hat{S_{q}}}
\end{equation}


		\item {Уравнение для квантового действия (консервативной системы).}
\begin{equation}
-\frac{i\hbar}{2m}\triangle s+\frac{1}{2m}(\nabla s)^{2}+U(q)-E=0
\end{equation}


\begin{equation}
s=S_{q}+Et
\end{equation}


		\item {Разложение квантового действия по степеням $\hbar$.}
\begin{equation}
S_{q}=\sum_{k=0}^{+\infty}(\frac{\hbar}{i})^{k}s_{k}
\end{equation}


		\item {Уравнение для квантового действия в одномерном случае.} 
\begin{equation}
-\frac{i\hbar}{2m}s^{\prime\prime}+\frac{1}{2m}(s^{\prime})^{2}+U(q)-E=0
\end{equation}


		\item {Уравнение для определения нулевого приближения квантового действия. Выражение для квазиклассического импульса.}  
\begin{equation}
\frac{1}{2m}(s_{0}^{\prime})^{2}=E\text{-}U(q)
\end{equation}


\begin{equation}
s_{0}=\pm\int\sqrt{2m(E-U(q))}dq=\pm\int p(q)dq
\end{equation}


\begin{equation}
2s{}_{0}^{\prime}s{}_{1}^{\prime}+s{}_{0}^{\prime\prime}=0
\end{equation}


\begin{equation}
s_{1}=-\frac{1}{2}\ln p(q)+C
\end{equation}


		\item {Критерий применимости квазиклассического приближения.}  
\begin{equation}
|\frac{\hbar s^{\prime\prime}}{(s^{\prime})^{2}}|=|\frac{d}{dq}\frac{\hbar}{s^{\prime}}|=|\frac{d}{dq}\frac{\hbar}{p}|=|\frac{d}{dq}\lambda|\ll1
\end{equation}


\begin{equation}
=|\frac{\hbar}{p^{2}}\frac{dp}{dq}|=|\frac{\hbar m}{p^{3}}(-\frac{dU}{dq})|\ll1
\end{equation}


		\item {Общий вид волновой функции квазиклассического приближения в классически разрешенной области.} 
\begin{equation}
\psi(q)=\frac{C_{1}}{\sqrt{p(q)}}\exp\{\frac{i}{\hbar}\int p(q)dq\}(1+O(\frac{d\lambda}{dq}))+\frac{C_{2}}{\sqrt{p(q)}}\exp\{-\frac{i}{\hbar}\int p(q)dq\}(1+O(\frac{d\lambda}{dq}))
\end{equation}


		\item {Возможные постановки задачи в квазиклассическом приближении.} 
		
		Поскольку сшивка волновых функций в классически разрешнной и в
классически запрещнной зонах существенным образом зависит от граничных
условий то рассматриваются разные задачи а именно:

1. нахождение уровней энергии в потенциальной яме

2. нахождение коэффициента проникновения частицы через потенциаль
ный барьер.

		\item {Точки поворота (определение).}  
\begin{equation}
p(q_{0})=0
\end{equation}
или что то же самое:
\begin{equation}
E=U(q_{0})
\end{equation}


		\item {Правило квантования Бора-Зоммерфельда для определения энергетического спектра связанных состояний.}  
\begin{equation}
\intop_{a}^{b}p(q)dq=\pi\hbar(n+\frac{1}{2})
\end{equation}


или: 
\begin{equation}
\oint p(q)dq=2\pi\hbar(n+\frac{1}{2})
\end{equation}


		\item {Определение квазиклассического периода.} 

\begin{equation}
T=\oint\frac{dq}{v}
\end{equation}


		\item {Нормировка волновой функции связанного состояния в квазиклассическом приближении.} 
\begin{equation}
\begin{split}
1=\intop_{-\infty}^{+\infty}|\psi(q)|^{2}dq\approx\intop_{a}^{b}|\psi(q)|^{2}dq=4|C|^{2}\intop_{a}^{b}\frac{\cos^{2}\{\frac{\intop_{q}^{b}p(q)dq}{\hbar}-\frac{\pi}{4}\}}{p(q)}dq \\
\approx2|C|^{2}\intop_{a}^{b}\frac{dq}{p(q)}=\frac{|C|^{2}T}{m}=\frac{2\pi|C|^{2}}{m\omega}
\end{split}
\end{equation}


\begin{equation}
C=\sqrt{\frac{m\omega}{2\pi}}
\end{equation}


\begin{equation}
\psi(q)=\sqrt{\frac{2m\omega}{\pi p}}\cos\{\frac{\intop_{q}^{b}p(q)dq}{\hbar}-\frac{\pi}{4}\}
\end{equation}


		\item {Плотность энергетического спектра связанных состояний в квазиклассическом приближении.} 

\begin{equation}
\begin{split}
\frac{\triangle n}{\triangle E}\approx\frac{dn}{dE}=\frac{d}{dE}\frac{1}{2\pi\hbar}\oint p(q)dq=\frac{d}{dE}\frac{1}{2\pi\hbar}\oint\sqrt{2m(E-U(q))}dq \\
=\frac{1}{2\pi\hbar}\oint\frac{m}{p(q)}dq=\frac{1}{2\pi\hbar}\oint\frac{dq}{v}=\frac{T}{2\pi\hbar}=\frac{1}{\hbar\omega}
\end{split}
\end{equation}


		\item {Постановка задачи о проникновении частицы через потенциальный барьер.} 

\begin{equation}
\text{ясно-понятно}
\end{equation}


		\item {Вероятность проникновения через потенциальный барьер.}
\begin{equation}
D\text{\ensuremath{\approx}}\exp\{-\frac{2}{\hbar}\int_{a}^{b}|p(q)|dq\}
\end{equation}
	\end{enumerate}
	
\subsection*{Стационарная теория возмущений}
	\begin{enumerate}
		\item {Формулировка задачи стационарной теории возмущений.}
		
		Теория возмущений применяется для приближенного решения задач, которые не получается посчитать явно. Суть состоит в следующем: пусть возможно следующее разделение гамильтониана:
		\begin{equation}
			\hat{H} = \hat{H}_0 + \hat{V}
		\end{equation}
		и для оператора $\hat{H}_0$ задача решается. Тогда оператор $\hat{V}$ называют {возмущением}. Далее решение стационарного уравнения ищется в следующем виде:
		\begin{equation}
			\ket{\psi} = \sum\limits_{k} c_k^{(n)} \ket{\psi_k^{(0)}} \label{disturbanceVectors}
		\end{equation}
		где $\ket{\psi_k^{(0)}}$ - собственные векторы оператора $\hat{H_0}$. Подставим волновую функцию в стационарное уравнение:
		\begin{equation}
			\sum\limits_{k}c_k^{(n)}\big(E_k^{(0)} \ket{\psi_k^{(0)}} + \hat{V}\ket{\psi_k^{(0)}}\big) = E\sum\limits_{k}c_k^{(n)} \ket{\psi_k^{(0)}}
		\end{equation}
		Умножим скалярно на $\bra{\psi_m^{(0)}}$:
		\begin{equation}
			\sum\limits_{k}c_k^{(n)}\big(E_k^{(0)} \delta_{km} + V_{km}\big) = E\sum\limits_{k}c_k^{(n)}\delta_{km}
		\end{equation}
		где $V_{nm}$ - матричный элемент оператора $\hat{V}$. Перенесем все слагаемые в левую часть:
		\begin{equation}
			\sum\limits_{k}c_k^{(n)}\big(V_{km} - (E - E_k^{(0)} ) \delta_{km}\big) = 0
		\end{equation}
		Из этого уравнения находятся поправки к уровням энергии.
		
		\item {Критерий применимости теории возмущений.}
		
		Ряд в формуле~\eqref{disturbanceVectors} может не сходиться. Обычно поправку к имеющимся состояниям ищут в виде:
		\begin{equation}
			\ket{\psi_n} = \ket{\psi_n^{(0)}} + \ket{\delta \psi_n}
		\end{equation}
		где $\braket{\psi_n^{(0)}|\delta \psi_n} = 0$. Для того, чтобы результаты приближенного вычисления слабо отличались от точного решения, необходимо, чтобы поправка была мала:
		\begin{equation}
			\|\ket{\delta \psi_n}\| \ll 1
		\end{equation}
		
		\item {Поправка первого порядка к уровням энергии для невырожденного спектра.}
		
		\begin{equation}
			E_{n}^{(1)} = V_{nn}
		\end{equation}
		\item {Поправки к состояниям невырожденного спектра в первом порядке теории возмущений.}
		
		\begin{equation}
			\ket{\delta\psi_n^{(1)}} = \sum\limits_{m\neq n} \frac{V_{mn}}{E_n^{(0)} - E_m^{(0)}} \ket{\psi_m^{(0)}}
		\end{equation}
		
		\item {Поправка второго порядка к уровням энергии для невырожденного спектра.}
		
		\begin{equation}
			E_n^{(2)} = \sum\limits_{m\neq n} \frac{|V_{mn}|^2}{E_n^{(0)} - E_m^{(0)}}
		\end{equation}	
		
		\item {Как изменится энергия основного невырожденного уровня системы, при помещении ее во внешнее поле?}	
		
		\item {Постановка стационарной теории возмущений в случае вырожденного энергетического спектра.}
		При вырожденном спектре гамильтониана задача ставится несколько иначе:
		\begin{equation}
			\hat{H}_0 \ket{\psi^{(0)}_{n,\alpha}} = E_n^{(0)}\ket{\psi^{(0)}_{n,\alpha}}
		\end{equation}
		где $\alpha$ - параметр вырождения. Для такой ситуации выбор нулевых собственных векторов неоднозначен, так как любая комбинация вида
		\begin{equation}
			\ket{\psi^{(0)}_{n,\alpha}} = \sum\limits_{\beta} C_\beta \ket{\psi^{(0)}_{n,\beta}} 
		\end{equation}
		удовлетворяет уравнению Шредингера. Трюк состоит в том, чтобы найти такую линейную комбинацию, которая удовлетворяла бы критерию применимости стационарной теории возмущений. Найдем их, подставляя в уравнение Шредингера:
		\begin{equation}
			\sum\limits_{\beta} C_\beta \Big( \hat{V} - (E - E_n^{(0)}) \Big) \ket{\psi_{n,\beta}^{(0)}} = 0
		\end{equation}
		Домножив скалярно слева на $\bra{\psi_{n,\beta}^{(0)}}$, получаем:
		\begin{equation}
			\sum\limits_{\beta} C_\beta \Big( V_{\beta\alpha} - (E - E_n^{(0)})\delta_{\beta\alpha} \Big) = 0
		\end{equation}
		Это уравнение имеет нетривиальное решение тогда и только тогда, когда матрица $V_{\beta\alpha} - \varepsilon \delta_{\beta\alpha}$ вырождена. Уравнение на параметр $\varepsilon$, при котором матрица вырождена, называется {секулярное уравнение}:
		\begin{equation}
			\det \|V_{\beta\alpha} - \varepsilon \delta_{\beta\alpha}\| = 0
		\end{equation}
		
		\item {Правильные функции нулевого приближения и секулярное уравнение в теории возмущений.}
		
		Секулярное уравнение было определено выше. Правильные волновые функции это собственные векторы матрицы $V_{\beta\alpha} - \varepsilon \delta_{\beta\alpha}$.
		
		\item {Уравнение для определения функции Грина стационарного уравнения Шредингера.}
		
		\begin{equation}
			(\hat{H} - E) G_E(\textbf{r}, \textbf{r}') = \delta(\textbf{r} - \textbf{r}')
		\end{equation}
		\item {Определение функции Грина стационарного уравнения Шредингера как представление оператора.}
		
		Функцию Грина можно считать матрицей некоторого оператора:
		\begin{equation}
			G_E(\textbf{r}, \textbf{r}') = \braket{\textbf{r}|\hat{G}_E|\textbf{r}'}
		\end{equation}
		Найдем явный вид этого оператора:
		\begin{equation}
			\braket{\textbf{r}|(\hat{H} - E)\hat{G}_E | \textbf{r}'} = \delta (\textbf{r} - \textbf{r}') = \braket{\textbf{r}|\hat{1}| \textbf{r}'}
		\end{equation}
		Следовательно, оператор $\hat{G}_E$ равен:
		\begin{equation}
			\hat{G}_E = (\hat{H} - E)^{-1} = \frac{1}{\hat{H} - E}
		\end{equation}
		
		\item {Ряд теории возмущений для функции Грина стационарного уравнения Шредингера.}
		
		Переходя к теории возмущений, зная оператор Грина невозмущенной задачи $\hat{G}^{(0)}$, можно переписать выражение для $\hat{G}$:
		\begin{equation}
			\hat{G} = (\hat{H}_0 - E + \hat{V})^{-1} = \hat{G}_0 (\hat{1} - \hat{V}\hat{G})
		\end{equation}
		
		
		\item {Что определяют полюса функции Грина стационарного уравнения Шредингера?}	
		
		\begin{equation}
			\bra{n}\hat{G}_{E}\ket{n'}=\bra{n}(\hat{H}-E)^{-1}\ket{n'}=\delta_{n,n'}\frac{1}{E_{n}-E}
		\end{equation}
		Это функция Грина в энергетическом представлении. Отсюда видно,
что её полюсы определяют спектр гамильтониана.

		\item {Функция Грина в энергетическом представлении.}
		
		Смотри предыдущий вопрос.
		
		\item {Выражение для поправок к состояниям в теории возмущений с помощью функции Грина.}
		
		\begin{equation}
			\ket{\delta\psi^{(n)}}=\big(-\hat{G^{(0)}}\hat{V}\big)^{n}\ket{\psi^{(0)}}
		\end{equation}
		
		\item {Выражение для поправок к уровням энергии в теории возмущений с помощью функции Грина.}
		
		\begin{gather}
			\delta E_{n}^{(1)}=\bra{n}\hat{V}\ket{n} \\
\delta E_{n}^{(2)}=\bra{n}\hat{V}(-\hat{G^{(0)}})\hat{V}\ket{n}
\\
\delta E_{n}^{(3)}=\bra{n}\hat{V}(-\hat{G^{(0)}})\hat{V}(-\hat{G^{(0)}})\hat{V}\ket{n}
		\end{gather}
		
		\item {Функция Грина стационарного уравнения Шредингера в координатном представлении для дискретного спектра.}
		
		\begin{equation}
			\bra{\boldsymbol{r}}\hat{G}\ket{\boldsymbol{r}'}=\sum_{n}\frac{\psi_{\text{n}}^{*}(\boldsymbol{r}')\psi_{n}(\boldsymbol{r})}{E_{n}-E}
		\end{equation}
		
		\item {Функция Грина стационарного уравнения Шредингера в случае непрерывного спектра.}		
		
		\begin{equation}
			\bra{\boldsymbol{r}}\hat{G}\ket{\boldsymbol{r'}}=\int\frac{\psi_{\nu}^{*}(\boldsymbol{r}')\psi_{\nu}(\boldsymbol{r})}{E_{\nu}-E\pm i\gamma}d\nu
		\end{equation}
		
		\item {Функция Грина свободной частицы.}	
		
		\begin{equation}
			\hat{G}^{(\pm)}(\boldsymbol{r,r')}=\frac{m}{2\pi\hbar^{2}|r-r'|}\exp\{\pm i|r-r'|\sqrt{\frac{2mE}{\hbar^{2}}}\}
		\end{equation}
		
		$+$ отвечает расходящейся волне, $-$ отвечает сходящейся волне
		
		\item {Основное интегральное уравнение для определения волновой функции в случае непрерывного спектра.}
		
		\begin{equation}
			\psi(\boldsymbol{r})=A\exp\{i\boldsymbol{kr}\}-\frac{m}{2\pi\hbar^{2}}\int\frac{\exp\{ik|r-r'|\}}{|\boldsymbol{r-r'}|}\hat{V}(\boldsymbol{r'})\psi(\boldsymbol{r')}d\boldsymbol{r'}
		\end{equation}
		
		\item {Асимптотический вид волновой функции в задаче о рассеянии.}
		
		\begin{equation}
			\psi(\boldsymbol{r)}\approx A\exp\{i\boldsymbol{kr}\}+f(\theta,\phi)\frac{e^{ikr}}{r}
		\end{equation}
		
		\item {Определение дифференциального сечения рассеяния и его связь с амплитудой рассеяния.}
		
		Дифференциальное сечение рассеяния - вероятность пройти рассеянной
частице в единицу времени через элемент телесного угла.
		\begin{equation}
			\frac{d\sigma}{d\Omega}=|f(\theta,\phi)^{2}|
		\end{equation}
		
		\item {Борновское приближение в теории рассеяния.}

		Борновское приближение - первый порядок теории возмущения, применнный
к задаче рассеяния на неком потенциале. Критерий применимости -  тот же
что и в теории возмущения

		\begin{equation}
			|\ket{\delta\psi}|=\big|\frac{m}{2\pi\hbar^{2}}\int\frac{\exp\{ik|\boldsymbol{r-r'}|\}}{|\boldsymbol{r}-\boldsymbol{r}'|}\hat{V}(\boldsymbol{r')}\exp\{i\boldsymbol{kr'}\}dr'\big|\ll1
		\end{equation}
		
		\item {Критерий применимости борновского приближения для рассеяния медленных частиц.}
		
		Медленные частицы:
		
		\begin{equation}
			\lambda\gg a,\ \exp\{i\boldsymbol{kr'}\}\approx1,\ \exp\{ik|\boldsymbol{r-r'}|\}\approx1
		\end{equation}
		
		Критерий:
		\begin{equation}
			V_{0}\ll\frac{\hbar^{2}}{ma^{2}}
		\end{equation}
		
		\item {Критерий применимости борновского приближения для рассеяния быстрых частиц.}
		
		Быстрые частицы:
		
		\begin{equation}
			\lambda\ll a
		\end{equation}
		
		Критерий:
		\begin{equation}
			V_{0}\ll\frac{\hbar^{2}}{ma^{2}}ka
		\end{equation}
		
		\item {Как зависит от углов дифференциальное сечение рассеяния медленных частиц в борновском приближении?}
		
		\begin{equation}
			f(\theta)=-\frac{2m}{\hbar^{2}}\int_{0}^{+\infty}V(r)\frac{\sin(qr)}{q}rdr,\boldsymbol{q=k'-k},q=2k\sin\frac{\theta}{2}
		\end{equation}
		
		Для медленных частиц:
		\begin{equation}
			f(\theta)\approx const
		\end{equation}
		
		\item { Особенности угловой зависимости дифференциального сечения рассеяния быстрых частиц в борновском приближении.}
		
		Для быстрых частиц в интеграле нужно учитывать только
		\begin{equation}
			qa\leq1, \ 2ka\sin\frac{\theta}{2},\ \theta\leq\frac{1}{ka}
		\end{equation}
		
	\end{enumerate}
	

\subsection*{Нестационарная теория возмущений}
\begin{enumerate}
	\item {Как зависит от времени вектор состояния квантовой системы в представлении Гайзенберга?}
	
	В представлении Шредингера базисные векторы неизменны, но изменяется
состояние. В представлении Гайзенберга все наоборот - базисные состояния
меняются, а состояние остатся неизменным.
	
	\item {Как зависит от времени оператор в представлении Гайзенберга?}
	
	Пусть $\hat{A}_S$ - оператор в представлении Шрёдингера, тогда этот же оператор в представлении Гайзенберга будет иметь вид:
	\begin{equation}
		\hat{A}_{H}=\hat{U}^{+}\hat{A}_{S}\hat{U}
	\end{equation}
	
	\item {Как определяется вектор состояния квантовой системы в представлении взаимодействия?}
	
	\begin{equation}
		\ket{\psi_{I}(t)}=\hat{U_{0}^{+}}(t)\ket{\psi(t)}
	\end{equation}
	
	\item {Уравнение Шредингера в представлении взаимодействия.}
	\begin{equation}
		i\hbar\frac{\partial}{\partial t}\psi_{I}(t)=\hat{V}_{I}\psi_{I}(t)
	\end{equation}
	
	\item {Итерационный ряд для определения состояния в нестационарной теории возмущений, T - exp.}
	
	\begin{equation}
		\psi_{I}(t)=\hat{T}exp\big\{-\frac{i}{\hbar}\int_{t_{0}}^{t}\hat{V_{I}}(t')dt'\big\}\psi_{I}(t_{0})=\big[\sum_{k=0}\frac{1}{k!}\big(-\frac{i}{\hbar}\big)^{k}\int\dots \int_{[t_{0,}t]^{k}}\prod_{i=1}^{k}dt^{(i)}\prod_{i=1}^{k}\hat{V_{I}}(t^{(i)})\big]\psi_{I}(t_{0})
	\end{equation}
	
	\item {Общее выражение вероятности перехода в нестационарной теории возмущений.}
	
	Вероятность перейти из начального состояния $i$ в состояние $f$ равна:
	\begin{equation}
		W_{if}=\big|\bra{f}\hat{T}exp\big\{-\frac{i}{\hbar}\int_{t_{0}}^{t}\hat{V_{I}}(t')dt'\big\}\ket{i}\big|^{2}
	\end{equation}
	
	\item {Выражение вероятности перехода в первом порядке нестационарной теории возмущений.}
	
	\begin{equation}
		W_{if}^{(1)}=\frac{1}{\hbar^{2}}\big|\int_{t_{0}}^{t}\bra{f}\hat{V_{I}}(t')\ket{i}dt'\big|^{2}=\frac{1}{\hbar^{2}}\big|\int_{t_{0}}^{t}\exp\{i\omega_{fi}t'\}\hat{V_{fi}}(t')dt'\big|^{2}
	\end{equation}
	
	\item {Критерий применимости нестационарной теории возмущений.}
	
	\begin{equation}
		|W_{if}|\ll1
	\end{equation}
	
	\item {Соотношение неопределенностей для энергии и времени.}
	
	\begin{gather}
		W_{if}\approx\frac{1}{\hbar^{2}}|V_{fi}|^{2}(t-t_{0})^{2}\ll1 \\
		\triangle t\triangle E\sim\hbar
	\end{gather}
	
	\item {Определение вероятности перехода в непрерывном спектре.}
	\begin{equation}
		dW_{if}=\big|\bra{f}\hat{T}exp\big\{-\frac{i}{\hbar}\int_{t_{0}}^{t}\hat{V_{I}}(t')dt'\big\}\ket{i}\big|^{2}d\nu_{f},
	\end{equation}
	Конечное состояние содержится в промежутке $d\nu_f$, с энергией в промежутке $[E_f, E_f + dE_f$.
	
	\item {Вероятность перехода в единицу времени, “золотое правило” Ферми.}
	\begin{gather}
		\omega_{if}=\frac{dW_{if}}{dt}=\frac{2\pi}{\hbar}\big|F_{fi}\big|^{2}\rho(E_{f})\big|_{E_{f}=E_{i}+\hbar\omega} \\
		\rho(E_{f})=\frac{d\nu(E_{f)}}{dE_{f}} \\
		\hat{V}=\hat{F}e^{-i\omega t}+\hat{F}^{+}e^{i\omega t}
	\end{gather}
	
	\item {Понятие квазистационарного состояния, ширина уровня.}
	
	\begin{equation}
		e^{-wt} = |e^{-\frac{i}{\hbar}E_n t}|^2 \rightarrow E_n = E_n^{(0)} - i\frac{\Gamma_n}{2}, \ \Gamma_n=\hbar w
	\end{equation}
	Величина $\Gamma_n$ - ширина уровня энергии.
	
	\item {Поправка второго порядка к уровню энергии в непрерывном спектре и ее связь с шириной
уровня квазистационарного состояния.}

	\begin{gather}
		E_n^{(2)} = \sum\limits_{m\neq n} \frac{|V_{mn}|^2}{E_n^{(0)} - E_m^{(0)}} + \int \frac{|V_{\nu n}|^2}{E_n^{(0)} - E_\nu} d\nu_E \\
		\frac{\Gamma}{2} = \left.\pi |V_{fn}|^2 \rho(E_f)\right|_{E_f=E_n}
	\end{gather}
	
	\item {Связь ширины уровня квазистационарного состояния и вероятности перехода в единицу времени в непрерывном спектре.}
	
	\begin{equation}
		w_{fi} = \frac{\Gamma}{2\hbar}
	\end{equation}
\end{enumerate}
	
{\center{\section*{Квантовая механика, часть 3}}}
\subsection*{Многочастичная теория}
\begin{enumerate}
	\item {Как можно записать вектор состояния системы, состоящей из двух невзаимодействующих подсистем, находящихся в состояниях $\ket{\psi_1}$ и $\ket{\psi_2}$?}	
	
	Вектор состояния ситсемы определяется как тензорное произведение состояний подсистем:
	\begin{equation}
		\ket{\psi} = \ket{\psi_1} \otimes \ket{\psi_2} = \ket{\psi_1} \ket{\psi_2}
	\end{equation}
	Знак тензорного произведения обычно опускается.
	
	\item {Замкнутая система состоит из двух невзаимодействующих подсистем, обладающих моментами $l_1$ и $l_2$ соответственно. Сколько линейно независимых векторов определяют состояния с определенным суммарным моментом?}	
	
	Если $l_2 \leq l_1$, то число векторов с определенным сумарным моментом равно $2l_2 + 1$. Доказательство этого факта приводится ниже при подсчете возможных значений суммарного момента.
	
	\item {Замкнутая система состоит из двух невзаимодействующих подсистем, обладающих моментами $l_1$ и $l_2$ соответственно. Какими квантовыми числами определяется состояние системы с заданным суммарным моментом $L$? Записать дираковский вектор состояния.}	
	
	Момент каждой системы по отдельности сохраняется, значит $l_1, l_2$ являются квантовыми числами. Для заданного суммарного момента $L$ можно одновременно измерить его проекцию на выбранную ось. Таким образом, дираковский вектор состояния определяется следующим образом:
	
	\begin{equation}
		\ket{\psi} = \ket{l_1, l_2, L, M}
	\end{equation}
	
	\item {Какие значения может принимать суммарный момент $L$ системы, состоящей из двух подсистем, обладающих моментами $l_1$ и $l_2$ соответственно ?}
	
	Суммарный момент $L$ принимает все значения, которые находятся между моментом, когда моменты систем сонаправлены, и моментом, когда моменты систем противоположно направленны друг к другу. Так как момент квантуется с шагом $1$, то возможные значения момента $L$ равны:
	\begin{equation}
		|l_1 - l_2| \leq L \leq l_1 + l_2
	\end{equation}
	
	\item {Пусть $\textbf{L} = \textbf{l}_1 + \textbf{l}_2$. Показать, что коммутатор $[\hat{L}_z,\hat{\textbf{L}}^2] = 0$.}
	
	\begin{equation}
	\begin{split}
		[\hat{L}_z,\hat{\textbf{L}}^2] = [\hat{l}_{1z} + \hat{l}_{2z}, (\hat{l}_{1\alpha} + \hat{l}_{2\alpha})(\hat{l}_{1\alpha} + \hat{l}_{2\alpha})] = [\hat{l}_{1z} + \hat{l}_{2z}, \hat{\textbf{l}}_{1}^2 + \hat{\textbf{l}}_{2}^2 + 2\hat{l}_{1\alpha}\hat{l}_{2\alpha}] = \\
		= 2[ \hat{l}_{1z}, \hat{l}_{1\alpha}\hat{l}_{2\alpha}] + 2[ \hat{l}_{2z}, \hat{l}_{1\alpha}\hat{l}_{2\alpha}] = 2[ \hat{l}_{1z}, \hat{l}_{1\alpha}]\hat{l}_{2\alpha} + 2\hat{l}_{1\alpha}[ \hat{l}_{2z}, \hat{l}_{2\alpha}] = 2 e_{z\alpha\beta} \big( \hat{l}_{1\beta} \hat{l}_{2\alpha} + \hat{l}_{1\alpha} \hat{l}_{2\beta}\big ) = 0
	\end{split}
	\end{equation}
	
	\item {Как с помощью коэффициентов Клебша-Гордана определить состояние системы с полным моментом $J$, если известны состояния подсистем $\ket{j_1, m_1}$ и $\ket{j_2, m_2}$?}
	
	\begin{equation}
		\ket{j_1, j_2, J, M} = \sum\limits_{m_1 + m_2 = M} C^{J, M}_{j_1,m_1;,j_2,m_2} \ket{j_1, m_1}\ket{j_2, m_2}
	\end{equation}
	По сути, коэффициенты Клебша-Гордана это матрица перехода из базиса состояний с определенными проекциями подсистем к базису состояний с определенным суммарным моментом и его проекциейю
	
	\item {Чему равно выражение $\sum\limits_{m1,m2}\braket{j_1 j_2; J M|j_1 m_1} \ket{j_2 m_2} \bra{j_2 m_2} \braket{j_1 m_1|j_1 j_2; J' M'}$?}
	
	\begin{equation}
		\sum\limits_{m1,m2}\braket{j_1 j_2; J M|j_1 m_1} \ket{j_2 m_2} \bra{j_2 m_2} \braket{j_1 m_1|j_1 j_2; J' M'} = \delta_{JJ'}\delta_{MM'}
	\end{equation}
	
	\item {Чему равно выражение $\sum\limits_{J,M} \bra{j_2 m_2} \braket{j_1 m_1|j_1 j_2; J M } \braket{j_1 j_2; J M |j_1 m_1'} \ket{j_2 m_2'}$?}
	
	\begin{equation}
		\sum\limits_{J,M} \bra{j_2 m_2} \braket{j_1 m_1|j_1 j_2; J M } \braket{j_1 j_2; J M |j_1 m_1'} \ket{j_2 m_2'} = \delta_{m_1 m_1'} \delta_{m_2 m_2'}
	\end{equation}
	\item {В каком виде можно представить состояние $N$ связанных гармонических осцилляторов?}
	
	Линейными преобразованиями операторов координат любую систему связных гармонических осцилляторов можно свести к системе несвязных осцилляторов:
	\begin{equation}
	\frac{1}{2} \sum\limits_{k} \hat{p}^2_k + \frac{1}{2} \sum\limits_{k,l} U_{kl}\hat{q}_k\hat{q}_l \to \frac{1}{2} \sum\limits_{\alpha} \hat{p}^2_\alpha + \omega^2_\alpha\hat{q}^2_\alpha
	\end{equation}
	Таким образом, состояние системы осцилляторов можно представить как прямое произведение состояний осцилляторов, которые характеризуются одном квантовым числом:
	\begin{equation}
		\ket{\psi} = \bigotimes\limits_{\alpha = 1}^N \ket{n_\alpha}
	\end{equation}
	
	\item {В каком виде можно представить гамильтониан системы связанных гармонических осцилляторов?}
	
	Смотри предыдущий вопрос.
	
	\item {Как определяется спектр системы связанных гармонических осцилляторов?}
	
	Исходя из вида гамильтониана системы связных осцилляторов
	\begin{equation}
		\hat{H} = \sum\limits_{\alpha} \hbar \omega_\alpha \Big( \hat{a}^+_\alpha \hat{a}_\alpha + \frac{1}{2} \Big)
	\end{equation}
	получаем спектр системы:
	\begin{equation}
		E = \sum\limits_{\alpha} \hbar \omega_\alpha \Big( n_\alpha + \frac{1}{2} \Big)
	\end{equation}
	
	\item {Как с помощью повышающих операторов можно получить произвольное состояние системы гармонических осцилляторов из основного?}

	Аналогично одномерному осциллятору:
	\begin{equation}
		\ket{\psi} = \prod\limits_{\alpha} \frac{(\hat{a}^+_\alpha)^{n_\alpha}}{\sqrt{n_\alpha!}} \ket{00...0}
	\end{equation}		
	
	\item {Пусть $\Psi(x_1, x_2;t)$ волновая функция системы двух тождественных частиц, чему равен результат действия на нее оператора перестановки частиц $\hat{P}\Psi(x_1, x_2;t) = \ ?$}
	
	Волновая функция системы тождественных частиц всегда является собственной функцией оператора перестановки, следовательно
	\begin{equation}
		\hat{P}\Psi(x_1, x_2;t) = \Psi(x_2, x_1;t) = \pm \Psi(x_1, x_2;t)
	\end{equation}
	Знак в равенстве определяется типом системы (бозе или ферми).
	
	\item {В каком виде можно записать гамильтониан системы двух взаимодействующих тождественных частиц?}
	
	\begin{equation}
		\hat{H}(1,2) = \frac{\hat{\textbf{p}}_1^2}{2m} + \frac{\hat{\textbf{p}}_2^2}{2m} + U(\hat{\textbf{r}}_1) + U(\hat{\textbf{r}}_2) + V(\hat{\textbf{r}}_1, \hat{\textbf{r}}_2)
	\end{equation}
	Оператор $\hat{V}$ симметричен.
	
	\item {Как для системы двух ферми-частиц связаны между собой волновые функции $\Psi(x_1, x_2;t)$ и $\Psi(x_2, x_1;t)$?}
	\begin{equation}
		\Psi(x_1, x_2;t) = -\Psi(x_2, x_1;t)
	\end{equation}
	
	\item {Как для системы двух бозе-частиц связаны между собой волновые функции $\Psi(x_1, x_2;t)$ и $\Psi(x_2, x_1;t)$?}
	
	\begin{equation}
		\Psi(x_1, x_2;t) = \Psi(x_2, x_1;t)
	\end{equation}
	
	\item {В каком виде можно представить вектор состояния системы N невзаимодействующих тождественных частиц? Как определяется полный набор квантовых чисел?}
	
	Аналогично двум частицам	
	\begin{equation}
		\ket{\psi} = \bigotimes\limits_{n=1}^N \ket{\psi_n}
	\end{equation}
	Полный набор квантовых чисел равен объединению наборов квантовых чисел каждой частицы.
	
	\item {Как можно записать волновую функцию $N$ невзаимодействующих ферми-частиц?}
	\begin{equation}
		\braket{\textbf{r}_1...\textbf{r}_N|\psi_1 ... \psi_N} = \frac{1}{\sqrt{N!}}\det \|\psi_{n_i}(\textbf{r}_k) \ket{m_k} \|
	\end{equation}
	где $\psi_{n_i}(\textbf{r})$ - волновые функции частиц, $\ket{m_i}$ - спиновые вектора.
	
	\item {Как можно записать волновую функцию $N$ невзаимодействующих бозе-частиц?}
	
	\begin{equation}
		\braket{\textbf{r}_1...\textbf{r}_N|\psi_1 ... \psi_N} = \frac{1}{\sqrt{N!}}\text{perm} \|\psi_{n_i}(\textbf{r}_k) \ket{m_k} \|
	\end{equation}
	Перманент вычисляется аналогично определителю, только в сумме отсутствуют $-1$.
	
	\item {Для системы двух невзаимодействующих ферми-частиц записать волновую функцию с определенным суммарным спином $S$.}
	
	Состояние должно быть антисимметричным, поэтому
	\begin{equation}
		\Psi(\textbf{r}_1, \textbf{r}_2, S, M_S) = \frac{1}{\sqrt{2}}\big( \psi_1(\textbf{r}_1) \psi_2(\textbf{r}_2) + (-1)^S \psi_2(\textbf{r}_1) \psi_1(\textbf{r}_2) \big) \ket{S, M_S}
	\end{equation}
	
	Далее будет показано, векторы с определенным спином $S$ имеют определенную четность
	
	\item {Для системы двух невзаимодействующих бозе-частиц записать волновую функцию с определенным суммарным спином $S$.}
	
	Волновая функция не отличается от волновой функции для ферми-частиц.
	
	\item {Записать спиновые векторы состояния системы двух электронов, обладающих суммарным спином $S = 1$.}
	
	\begin{gather}
		\ket{1,1} = \ket{+}\ket{+} \\
		\ket{1,0} = \frac{1}{\sqrt{2}}( \ket{-}\ket{+} + \ket{+}\ket{-}) \\
		\ket{1,-1} = \ket{-}\ket{-}
	\end{gather}
	
	\item {Записать спиновый вектор состояния системы двух электронов, обладающих суммарным спином $S = 0$.}
	
	\begin{equation}
		\ket{0,0} = \frac{1}{\sqrt{2}}( \ket{-}\ket{+} - \ket{+}\ket{-}) 
	\end{equation}
	
	\item {Система двух электронов находится в спиновом состоянии, которое описывается вектором $\ket{+}\ket{+}$. Построить вектор состояния $\ket{S = 1, M_S = 0}$, описывающий cостояние с суммарнымспином $S = 1$.}
	
	Действуя оператором $\hat{S}_- = \hat{s}_{1-} + \hat{s}_{2-}$, получаем один из векторов из предыдущих вопросов.
		
	\item {Система двух электронов находится в спиновом состоянии, которое описывается вектором $\ket{+} \ket{-}$. Представить его в виде суперпозиции состояний с определенным суммарным спином $S$.}
	
	\begin{equation}
		\ket{+} \ket{-} = \frac{1}{\sqrt{2}} \big( \ket{1,0} - \ket{0,0} \big)
	\end{equation}
	
	\item {Координатная часть волновой функции двух электронов имеет вид
$\frac{1}{\sqrt{2}}\big( \psi_1(\textbf{r}_1) \psi_2(\textbf{r}_2) - \psi_2(\textbf{r}_1) \psi_1(\textbf{r}_2) \big)$ Чему равен суммарный спин $S$?}

	$S = 1$, что следует из вопроса №20.
	
	\item {Координатная часть волновой функции двух электронов имеет вид
$\frac{1}{\sqrt{2}}\big( \psi_1(\textbf{r}_1) \psi_2(\textbf{r}_2) + \psi_2(\textbf{r}_1) \psi_1(\textbf{r}_2) \big)$ Чему равен суммарный спин $S$?}

	$S=0$.	
	
	\item {Как определяется обменный интеграл для двух слабо взаимодействующих между собой электронов?}
	
	\begin{equation}
		J = \int \psi_{n_1}^*(\textbf{r}_1) \psi_{n_2}(\textbf{r}_1) \psi_{n_2}^*(\textbf{r}_2) \psi_{n_1}(\textbf{r}_2) V(\textbf{r}_1 - \textbf{r}_2) d\textbf{r}_1 d\textbf{r}_2
	\end{equation}
	
	\item {Записать результат действия оператора рождения на вектор $N$-частичного состояния тождественных частиц $a^+(\varphi)\ket{\psi_1, . . . , \psi_N} = \ ?$}
	
	\begin{equation}
		a^+(\varphi)\ket{\psi_1, . . . , \psi_N} = \ket{\varphi,\psi_1, . . . , \psi_N}
	\end{equation}
	
	\item {Записать результат действия оператора уничтожения на вектор N-частичного состояния тождественных ферми-частиц $a(\varphi)\ket{\psi_1, . . . , \psi_N} = \ ?$}
	
	\begin{equation}
		a(\varphi)\ket{\psi_1, . . . , \psi_N} = \sum\limits_{k} (-1)^{k-1} \braket{\varphi|\psi_k} \ket{\psi_1,...,\psi_{k-1},\psi_{k+1},...,\psi_N}
	\end{equation}
	
	\item {Записать результат действия оператора уничтожения на вектор N-частичного состояния тождественных бозе-частиц $a(\varphi)\ket{\psi_1, . . . , \psi_N} = \ ?$}
	
	\begin{equation}
		a(\varphi)\ket{\psi_1, . . . , \psi_N} = \sum\limits_{k} \braket{\varphi|\psi_k} \ket{\psi_1,...,\psi_{k-1},\psi_{k+1},...,\psi_N}
	\end{equation}
	
	\item {Чему равен коммутатор для системы тождественных бозе-частиц \\$a^+(\varphi_1)a^+(\varphi_2) - a^+(\varphi_2)a^+(\varphi_1)?$}
	
	\begin{equation}
		[a^+(\varphi_1), a^+(\varphi_2)] = 0
	\end{equation}
	
	\item {Чему равен коммутатор для системы тождественных бозе-частиц \\$a(\varphi_1)a(\varphi_2) - a(\varphi_2)a(\varphi_1)?$}
	
	\begin{equation}
		[a(\varphi_1), a(\varphi_2)] = 0
	\end{equation}		
	
	\item {Чему равен коммутатор для системы тождественных бозе-частиц \\$a(\varphi_1)a^+(\varphi_2) - a^+(\varphi_2)a(\varphi_1)?$}
	
	\begin{equation}
		[a(\varphi_1), a^+(\varphi_2)] = \braket{\varphi_1|\varphi_2}
	\end{equation}
	
	\item {Чему равен антикоммутатор для системы тождественных ферми-частиц \\$a^+(\varphi_1)a^+(\varphi_2) + a^+(\varphi_2)a^+(\varphi_1)?$}
	
	\begin{equation}
		a^+(\varphi_1)a^+(\varphi_2) + a^+(\varphi_2)a^+(\varphi_1) = 0
	\end{equation}
	
	\item {Чему равен антикоммутатор для системы тождественных ферми-частиц \\$a(\varphi_1)a(\varphi_2) + a(\varphi_2)a(\varphi_1)?$}
	
	\begin{equation}
		a(\varphi_1)a(\varphi_2) + a(\varphi_2)a(\varphi_1) = 0
	\end{equation}
	
	\item {Чему равен антикоммутатор для системы тождественных ферми-частиц \\$a(\varphi_1)a^+(\varphi_2) + a^+(\varphi_2)a(\varphi_1)?$}
	
	\begin{equation}
		a(\varphi_1)a^+(\varphi_2) + a^+(\varphi_2)a(\varphi_1) = \braket{\varphi_1|\varphi_2}
	\end{equation}
	
	\item {Чему равен антикоммутатор для системы тождественных ферми-частиц \\ $a_{\textbf{p}_1,\sigma} a^+_{\textbf{p}_2,\sigma'} + a^+_{\textbf{p}_2,\sigma'}a_{\textbf{p}_1,\sigma}$, где ${\textbf{p},\sigma}$ – состояние с определенным импульсом и проекцией спина?}
	
	\begin{equation}
		a_{\textbf{p}_1,\sigma} a^+_{\textbf{p}_2,\sigma'} + a^+_{\textbf{p}_2,\sigma'}a_{\textbf{p}_1,\sigma} = \delta(\textbf{p}_1 - \textbf{p}_2)\delta_{\sigma\sigma'}
	\end{equation}
	
	\item {Чему равен коммутатор для системы тождественных бозе-частиц \\ $a_{\textbf{p}_1,\sigma} a^+_{\textbf{p}_2,\sigma'} - a^+_{\textbf{p}_2,\sigma'}a_{\textbf{p}_1,\sigma}$, где ${\textbf{p},\sigma}$ – состояние с определенным импульсом и проекцией спина?}
	
	\begin{equation}
		a_{\textbf{p}_1,\sigma} a^+_{\textbf{p}_2,\sigma'} - a^+_{\textbf{p}_2,\sigma'}a_{\textbf{p}_1,\sigma} = \delta(\textbf{p}_1 - \textbf{p}_2)\delta_{\sigma\sigma'}
	\end{equation}
	
	\item {Для системы тождественных бозе-частиц определить результат действия оператора рождения на вектор состояния в пространстве чисел заполнения $a^+_\beta \ket{n_1, n_2, ...} = \ ?$}
	
	\begin{equation}
		a^+_\beta \ket{n_1, n_2, ...} = \sqrt{n_\beta + 1} \ket{n_1, n_2, ..., n_\beta +1,...}
	\end{equation}
	
	\item {Для системы тождественных бозе-частиц определить результат действия оператора уничтожения на вектор состояния в пространстве чисел заполнения $a_\beta \ket{n_1, n_2, ...} = \ ?$}
	
	\begin{equation}
		a_\beta \ket{n_1, n_2, ...} = \sqrt{n_\beta} \ket{n_1, n_2, ..., n_\beta -1,...}
	\end{equation}
	
	\item {Какой смысл имеет оператор $a^+_\beta a_\beta$, действующий в пространстве чисел заполнения?}
	Этот оператор измеряет число частиц в состоянии $\beta$:
	\begin{equation}
		a^+_\beta a_\beta \ket{n_1, n_2, ...} = n_\beta \ket{n_1, n_2, ...}
	\end{equation}
	
	\item {Чему равно действие оператора $\sum\limits_{\beta}a^+_\beta a_\beta$ на вектор состояния в пространстве чисел заполнения?}
	
	Этот оператор подсчитывает число частиц в системе:
	\begin{equation}
		\sum\limits_{\beta}a^+_\beta a_\beta \ket{n_1, n_2, ...} =\sum\limits_{\beta} n_\beta \ket{n_1, n_2, ...} = N\ket{n_1, n_2, ...}
	\end{equation}
	\item {Для системы тождественных ферми-частиц определить результат действия оператора рождения на вектор состояния в пространстве чисел заполнения $a^+_\beta \ket{n_1, n_2, ...} = \ ?$}
	
	\begin{equation}
		a^+_\beta \ket{n_1, n_2, ...} = \left\{ 
  \begin{array}{l l}
    0 & \quad n_\beta = 1\\
    \ket{n_1, n_2, ..., n_{\beta -1}, 1 , n_{\beta -1},...} & \quad n_\beta = 0
  \end{array} \right.
	\end{equation}
	
	\item {Для системы тождественных ферми-частиц определить результат действия оператора уничтожения на вектор состояния в пространстве чисел заполнения $a_\beta \ket{n_1, n_2, ...} = \ ?$}
	
	\begin{equation}
		a_\beta \ket{n_1, n_2, ...} = \left\{ 
  \begin{array}{l l}
    0 & \quad n_\beta = 0\\
    \ket{n_1, n_2, ..., n_{\beta -1}, 0 , n_{\beta -1},...} & \quad n_\beta = 1
  \end{array} \right.
	\end{equation}
	
	\item {Для системы тождественных ферми-частиц выразить оператор $a_\alpha a_\alpha^+$ через оператор числа частиц $\hat{N}_\alpha$ в состоянии $\ket{\alpha}$.}
	
	\begin{equation}
		a_\alpha a_\alpha^+  = 1 - \hat{N}_\alpha
	\end{equation}
	
	\item {Для системы бозе-частиц определить среднее значение оператора $a_\alpha^+ a_\alpha$.}
	
	\begin{equation}
		\braket{\psi| a_\alpha^+ a_\alpha|\psi} = n_\alpha
	\end{equation}
	
	\item {Для системы бозе-частиц определить среднее значение оператора $a_\alpha a_\alpha^+$.}
	
	\begin{equation}
		\braket{\psi| a_\alpha a_\alpha^+|\psi} = n_\alpha + 1
	\end{equation}
	
	\item {Для системы ферми-частиц определить среднее значение оператора $a_\alpha^+ a_\alpha$.}
	
	\begin{equation}
		\braket{\psi| a_\alpha^+ a_\alpha|\psi} = n_\alpha
	\end{equation}
	
	\item {Для системы ферми-частиц определить среднее значение оператора $a_\alpha^+ a_\alpha$.}
	
	\begin{equation}
		\braket{\psi| a_\alpha a_\alpha^+|\psi} = 1 - n_\alpha
	\end{equation}	
	
	\item {Чему равен коммутатор полевых операторов для системы тождественных бозе-частиц $\hat{\psi}(\textbf{r}')\hat{\psi}^+(\textbf{r})-\hat{\psi}^+(\textbf{r})\hat{\psi}(\textbf{r}')$?}
	
	\begin{equation}
		\hat{\psi}(\textbf{r}')\hat{\psi}^+(\textbf{r})-\hat{\psi}^+(\textbf{r})\hat{\psi}(\textbf{r}') = \delta(\textbf{r}' - \textbf{r})
	\end{equation}
	
	\item {Чему равен антикоммутатор полевых операторов для системы тождественных ферми-частиц $\hat{\psi}(\textbf{r}')\hat{\psi}^+(\textbf{r}) + \hat{\psi}^+(\textbf{r})\hat{\psi}(\textbf{r}')$}
	
	\begin{equation}
		\hat{\psi}(\textbf{r}')\hat{\psi}^+(\textbf{r}) + \hat{\psi}^+(\textbf{r})\hat{\psi}(\textbf{r}') = \delta(\textbf{r}' - \textbf{r})
	\end{equation}
	
	\item {Записать полевой оператор $\hat{\psi}(\textbf{r)}$ через операторы уничтожения $a_n$ состояний дискретного базиса $\varphi_n(\textbf{r})$.}

	\begin{equation}
		\hat{\psi}(\textbf{r)} = \sum\limits_{n} \varphi_n(\textbf{r}) a_n
	\end{equation}		
	
	\item {Записать полевой оператор $\hat{\psi}^+(\textbf{r)}$ через операторы рождения $a_n^+$ состояний дискретного базиса $\varphi_n(\textbf{r})$.}		
	
	\begin{equation}
		\hat{\psi}^+(\textbf{r)} = \sum\limits_{n} \varphi_n^*(\textbf{r}) a_n^+
	\end{equation}	
	
	\item {Выразить оператор плотности частиц $\hat{\rho}(\textbf{r})$ через полевые операторы.}	
	\begin{equation}
		\hat{\rho}(\textbf{r}) = \hat{\psi}^+(\textbf{r)} \hat{\psi}(\textbf{r)}
	\end{equation}
	
	\item {Выразить оператор числа частиц $\hat{N}$ через полевые операторы.}
	
	\begin{equation}
		\hat{N} = \int d\textbf{r}\hat{\psi}^+(\textbf{r)} \hat{\psi}(\textbf{r)}
	\end{equation}
	
	\item {Записать полевой оператор $\hat{\psi}(\textbf{r)}$ через операторы уничтожения $a_{\textbf{p},\sigma}$ в базисе состояний свободных частиц.}
	
	\begin{equation}
		\hat{\psi}(\textbf{r)} = \int \frac{d\textbf{p}}{(2\pi\hbar)^\frac{3}{2}} e^{i\frac{\textbf{pr}}{\hbar}}a_\textbf{p}
	\end{equation}
	
	\item {Записать полевой оператор $\hat{\psi}^+(\textbf{r)}$ через операторы уничтожения $a^+_{\textbf{p},\sigma}$ в базисе состояний свободных частиц.}
	
	\begin{equation}
		\hat{\psi}^+(\textbf{r)} = \int \frac{d\textbf{p}}{(2\pi\hbar)^\frac{3}{2}} e^{-i\frac{\textbf{pr}}{\hbar}}a^+_\textbf{p}
	\end{equation}
	
	\item {Записать операторы уничтожения $a_{\textbf{p},\sigma}$ через полевые операторы $\hat{\psi}(\textbf{r)}$.}
	
	\begin{gather}
		a_\textbf{p} = \int \frac{d\textbf{p}}{(2\pi\hbar)^\frac{3}{2}} e^{-i\frac{\textbf{pr}}{\hbar}} \hat{\psi}(\textbf{r)} \\
		a^+_\textbf{p} = \int \frac{d\textbf{p}}{(2\pi\hbar)^\frac{3}{2}} e^{i\frac{\textbf{pr}}{\hbar}} \hat{\psi}^+(\textbf{r)}
	\end{gather}
	
	\item {Записать одночастичный оператор в представлении чисел заполнения.}
	
	\begin{equation}
		\hat{f}^{(1)} = \sum\limits_{m,n} f_{mn} a^+_m a_n
	\end{equation}
	
	\item {Записать оператор двухчастичного взаимодействия в представлении чисел заполнения.}
	
	\begin{equation}
		\hat{V}^{(2)} = \frac{1}{2!}\sum\limits_{m,m',n,n'} V_{mn,m'n'} a^+_m a^+_n a_{n'}a_{m'}
	\end{equation}
	
	\item {Записать гамильтониан системы тождественных частиц в представлении чисел заполнения
в собственном базисе.}

	\begin{equation}
		\hat{H} = A + \sum\limits_{\textbf{k}}\varepsilon(\textbf{k}) a^+_{\textbf{k}}a_{\textbf{k}} + \sum\limits_{\textbf{k},\textbf{k}'}C_{\textbf{k},\textbf{k}'} a^+_{\textbf{k}}a_{\textbf{k}'} + \frac{1}{2}\sum\limits_{\textbf{k}_1,\textbf{k}_2,\textbf{k}_3,\textbf{k}_4}D_{\textbf{k}_1,\textbf{k}_2,\textbf{k}_3,\textbf{k}_4} a^+_{\textbf{k}_1}a^+_{\textbf{k}_2}a_{\textbf{k}_3}a_{\textbf{k}_4}
	\end{equation}
\end{enumerate}

\subsection*{Магнитные взаимодействия}

\begin{enumerate}
	\item {Для заряженной частицы записать связь оператора кинематического импульса с обобщенным.}
	\begin{equation}
		\boldsymbol{\hat{\mathcal{P}}=\hat{p}+\frac{e}{c}\hat{A}}
	\end{equation}
	
	\item {Связь оператора магнитного момента со спином частицы.}
	\begin{gather}
		\hat{\boldsymbol{\mu}}=\mu\frac{\hat{\boldsymbol{s}}}{s}=\hbar\gamma\hat{\boldsymbol{s}}=g\mu_{0}\boldsymbol{\hat{s}}\\
		\gamma=\frac{\mu}{\hbar s},\mu_{0}=\frac{e\hbar}{2m_{e}c}
	\end{gather}
	
	\item {Записать гамильтониан Паули системы заряженных частиц.}
	\begin{equation}
		\hat{\boldsymbol{H}}=\sum_{\alpha}\big\{\frac{1}{2m_{\alpha}}\big(\boldsymbol{\hat{\mathcal{P}_{\alpha}}-\frac{e_{\alpha}}{c}\hat{A_{\alpha}}\big)^{2}}+e_{\alpha}\varphi_{\alpha}-g_{\alpha}\mu_{0\alpha}(\hat{s_{\alpha},}\hat{\mathcal{H}_{\alpha})}\big\}
	\end{equation}
	
	\item {Преобразование волновой функции частицы при калибровочном преобразовании потенциалов.}
	\begin{gather}
		\psi(\boldsymbol{r},t)=\exp\{-i\frac{f(\boldsymbol{r},t)}{\hbar c}\}\psi'(\boldsymbol{r},t)\\
		\varphi'=\varphi-\frac{1}{c}\frac{\partial f}{\partial t}\\
		\boldsymbol{A'=A+\nabla}f
	\end{gather}
	
	\item {Записать оператор взаимодействия системы заряженных частиц с электромагнитным полем.}
	\begin{equation}
		\hat{V}=\sum_{\alpha}\big\{ e_{\alpha}\varphi_{\alpha}-\frac{e_{\alpha}}{m_{\alpha}c}\hat{\mathcal{\mathcal{P}_{\alpha}}}\hat{\boldsymbol{A}_{\alpha}}+\frac{e_{\alpha}^{2}}{2m_{\alpha}c^{2}}\hat{\boldsymbol{A}^{2}}-g_{\alpha}\mu_{0\alpha}(\hat{s_{\alpha},}\hat{\mathcal{H}_{\alpha})}\big\}
	\end{equation}
	
	\item {Какой вид имеет оператор взаимодействия системы заряженных частиц с однородным магнитным полем?}
	\begin{gather}
		\boldsymbol{A}=\frac{1}{2}[\mathcal{\boldsymbol{H}} \times \boldsymbol{r}]\\
		\hat{V}=\mu_{0}\mathcal{\boldsymbol{H}}(\boldsymbol{L+2S)}+\sum_{\alpha}\frac{e_{\alpha}^{2}}{2m_{\alpha}c^{2}}\hat{\boldsymbol{A}^{2}}
	\end{gather}
	
	\item {В каком виде можно представить оператор спин-орбитального взаимодействия заряженной частицы в центральном поле? Каков порядок этого взаимодействия по сравнению с атомной?}
	\begin{gather}
		\hat{V_{s-o}}=-(\hat{\boldsymbol{\mu}}\hat{\boldsymbol{\mathcal{H}}})=A(r)(\boldsymbol{s,l})\\
		A\sim\alpha^{2}E_{0}
	\end{gather}
	
	\item {Оценить порядок величины сверхтонкого взаимодействия в атоме водорода.}
	\begin{equation}
		V_{dd}\sim\frac{\mu_{0B}\mu_{N}}{r^{3}}\sim\frac{e^{2}\hbar^{2}}{mMc^{2}}\frac{m^{3}e^{6}}{\hbar^{6}}\sim\frac{m}{M}\alpha^{2}E_{0}
	\end{equation}
	
	\item {Как определяется гамильтониан свободного электромагнитного поля?}
	\begin{equation}
		\hat{H}=\sum_{\alpha,\boldsymbol{k}}\hbar\omega_{k}(a_{\boldsymbol{k},\alpha}^{+}a_{\boldsymbol{k},\alpha}+\frac{1}{2})
	\end{equation}
	
	\item {Чему равен коммутатор операторов рождения и уничтожения для свободного электромагнитного поля $[a_{\textbf{k},\alpha}, a^+_{\textbf{k},\beta}] = \ ?$}
	\begin{equation}
		[a_{k,\alpha}\text{, }a^+_{k,\beta}]=\delta_{\alpha\beta}
	\end{equation}
	
	\item {Запишите энергетически спектр свободного электромагнитного поля.}
	\begin{equation}
		\sum_{\alpha,\boldsymbol{\textbf{k}}}\hbar\omega_{k}(n_{\textbf{k},\alpha}+\frac{1}{2})
	\end{equation}
	
	\item {Как можно определить произвольное состояние свободного электромагнитного поля из основного (вакуума)?}
	\begin{equation}
		\ket{\psi}=\prod_{i}\frac{(a_{\boldsymbol{k}_{i},\alpha_{i}}^{+})^{n_{k_{i},\alpha_{i}}}}{\sqrt{n_{k_{i},\alpha_{i}}!}}\ket{0}
	\end{equation}
	
	\item {Оценить порядок величины электрического дипольного взаимодействия системы зарядов со свободным электромагнитным полем.}
	\begin{equation}
		d\sim ea_{0}
	\end{equation}
	
	\item {Как определяется время жизни системы зарядов в возбужденном состоянии в электрическом дипольном приближении?}
	\begin{equation}
		\tau\sim\frac{3\hbar c^{3}}{4\omega_{if}^{3}|\boldsymbol{d_{if}}|^{2}}
	\end{equation}
	
	\item {Оценить время жизни свободного атома в возбужденном состоянии в электрическом дипольном приближении.}
	\begin{equation}
		w\sim\frac{m^{3}e^{12}}{\hbar^{9}c^{3}}(ea_{0})^{2}
	\end{equation}
\end{enumerate}

\subsection*{Матрица плотности}
\begin{enumerate}
	\item {Как, зная матрицу плотности системы, определить среднее значения оператора физической величины?}

	\begin{equation}
		f = \text{Tr} \hat{f}\hat{\rho}
	\end{equation}
	
	\item {В каком виде всегда можно представить матрицу плотности чистого состояния?}

	\begin{equation}
		\hat{\rho}_c = \ket{\psi}\bra{\psi}
	\end{equation}
	
	\item {Для матрицы плотности $\text{Tr} \rho = \ ?$}
	
	\begin{equation}
		\text{Tr}\rho = 1
	\end{equation}
	
	\item {В каком случае Tr$\rho^2 = 1$?}
	
	\begin{equation}
		\text{Tr} \rho^2 = \sum\limits_{a,n}w^2_a |c_n^a|^2 = \sum\limits_{a} w^2_a = 1 \iff w_a = \delta_{aa'}
	\end{equation}
	
	\item {В каком виде всегда можно представить матрицу плотности смешанного состояния?}
	
	\begin{equation}
		\hat{\rho} = \sum\limits_{a} w_a \ket{\psi_a}\bra{\psi_a}
	\end{equation}
	
	\item {Для матрицы плотности определить связь $\rho$ и $\rho^+$.}
	
	\begin{equation}
		\rho = \rho^+
	\end{equation}
	
	\item {Какой физический смысл имеют диагональные элементы матрицы плотности?}
	
	Диагональные элементы определяют вероятности обнаружения системы в данном собственном состоянии.
	
	\item {Записать уравнение Лиувилля для эволюции матрицы плотности.}
	
	\begin{equation}
		\frac{\partial \hat{\rho}}{\partial t} = -\frac{i}{\hbar} [\hat{H}, \hat{\rho]}
	\end{equation}
	
	\item {Записать формальное решение уравнения Лиувилля с помощью оператора эволюции.}
	
	\begin{equation}
		\hat{\rho}(t) = \hat{U}(t)\hat{\rho}_0 \hat{U}^+(t)
	\end{equation}
	
	\item {Записать уравнение Лиувилля для матрицы плотности в матричной форме.}
	
	\begin{equation}
		i\hbar \frac{\partial \rho_{nk}}{\partial t} = \sum\limits_{m} H_{nm}\rho_{mk} - \rho_{nm}H_{mk}
	\end{equation}
	
	\item {Определение энтропии системы с помощью матрицы плотности.}
	
	\begin{equation}
		S(\rho) = -\sum\limits_{\nu} \rho_\nu \ln \rho_\nu
	\end{equation}
	где $\rho_\nu$ - собственные значения оператора $\rho$.
	
	\item {Какой вид имеет равновесная матрица плотности малого канонического ансамбля?}
	
	\begin{equation}
		\hat{\rho} = \frac{e^{-\beta\hat{H}}}{\text{Tr}e^{-\beta\hat{H}}}
	\end{equation}
	
	\item {Зная равновесную матрицу плотности малого канонического ансамбля, записать определение его энергии.}
	
	\begin{equation}
		E = - \frac{\partial}{\partial \beta} \ln \text{Tr}e^{-\beta\hat{H}}
	\end{equation}
\end{enumerate}

\end{document}
